
\chapter{Built-in Plugins}
\label{ch:BuiltinPlugins}

\section{Angle Measure Plugin}
\label{sec:plugins:AngleMeasure}

%\url{http://porpoisehead.net/images/plugin-angle-measure.jpg}

The Angle Measure plugin is a small tool which is used to measure the
angular distance between two points on the sky. *goes misty eyed* I
recall measuring the size of the Cassini Division when I was a student.
It was not the high academic glamor one might expect... It was cloudy...
It was rainy... The observatory lab had some old scopes set up at one
end, pointing at a \emph{photograph} of Saturn at the other end of the
lab. We measured. We calculated. We wished we were in Hawaii. A picture
is worth a thousand words.

\subsection{Using plugin}
\label{sec:plugins:AngleMeasure:using}

\begin{enumerate}
\item
  Enable the tool by clicking the tool-bar button, or by pressing
  \textbf{control-A}. A message will appear at the bottom of the screen
  to tell you that the tool is active.
\item
  Drag a line from the first point to the second point using the left
  mouse button
\item
  To clear the measurement, click the right mouse button
\item
  To deactivate the angle measure tool, press the tool-bar button again,
  or press \textbf{control-A} on the keyboard.
\end{enumerate}

\section{Bright Novae Plugin}
\label{sec:plugins:BrightNovae}

The Bright Novae plugin provides visualization of some bright novae in
the Milky Way galaxy (Fig.~\ref{fig:NovaCygni1975}).

%Example (\href{http://en.wikipedia.org/wiki/V1500_Cygni}{\textbf{Nova
%Cygni 1975}}, also known as \textbf{V1500 Cyg}):

\begin{figure}[h]
%\centering\includegraphics[scale=0.6]{NovaCygni1975wiki.jpg}
\includegraphics[width=\textwidth]{NovaCygni1975wiki.jpg}
\label{fig:NovaCygni1975}
\caption{Nova Cygni 1975 (also known as \textbf{V1500 Cyg})}
\end{figure}

\subsection{Using the Bright Novae plugin}
\label{sec:plugins:BrighrNovae:using}

\begin{enumerate}
\item Enable the tool by clicking the tool-bar button ``Load at startup''
\item Set date and time (30 August 1975 year for \emph{Nova Cygni 1975} as example\footnote{\url{http://en.wikipedia.org/wiki/V1500_Cygni}})
\end{enumerate}

\subsection{Section \big[Novae\big] in config.ini file}
\label{sec:plugins:BrightNovae:config}

You can edit \file{config.ini} file by yourself for changes of the
settings for the Bright Novae plugin -- just make it carefully!

\begin{longtabu} to \textwidth {l|l|X}
\toprule
\emph{ID} & \emph{Type} & \emph{Description}\tabularnewline
\midrule
last\_update & string & Date and time of last update\tabularnewline
\midrule
update\_frequency\_days & int & Frequency of updates, in days\tabularnewline
\midrule
updates\_enable & bool & Enable updates of bright novae catalog from Internet \tabularnewline
\midrule
url & string & URL of bright novae catalog \tabularnewline
\bottomrule
\end{longtabu}

\subsection{Format of bright novae catalog}
\label{sec:plugins:BrightNovae:format}

To add a new nova, open a new line after line 5 and paste the following,
note commas and brackets, they are important:

\begin{configfile}
"Nova designation":
{
    "name": "name of nova",
    "type": "type of nova",
    "maxMagnitude": value of maximal visual magnitude,
    "minMagnitude": value of minimal visual magnitude,
    "peakJD": JD for maximal visual magnitude,
    "m2": Time to decline by 2mag from maximum (in days),
    "m3": Time to decline by 3mag from maximum (in days),
    "m6": Time to decline by 6mag from maximum (in days),
    "m9": Time to decline by 9mag from maximum (in days),
    "distance": value of distance between nova and 
                Earth (in thousand of Light Years),
    "RA": "Right ascension (J2000)",
    "Dec": "Declination (J2000)"
},

\end{configfile}

\newpage
\noindent For example, the record for \textbf{Nova Cygni 1975} (\textbf{V1500 Cyg}) looks like:
\begin{configfile}
"V1500 Cyg":
{
    "name": "Nova Cygni 1975",
    "type": "NA",
    "maxMagnitude": 1.69,
    "minMagnitude": 21,
    "peakJD": 2442655,
    "m2": 2,
    "m3": 4,
    "m6": 32,
    "m9": 263
    "distance": 6.36,
    "RA": "21h11m36.6s",
    "Dec": "48d09m02s"
},
\end{configfile}

\subsection{Light curves}
\label{sec:plugins:BrightNovae:lightcurves}

This plugin uses a very simple model for calculation of light curves for
novae stars. This model is based on time for decay by \emph{N}
magnitudes from the maximum value, where \emph{N} is 2, 3, 6 and 9. If a
nova has no values for decay of magnitude then this plugin will use
generalized values for it.

\section{Compass Marks Plugin}
\label{sec:plugins:CompassMarks}

%\url{http://porpoisehead.net/images/plugin-compass-marks.jpg}

Stellarium helps the user get their bearings using the cardinal point
feature - the North, South, East and West markers on the horizon.
Compass Marks takes this idea and extends it to add markings every few
degrees along the horizon, and includes compass bearing values in
degrees.

\subsection{Using the plugin}
\label{sec:plugins:CompassMarks:using}

There is a tool bar button for toggling the compass markings, or you can
press \key{control-C}.

Note that when you first enable compass marks, the cardinal points will
be turned off. You can have both active at once, but there is a small
bug which means you have to press \key{Q} \emph{two times} to
re-enable cardinal points after enabling the compass markings.

\section{Remote Control Plugin}
\label{sec:plugins:RemoteControl}

The Remote Control plugin was developed in 2015 during the 
\href{http://sophia.estec.esa.int/socis/}{ESA Summer of Code in Space} 
initiative. It enables the user to control Stellarium through an external web 
interface using a standard web browser like Firefox or Chrome, instead of using 
the main GUI. This works on the same computer Stellarium runs as well as over 
the network. Even more, multiple "remote controls" can access the same 
Stellarium instance at the same time, without getting in the way of each other. 
Much of the functionality the main interface provides is already available 
through it, and it is still getting extended.

The plugin may be useful for presentation scenarios, hiding the GUI from the 
audience and allowing the presenter to change settings on a separate monitor 
without showing distracting dialog windows. It also allows to start and stop 
scripts remotely. Because the web interface can be customized (or completely 
replaced) with some knowledge of HTML, CSS and JavaScript, another possibility 
is a kiosk mode, where untrusted users can execute a variety of predefined 
actions (like starting recorded tours) without having access to all Stellarium 
settings. The web API can also be accessed directly (without using a browser 
and the HTML interface), allowing control of Stellarium with external programs 
and scripts using HTTP calls like with the tools \file{wget} and \file{curl}.

\subsection{Using the plugin}
\label{sec:plugins:RemoteControl:using}

\begin{figure}[h]
\centering\includegraphics[width=\columnwidth]{remote_web}
\caption{The default remote control web interface}
\end{figure}

After enabling the plugin, you can set it up through the configuration dialog. 
When ``Enable automatically on startup'' is checked (it is by default), the web 
server is automatically started whenever Stellarium starts. You can also 
manually start/stop the server using the ``Server enabled'' checkbox and the 
button \includegraphics[scale=0.5]{remote} in the toolbar.

The plugin starts a HTTP server on the specified port. The default port is 
8090, so you can try to reach the remote control after enabling it by starting 
a browser on the same computer and entering \url{http://localhost:8090} in the 
address bar. When trying to access the remote control from another computer, 
you need the IP address or the hostname of the server on which Stellarium runs. 
The plugin shows the locally detected address, but depending on your network or 
if you need external access you might need to use a different one 
--- contact your network administrator if you need help with that.

The access to the remote control may optionally be restricted with a simple 
password.

\textbf{Warning:} \emph{currently no network encryption is used, meaning that 
an attacker having access to your network can easily find out the password by 
waiting for a user entering it. Access from the Internet to the 
plugin should generally be restricted, except if countermeasures such as VPN 
usage are taken! If you are in a home network using NAT (network access 
translation), this should be enough for basic security except if port 
forwarding or a DMZ is configured.}

If you are familiar with the main Stellarium interface, you should easily find 
your way around the web interface. Tabs at the top allow access to 
different settings and controls. The remote control automatically uses the 
same language as set in the main program.

The contents of the various tabs:
\begin{description}
\item[Main] Contains the time controls and most of the buttons of the 
main bottom toolbar. An additional control allows moving the view like when 
dragging the mouse or using the arrow keys in Stellarium, and a slider enables 
the changing of the field of view. There are also buttons to quickly execute 
time jumps using the commonly used astronomical time intervals.
\item[Selection] Allows searching and selecting objects like in the GUI dialog. 
SIMBAD search is also supported. Quick select buttons are available for the 
primary solar system objects. It also displays the information text for current 
selection.
\item[Sky] Settings related to the sky display as shown in the ``View'' dialog 
as shown in \autoref{sky-tab}.
\item[DSO] The deep-sky object catalog, filter and display settings like in 
\autoref{gui-for-manage-by-deep-sky-objects}.
\item[Landscape] Changing and configuring the background landscape, see 
\autoref{landscape-tab}
\item[Actions and scripts] Lists all registered actions, and allows starting 
and stopping of scripts (\autoref{ch:scripting}). If there is no button for the 
action you want in another tab, you can find all actions which can be 
configured as a keyboard shortcut (\autoref{the-configuration-window}) here.
\item[Location] Allows changing the location, like in 
\autoref{setting-your-location}. Custom location saving is currently not 
supported.
\item[Projection] Switch the projection method used, like \autoref{marking-tab}.
\end{description}

\subsection{Developer information}
\label{sec:plugins:RemoteControl:developer}

If you are a developer and would like to add functionality to the Remote 
Control API, customize the web interface or access the API through another 
program, further information can be found in the 
\href{http://stellarium.org/doc-plugins/head/}{plugin's developer 
documentation}.

\section{Text User Interface Plugin}
\label{sec:plugins:TUI}

%\url{http://porpoisehead.net/images/plugin-tui.jpg}

Older versions of Stellarium used to have a little menu system which was
controlled by the cursor keys. This was used primarily by planetarium
system operators to change settings, run scripts and so on. In the
0.10.x series, this function vanished as we totally re-designed the user
interface. This plugin re-implements the ``TUI'', as it was known. Full
list of the commands for the TUI plugin you can read in the section
\href{TUI_Commands}{TUI Commands}.

\subsection{Using the Text User Interface}
\label{sec:plugins:TUI:using}

\begin{enumerate}
\item
  Activate the text menu using the \key{M} key
\item
  Navigate the menu using the cursors keys.
\item
  To edit a value, press the right cursor until the value you wish to
  change it highlighted with \textgreater{} and \textless{} marks, e.g.\
  \textgreater{}3.142\textless{}. Then press the up and down cursors to
  change the value. You may also type in a new value with the other keys
  on the keyboard.
\end{enumerate}

\subsection{TUI Commands}
\label{sec:plugins:TUI:commands}
\begin{longtabu} to \textwidth {l|l|X}
\toprule
1 & Set Location & (menu group)\\
\midrule
1.1 & Latitude & Set the latitude of the observer in degrees\\
\midrule
1.2 & Longitude & Set the longitude of the observer in degrees\\
\midrule
1.3 & Altitude (m) & Set the altitude of the observer in meters\\
\midrule
1.4 & Solar System Body & Select the solar system body on which the observer is\\
\midrule
2 & Set Time & (menu group)\\
\midrule
2.1 & Sky Time & Set the time and date for which Stellarium will generate the view\\
\midrule
2.2 & Set Time Zone & Set the time zone. Zones are split into continent
or region, and then by city or province\\
\midrule
2.3 & Days keys & The setting ``Calendar'' makes the - = {[} {]} and
keys change the date value by calendar days (multiples of 24 hours). The
setting ``Sidereal day'' changes these keys to change the date by sidereal days\\
\midrule
2.4 & Preset Sky Time & Select the time which Stellarium starts with (if
the ``Sky Time At Start-up'' setting is ``Preset Time''\\
\midrule
2.5 & Sky Time At Start-up & The setting ``Actual Time'' sets
Stellarium's time to the computer clock when Stellarium runs. The
setting ``Preset Time'' selects a time set in menu item ``Preset Sky Time''\\
\midrule
2.6 & Time Display Format & Change how Stellarium formats time values.
``system default'' takes the format from the computer settings, or it is
possible to select 24 hour or 12 hour clock modes\\
\midrule
2.7 & Date Display Format & Change how Stellarium formats date values.
``system default'' takes the format from the computer settings, or it is
possible to select ``yyyymmdd'', ``ddmmyyyy'' or ``mmddyyyy'' modes\\
\midrule
3 & General & (menu group)\\
\midrule
3.1 & Sky Culture & Select the sky culture to use (changes constellation lines, names, artwork)\\
\midrule
3.2 & Sky Language & Change the language used to describe objects in the sky\\
\midrule
4 & Stars & (menu group)\\
\midrule
4.1 & Show & Turn on/off star rendering\\
\midrule
4.2 & Star Magnitude Multiplier & Can be used to change the brightness
of the stars which are visible at a given zoom level. This may be used
to simulate local seeing conditions - the lower the value, the less
stars will be visible\\
\midrule
4.3 & Maximum Magnitude to Label & Changes how many stars get labelled
according to their apparent magnitude (if star labels are turned on)\\
\midrule
4.4 & Twinkling & Sets how strong the star twinkling effect is - zero is
off, the higher the value the more the stars will
twinkle.\\
\midrule
5 & Colors & (menu group)\\
\midrule
5.1 & Constellation Lines & Changes the colour of the constellation lines\\
\midrule
5.2 & Constellation Names & Changes the colour of the labels used to name stars\\
\midrule
5.3 & Constellation Art Intensity & Changes the brightness of the constellation artconstellation art\\
\midrule
5.4 & Constellation Boundaries & Changes the colour of the constellation boundary lines\\
\midrule
5.5 & Cardinal Points & Changes the colour of the cardinal points markers\\
\midrule
5.6 & Planet Names & Changes the colour of the labels for planets\\
\midrule
5.7 & Planet Orbits & Changes the colour of the orbital guide lines for planets\\
\midrule
5.8 & Planet Trails & Changes the colour of the planet trails lines\\
\midrule
5.9 & Meridian Line & Changes the colour of the meridian line\\
\midrule
5.10 & Azimuthal Grid & Changes the colour of the lines and labels for the azimuthal grid\\
\midrule
5.11 & Equatorial Grid & Changes the colour of the lines and labels for the equatorial grid\\
\midrule
5.12 & Equator Line & Changes the colour of the equator line\\
\midrule
5.13 & Ecliptic Line & Changes the colour of the ecliptic line\\
\midrule
5.14 & Nebula Names & Changes the colour of the labels for nebulae\\
\midrule
5.15 & Nebula Circles & Changes the colour of the circles used to denote
the positions of nebulae (only when enabled int he configuration file,
note this feature is off by default)\\
\midrule
6 & Effects & (menu group)\\
\midrule
6.1 & Light Pollution Luminance & Changes the intensity of the light pollution simulation\\
\midrule
6.2 & Landscape & Used to select the landscape which Stellarium draws when ground drawing is enabled\\
\midrule
6.3 & Manual zoom & Changes the behaviour of the / and \textbackslash{}
keys. When set to ``No'', these keys zoom all the way to a level defined
by object type (auto zoom mode). When set to ``Yes'', these keys zoom in
and out a smaller amount and multiple presses are required\\
\midrule
6.4 & Object Sizing Rule & When set to ``Magnitude'', stars are drawn
with a size based on their apparent magnitude. When set to ``Point'' all
stars are drawn with the same size on the screen\\
\midrule
6.5 & Magnitude Sizing Multiplier & Changes the size of the stars when
``Object Sizing Rule'' is set to ``Magnitude''\\
\midrule
6.6 & Milky Way intensity & Changes the brightness of the Milky Way texture\\
\midrule
6.7 & Maximum Nebula Magnitude to Label & Changes the magnitude limit for labelling of nebulae\\
\midrule
6.8 & Zoom Duration & Sets the time for zoom operations to take (in seconds)\\
\midrule
6.9 & Cursor Timeout & Sets the number of seconds of mouse inactivity before the cursor vanishes\\
\midrule
6.10 & Setting Landscape Sets Location & If ``Yes'' then changing the
landscape will move the observer location to the location for that
landscape (if one is known). Setting this to ``No'' means the observer
location is not modified when the landscape is changed\\
\midrule
7 & Scripts & (menu group)\\
\midrule
7.1 & Local Script & Run a script from the scripts sub-directory of the
User Directory or Installation Directory (see section
\href{Advanced_Use\#Files_and_Directories}{Files and Directories})\\
\midrule
7.2 & CD/DVD Script & Run a script from a CD or DVD (only used in planetarium set-ups)\\
\midrule
8 & Administration & (menu group)\\
\midrule
8.1 & Load Default Configuration & Reset all settings according to the main configuration file\\
\midrule
8.2 & Save Current Configuration as Default & Save the current settings to the main configuration file\\
\midrule
8.3 & Shutdown & Quit Stellarium\\
\midrule
8.4 & Update me via Internet & Only used in planetarium set-ups\\
\midrule
8.5 & Set UI Locale & Change the language used for the user interface\\
\bottomrule
\end{longtabu}




%%% Local Variables: 
%%% mode: latex
%%% TeX-master: "guide"
%%% End: 

