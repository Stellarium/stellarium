%% Part of Stellarium User Guide
%% Status:
%% 2016-05-16 GZ new file. This file and its handling is so complicated, it deserves proper docs.
%% 2017-06-18 GZ for v0.16.0 we have split the config file. Adapted docs.
%% TODO: document the more obscure entries. Add links to the respective chapters.

%\chapterimage{chapter-t1-bg} % Chapter heading image

\section{Solar System Configuration Files}
\label{sec:ssystem.ini}

The files\footnote{Before v0.16, there was a single file, \file{ssystem.ini}, with slightly different rules.}
 \file{ssystem\_major.ini} and  \file{ssystem\_minor.ini}  
(in the \file{data/} subdirectory of the program directory, 
contain orbital and rotational data from which Stellarium configures the Solar
System objects. 

The minor bodies can be modified by placing a privately modified 
copy of \file{ssystem\_minor.ini} into your own
\file{data/} directory (see Chapter~\ref{sec:FilesAndDirectories}). 
You can edit the file either manually, or with the Solar System Editor plugin (see
Section~\ref{sec:plugins:SolarSystemEditor}). The private user file is automatically created by the Solar System Editor plugin.


%Deprecated parameters are marked by gray background. 
%Possible new parameters are marked by yellow background.
%% GZ This is no longer true. Reactivate somehow? 

Each object's data are described in its own section which is typically
named after the object name. Some section names (e.g.\ those using
diacriticals, spaces or other problematic characters) appear a bit mangled.

We list here examples for the major planets, larger moons with special
coordinate functions, minor moons with generic orbital elements, minor
planets and comets with elliptical and parabolic orbit elements.

All elements are stored in alphabetic order in the files, however it
makes more sense to present the elements in another structure which
better reflects the meaning. The actual order of the objects in the files, 
and order of entries inside an object section, is irrelevant.

\subsection{File ssystem\_major.ini}
\label{sec:ssystem.ini:major}


\subsubsection{Planet section}
\label{sec:ssystem.ini:Planet}

Example:
\begin{configfile}
[jupiter]
name=Jupiter
type=planet  
coord_func=jupiter_special

albedo=0.51
atmosphere=1
color=1., 0.983, 0.934
tex_map=jupiter.png #texture courtesy of Bj\xf6rn J\xf3nsson
halo=true
oblateness=0.064874
orbit_visualization_period=4331.87
parent=Sun
radius=71492
rot_equator_ascending_node=-22.203
rot_obliquity=2.222461
rot_periode=9.92491
rot_pole_de=64.49
rot_pole_ra=268.05
rot_rotation_offset=-1 # use JupiterGRS patch
\end{configfile}


where 
\begin{description}
\item[name] English name of the planet. May appear translated. 
\item[type] Mandatory for planets: \value{planet}
\item[parent]=Sun. The body which this object is running around. Default: Sun
\item[coord\_func] The planet positions are all computed with a dedicated function (VSOP or DE43x). 
\item[orbit\_visualization\_period] number of (earth) days for how
  long the orbit should be made visible. Typically Stellarium shows
  one orbit line. The orbit slowly drifts, however.

\item[atmosphere] (0 or 1) flag to indicate whether observer locations should enable atmosphere drawing.

\item[radius] Equator radius, km.
\item[oblateness] Flattening of the polar diameter. ($r_{pole}/r_{eq}$)
\item[albedo] total albedo (reflectivity) of the planet. Used for ``generic''
  magnitude computation, but the major planets usually have dedicated
  magnitude formulas, so the value is not evaluated unless you are
  observing from a location outside Earth.
\item[color] Used to colorize halo. At least one of the components should be 1.
\item[tex\_halo] File name of texture map for halo texture in
  \file{textures} folder. (Used when zoomed out far enough so that the
  actual sphere is not visible.)
\item[tex\_map] File name of texture map in \file{textures} folder.
\item[halo] Should be true to draw a halo (simple light disk) when object too small to draw a sphere.
\end{description}

\paragraph{Elements for the Physical Ephemeris}

\begin{description}
\item[rot\_equator\_ascending\_node] \emph{deprecated}
\item[rot\_obliquity] \emph{deprecated}
\item[rot\_pole\_ra]  constant of axis right ascension in ICRF, degrees
\item[rot\_pole\_ra1] change per century of axis right ascension in ICRF, degrees
\item[rot\_pole\_de] constant of axis declination in ICRF, degrees
\item[rot\_pole\_de1] change per century of axis declination in ICRF, degrees
\item[rot\_rotation\_offset] longitude of central meridian at J2000.0. 
                             The special value here indicates special treatment for the Great Red Spot.
\item[rot\_periode] Duration of one sidereal rotation, in earth hours.
\end{description}

rot\_equator\_ascending\_node and rot\_obliquity are the generic
values to specify the axis orientation with respect to the parent
object's equator, or the VSOP87 inertial plane in case of planets. However,
for planets the IAU has been using right ascension and declination of
the pole in the ICRF for decades now. The elements to use for planets
should therefore be the rot\_pole\_* values.

In many cases, the attitude/rotation formula is given like
\begin{eqnarray}
  \label{eq:PlanetOrientation}
  \alpha_0 &=& \mathtt{rot\_pole\_ra} + T * \mathtt{rot\_pole\_ra1}\\
  \delta_0 &=& \mathtt{rot\_pole\_de} + T * \mathtt{rot\_pole\_de1}\\
  W        &=& \mathtt{rot\_rotation\_offset} + d * \mathtt{rot\_periode}
\end{eqnarray}

These cases are covered by the data in \file{ssystem\_major.ini}. Other cases are dealt with in a different way.


\subsubsection{Moon section}
\label{sec:ssystem.ini:Moon}

All planet moons are defined only in \file{ssystem\_major.ini}.

Moons are special in that they orbit another planet. Therefore, the
rotational elements used to be specified relative to the equatorial plane of the
parent planet, and \texttt{orbit\_SemiMajorAxis} are specified in kilometers. 
However, current IAU reference material gives axis orientation
with right ascension and declination values for the pole in ICRF
coordinates, with some of them in motion. So again, if one of the
\texttt{rot\_pole\_...} values exist, we assume the current
standard. For more complicated motion, again some special functions
are applied.

%% TODO: CHANGE THIS WHEN AXES HAVE BEEN REMADE!

\begin{configfile}
[amalthea]
name=Amalthea
type=moon
parent=Jupiter
iau_moon_number=JV

coord_func=ell_orbit
orbit_AscendingNode=141.5521520794674
orbit_Eccentricity=0.006175744402949701
orbit_Epoch=2454619.50000
orbit_Inclination=0.3864576103404582
orbit_LongOfPericenter=245.4222355150120000
orbit_MeanLongitude=224.7924893552550000
orbit_Period=0.5016370462116355
orbit_SemiMajorAxis=181994.8658358799
orbit_visualization_period=0.5016370462116355

radius=73
rot_equator_ascending_node=213.7
rot_obliquity=15.5
rot_periode=12.039289109079252
rot_rotation_offset=235.50

albedo=0.06
color=1., 0.627, 0.492
halo=true
tex_map=amalthea.png
model=j5amalthea_MLfix.obj
\end{configfile}

where
\begin{description}
  \item[name] English name of planet moon. No number, just the name. May be translated.
  \item[type] \texttt{moon}

  \item[parent] English name of planet or parent body.
  \item[iau\_moon\_number] a short label (string) consisting of the planet's initial and the moon's Roman number in order of discovery.\newFeature{v0.16.1}

  \item[coord\_func] Must be \texttt{ell\_orbit} for moons with orbital elements given, or \texttt{<name>\_special} for 
  \item[orbit\_AscendingNode] $\Omega$
  \item[orbit\_Eccentricity] $e$
  \item[orbit\_Epoch] 
  \item[orbit\_Inclination] $i$ [degrees]
  \item[orbit\_LongOfPericenter] 
  \item[orbit\_MeanLongitude]
  \item[orbit\_Period] [days]
  \item[orbit\_SemiMajorAxis] $a$ [km]
  \item[orbit\_visualization\_period] [days] Defaults to \texttt{orbit\_Period} to show orbit as closed line. 
       This is in fact only useful in case of special functions for positioning where \texttt{orbit\_Period} is not given.

  \item[radius] Equator radius, km.
  \item[oblateness] Flattening of the polar diameter. ($r_{pole}/r_{eq}$)
  \item[rot\_equator\_ascending\_node]=213.7 TRY TO AVOID!
  \item[rot\_obliquity]=15.5                 TRY TO AVOID!

  \item[rot\_pole\_ra]  constant of axis right ascension in ICRF, degrees
  \item[rot\_pole\_ra1] change per century of axis right ascension in ICRF, degrees
  \item[rot\_pole\_de] constant of axis declination in ICRF, degrees
  \item[rot\_pole\_de1] change per century of axis declination in ICRF, degrees
  \item[rot\_rotation\_offset] longitude of central meridian at J2000.0. 
  \item[rot\_periode] Duration of one sidereal rotation, in earth hours. For moons in bound rotation 
       (which always show one face towards their parent planet), it is best to omit this value: it defaults to \texttt{orbit\_Period * 24}. 

  \item[albedo] total reflectivity [0..1]
  \item[color] for drawing halo (default: 1,1,1)
  \item[halo] [=true|false] to draw a simple diffuse dot when zoomed out. Default: true
  \item[tex\_map] name of spherical texture map. Many moons have been mapped by visiting spacecraft!
  \item[model] \newFeature{v0.16.0}(optional) name of a 3D model for a non-spherical body in the \file{model} subdirectory of the program directory. 
\end{description}



\subsubsection{Observers}
\label{sec:ssystem.ini:Observers}

Stellarium is great for excursions to the surface of any object with
known orbital elements.  Configuring a viewpoint away from a planet
requires a special kind of location.

\paragraph{Solar System Observer}
\label{sec:ssystem.ini:SolarSystemObserver}

The Solar System Observer has
been provided as a neutral view location high above the North pole of
the Solar System. It has been configured like this:
\begin{configfile}
[solar_system_observer]
name=Solar System Observer
parent=Sun
halo=false
hidden=true
coord_func=ell_orbit
orbit_Inclination=90
orbit_MeanLongitude=90
orbit_Period=70000000000
orbit_SemiMajorAxis=70000000000
rot_obliquity=90
type=observer
\end{configfile}
Note the inclination and mean longitude of 90 degrees, and that it is \texttt{hidden} and has no \texttt{halo}. 

\paragraph{Planet Observers}
\label{sec:ssystem.ini:PlanetObserver}

There are other ``observer'' locations for all planets \newFeature{v0.19.2} which have moons: Earth, Mars, Jupiter, Saturn, Uranus, and Neptune.
They have all been provided as neutral view locations above the Northern hemisphere of
the respective planet. For example:
\begin{configfile}
[earth_observer]
name=Earth Observer
parent=Earth
halo=false
hidden=true
coord_func=ell_orbit
orbit_Inclination=90
orbit_MeanLongitude=90
orbit_Period=100000000000
orbit_SemiMajorAxis=149600000
rot_obliquity=90
type=observer
\end{configfile}

The distance \texttt{orbit\_semiMajorAxis} has been configured with a
distance that resembles the semimajor axis of the respective planet:
when you want to compare the views from Earth and from the observer,
light time correction should be at least approximately equal.

\subsection{File ssystem\_minor.ini}
\label{sec:ssystem.ini:minor}

\subsubsection{Minor Planet section}
\label{sec:ssystem.ini:MinorPlanet}

\begin{configfile}
[4vesta]
type=asteroid  
minor_planet_number=4
name=Vesta

coord_func=comet_orbit
parent=Sun
orbit_ArgOfPericenter=151.19843
orbit_AscendingNode=103.85141
orbit_Eccentricity=0.0887401
orbit_Epoch=2457000.5
orbit_Inclination=7.14043
orbit_MeanAnomaly=20.86389
orbit_MeanMotion=0.27154465
orbit_SemiMajorAxis=2.3617933
orbit_visualization_period=1325.46

color=1., 1., 1.
halo=true
oblateness=0.0

albedo=0.423
radius=280
absolute_magnitude=3.2
slope_parameter=0.32
tex_map=vesta.png
model=4vesta_21_MLfix.obj
\end{configfile}

\texttt{type} can be \texttt{asteroid, dwarf planet, cubewano,
  plutino, scattered disc object, Oort cloud object}. With the
exception of Pluto (which is included in \file{ssystem\_major.ini} and cannot be changed), 
all positions for minor bodies are computed with the orbiting elements given in this way. 

Note that although all minor planets orbit the sun in an
elliptical orbit, the \texttt{coord\_func=comet\_orbit}. However, it
is more common for minor planets to specify epoch, mean anomaly and
mean motion.

Visual magnitude is modelled from absolute magnitude $H$ and slope parameter $G$.

Elements for rotational axis may be given just like for planets when
they are known. It is recommended to use the modern specification
(elements \texttt{rot\_pole\_...}).

A few asteroids have been visited by spacecraft, \newFeature{v0.16.0}
and for many other asteroids visual observations of stellar occultations 
by asteroids and light curve measurements have enabled researchers to derive 
3D shape models of asteroids. If a model is available in the \file{models} 
subdirectory of the program directory, this can be configured with a \texttt{model} entry.

\subsubsection{Comet section}
\label{sec:ssystem.ini:Comet}

Comets are tiny, but their close approaches to the
major planets cause fast changes in their orbital elements, so that
each apparition should be specified with a dedicated section in
\file{ssystem\_minor.ini}. Outgassing (a result of solar radiation and wind) also causes nongravitational perturbations to their trajectories.

The coord\_func is always specified as \texttt{comet\_orbit}.  Note the
specification of a time (JDE) at pericenter which is typical for
comets.

Comet brightness is evaluated from
\begin{equation}
  \label{eq:comet_magnitudes}
  \mathrm{mag}=\mathtt{absolute\_magnitude}+5\cdot\log{\mathrm{distance}} + 2.5\cdot\mathtt{slope\_parameter}\cdot\log(\mathrm{CometSunDistance})
\end{equation}

The term \texttt{slope\_parameter} may be a misnomer in case of
comets. From the literature \citep{AstronomicalAlgorithms:1998} (equation 33.13) we find
\begin{equation}
  \label{eq:comet_magnitudes_Meeus}
  \mathrm{mag}=g+5\log\Delta + \kappa\log r
\end{equation}
from which $\kappa=2\cdot\mathtt{slope\_parameter}$. In any case, $\kappa$ is typically [5\ldots15] and specific for each comet.

\texttt{albedo} is used to set the brightness for rendering the body,
if you are close enough. 

A large number of elements for historical comets is provided in the
file \file{ssystem\_1000comets.ini} in the installation directory. You
can copy\&paste what you need into your
\file{ssystem\_minor.ini} or add all with the Solar System Editor plugin (section~\ref{sec:plugins:SolarSystemEditor}). 
Unfortunately it is not possible to specify
several sets of orbital elements for different epochs of one apparition which would
allow automatic changes.

\paragraph{Periodic Comet}
\label{sec:ssystem.ini:Comet:Periodic}

\begin{configfile}
[1phalley]
type=comet  
name=1P/Halley

coord_func=comet_orbit
parent=Sun [can be omitted]
orbit_ArgOfPericenter=111.7154
orbit_AscendingNode=58.8583
orbit_Eccentricity=0.968004
orbit_Inclination=162.2453
orbit_PericenterDistance=0.57136
orbit_TimeAtPericenter=2446463.12979167
orbit_good=780

color=1.0, 1.0, 1.0
dust_brightnessfactor=1.5
dust_lengthfactor=0.4
dust_widthfactor=1.5

albedo=0.1
radius=5
absolute_magnitude=5.5
slope_parameter=3.2
tex_map=nomap.png
model=1682q1halley_MLfix.obj [optional]
\end{configfile}

You may want to e.g.\ change the name in this entry
to \texttt{name=1P/Halley (1982i)}. Note a rather short
duration of \texttt{orbit\_good}, which means the comet is only displayed 780
days before and after perihelion.

Some comets have been visited by spacecraft so that shape models of their cores 
may be available and can be configured with a \texttt{model} entry.\newFeature{v0.16.0}

\paragraph{Parabolic/Hyperbolic Comet}
\label{sec:ssystem.ini:Comet:Parabolic}

\begin{configfile}
[c2013us10%28catalina%29]
type=comet
name=C/2013 US10 (Catalina)

coord_func=comet_orbit
parent=Sun 
orbit_ArgOfPericenter=340.3533
orbit_AscendingNode=186.141
orbit_Eccentricity=1.000372
orbit_Inclination=148.8766
orbit_PericenterDistance=0.822958
orbit_TimeAtPericenter=2457342.20748843
orbit_good=1000

color=1.0, 1.0, 1.0
dust_brightnessfactor=1.5
dust_lengthfactor=0.4
dust_widthfactor=1.5

albedo=0.1
radius=5
absolute_magnitude=4.4
slope_parameter=4
tex_map=nomap.png
\end{configfile}

This has basically the same format. Note that eccentricity is larger than 1,
this means the comet is following a slightly hyperbolic
orbit. Stellarium shows data for this comet for almost 3 years
(\texttt{orbit\_good=1000} days) from perihelion.




%%% Local Variables: 
%%% mode: latex
%%% TeX-PDF-mode: t
%%% TeX-master: "guide"
%%% End: 

