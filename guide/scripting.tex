
\chapter{Scripting}
\label{ch:scripting}

Many functions in Stellarium are scriptable. The programming language
ECMAscript\footnote{\url{https://en.wikipedia.org/wiki/ECMAScript}} (also known as JavaScript) can be used to control most
aspects of the software to construct automated shows.

\section{Introduction}
\label{sec:scripting:introduction}

Since version 0.10.1, Stellarium includes a scripting feature based on the Qt Scripting Engine\footnote{\url{http://doc.qt.io/qt-5/qtscript-index.html}}. This makes it possible to write small programs within Stellarium to produce presentations, set up custom configurations, and to automate repetitive tasks. 

The core scripting language is ECMAscript, giving users access to all basic ECMAScript language features such as flow control, variables string manipulation and so on. Interaction with Stellarium-specific features is done via a collection of objects which represent components of Stellarium itself. See appendix~\ref{ch:ScriptingAPI} for more details.

\section{Scripts}
\label{sec:scripting:scripts}

\subsection{Analemma}
\textbf{File name:} analemma.ssc \\
\textbf{Type:} script

A demonstration of the analemma -- the path of the Sun across the sky during the year.

\subsection{Common objects include file}
\textbf{File name:} common\_objects.inc \\
\textbf{Type:} library

Define arrays containing object name (in English). Defines: planets, constellations, messier, nebulae.

\subsection{Constellations Tour}
\textbf{File name:} constellations\_tour.ssc \\
\textbf{Shortcut:} Ctrl+D,1 \\
\textbf{Type:} script

A tour of the western constellations.

\subsection{Earth best views from other bodies}
\textbf{File name:} earth\_1.ssc \\
\textbf{Type:} script

Best views of Earth from other Solar System bodies in the 21st Century.

\subsection{Earth Events from Mercury}
\textbf{File name:} earth\_2.ssc \\
\textbf{Type:} script

Various Earth events as seen from location on Mercury.

\subsection{Earth Events from a floating city on Venus}
\textbf{File name:} earth\_3.ssc \\
\textbf{Type:} script

Earth Events from a floating city on Venus

\subsection{Earth Events from Mars}
\textbf{File name:} earth\_4.ssc \\
\textbf{Type:} script

\subsection{Earth and other planet's Greatest Elongations and Oppositions from Mars}
\textbf{File name:} earth\_5.ssc \\
\textbf{Type:} script

Earth and other planet's Greatest Elongations and Oppositions from Mars 2000-3000.

\subsection{Earth and Mars Greatest Elongations and Transits from Callisto}
\textbf{File name:} earth\_6.ssc \\
\textbf{Type:} script

Earth Greatest Elongations and Transits from Callisto 2000-3000. Why Callisto? Well of the 4 Galilean Moons, Callisto is the only one outside of Jupiter's radiation belt. Therefore, if humans ever colonize Jupiter's moons, Callisto will be the one.

\subsection{Earth and other Planets from Ceres}
\textbf{File name:} earth\_7.ssc \\
\textbf{Type:} script

Earth the other visible Planet's Greatest Elongations and Oppositions from Ceres 2000-3000.

\subsection{Landscape Tour}
\textbf{File name:} landscapes.ssc \\
\textbf{Type:} script

Look around each installed landscape.

\subsection{Partial Lunar Eclipse}
\textbf{File name:} lunar\_partial.ssc \\
\textbf{Type:} script

Script to demonstrate a partial lunar eclipse.

\subsection{Total Lunar Eclipse}
\textbf{File name:} lunar\_total.ssc \\
\textbf{Type:} script

Script to demonstrate a total lunar eclipse.

\subsection{Messier Objects Tour}
\textbf{File name:} messier\_tour.ssc \\
\textbf{Type:} script

A tour of Messier Objects.

\subsection{Double eclipse from Deimos in 2017}
\textbf{File name:} phobos\_phun\_1.ssc \\
\textbf{Type:} script

Just before Mars eclipses the sun, Phobos pops out from behind and eclipses it first. Takes place between Scorpio and Sagittarius on April 26, 2017.

\subsection{Double eclipse from Deimos in 2031}
\textbf{File name:} phobos\_phun\_2.ssc \\
\textbf{Type:} script

Just before Mars eclipses the sun, Phobos pops out from behind and eclipses it first. Takes place between Taurus and Gemini on July 23, 2031.

\subsection{Eclipse from Olympus Mons Jan 10 2068}
\textbf{File name:} phobos\_phun\_3.ssc \\
\textbf{Type:} script

Phobos eclipsing the Sun as seen from Olympus Mons on Jan 10, 2068.

\subsection{Occultation of Earth and Jupiter 2048}
\textbf{File name:} phobos\_phun\_4.ssc \\
\textbf{Type:} script

Phobos occultations of Earth are common, as are occultations of Jupiter. But occultations of both on the same day are very rare. Here's one that takes place 1/23/2048. In real speed.

\subsection{3 Transits and 2 Eclipses from Deimos 2027}
\textbf{File name:} phobos\_phun\_5.ssc \\
\textbf{Type:} script

Phobos races ahead of Mars and transits the sun, passes through it and then retrogrades back towards the sun and just partially transits it again (only seen in the southern hemisphere of Deimos), then Mars totally eclipses the sun while Phobos transits in darkness between Mars and Deimos. When Phobos emerges from Mars it is still eclipsed and dimmed in Mars' shadow, only to light up later.

\subsection{Save and restore display state}
\textbf{File name:} save\_state.inc \\
\textbf{Type:} library

Provide functions for saving and restoring state (view settings etc.) for use in other scripts via the include feature.  A state is associated with some string identifier.  This is arbitrary.  You can have as many states as you like subject to whatever memory restrictions there are...

\subsection{Screensaver}
\textbf{File name:} screensaver.ssc \\
\textbf{Shortcut:} Ctrl+D,3 \\
\textbf{Type:} script

A slow, infinite tour of the sky, looking at random objects.

\subsection{Sky Culture Tour}
\textbf{File name:} sky\_cultures.ssc \\
\textbf{Shortcut:} Ctrl+D,2 \\
\textbf{Type:} script

Look at each installed sky culture.

\subsection{Solar Eclipse 2009}
\textbf{File name:} solar\_eclipse.ssc \\
\textbf{Type:} script

Script to demonstrate a total solar eclipse which has happened in 2009 (location=Rangpur, Bangladesh).

\subsection{Solar System Screensaver}
\textbf{File name:} solar\_system\_screensaver.ssc \\
\textbf{Shortcut:} Ctrl+D,0 \\
\textbf{Type:} script

Screensaver of various happenings in the Solar System. 261 events in all!

\subsection{Startup Script}
\textbf{File name:} startup.ssc \\
\textbf{Type:} script

Script which runs automatically at startup.

\subsection{Sun from different planets}
\textbf{File name:} sun.ssc \\
\textbf{Type:} script

Look at the Sun from big planets of Solar System and Pluto.

\subsection{Tycho's Supernova}
\textbf{File name:} supernova.ssc \\
\textbf{Type:} script

Flash of the supernova observed by Tycho Brahe in 1572. The Historical supernovae plugin should be enabled!

\subsection{Transit of Venus}
\textbf{File name:} transit\_of\_venus.ssc \\
\textbf{Type:} script

Transit of Venus as seen from Sydney Australia, 6th June 2012.

\subsection{Provide simple translation option for scripts}
\textbf{File name:} translation.inc \\
\textbf{Type:} library

Simple translation functions for scripts. Set translations with the setTr function, then use tr(string) everywhere in your script where you want to get a translated string.  The current application language is taken from the Application Language setting. See core.setAppLanguage and core.getAppLanguage for details.

\subsection{Mercury Triple Sunrise and Sunset}
\textbf{File name:} triple\_sunrise\_and\_sunsets.ssc \\
\textbf{Type:} script

Due to the quirks in Mercury's orbit and rotation at certain spots the sun will rise and set 3 different times in one Mercury day.

\subsection{Zodiac}
\textbf{File name:} zodiac.ssc \\
\textbf{Type:} script

This script displays the constellations of the Zodiac. That means the constellations which lie along the line which the Sun traces across the celestial sphere over the course of a year.

\section{Includes}
\label{sec:scripting:includes}

Stellarium provides mechanism for splitting scripts on different files. Typical functions or list of variables can be stored in separate .inc file and used within script through \textbf{include()} command:
\begin{script}
include("common_objects.inc");
\end{script}

\section{Script Console}
\label{sec:scripting:console}
It is possible to open, edit run and save scripts using the script console window. To toggle the script console, press \key{F12}. The script console also provides an output window in which script debugging output is visible.

\section{How to write a script}
\label{sec:scripting:HowToWriteScript}

\subsection{Examples}
\label{sec:scripting:examples}
The best source of examples is the \textbf{scripts} sub-directory of the main Stellarium source tree (see section \ref{sec:scripting:scripts} for getting list of scripts). This directory contains a sub-directory called \textbf{tests} which are not installed with Stellarium, but are nontheless useful sources of example code for various scripting features\footnote{The directory can be browsed online at \url{http://bazaar.launchpad.net/~stellarium/stellarium/trunk/files/head:/scripts/}. Script files end in .ssc and .inc. Download links are to the right.}.

\subsection{Minimal Script}
\label{sec:scripting:MinimalScript}
This script prints "Hello Universe" in the Script Console output window.
\begin{script}
core.debug("Hello Universe");
\end{script}

\subsection{Using a StelModule}
\label{sec:scripting:UsingStelModule}
This script uses the LabelMgr module to display "Hello Universe" in white on the screen for 3 seconds.
\begin{script}
LabelMgr.labelScreen("Hello Universe", 200, 200, true, 20, "#ff0000");
core.wait(3);
LabelMgr.deleteAllLabels();
\end{script}

%% TODO: copy most of the Scripting API docs here.