\chapter{Scripting API}
\label{ch:ScriptingAPI}

Interation with Stellarium-specific functionality is done by calling the \textbf{public slots} of instances of a group of Stellarium's core classes.

\section{core}
\label{sec:ScriptingAPI:core}
Provide script API for Stellarium global functions.

\subsection{void core.clear(QString state)}
\label{sec:ScriptingAPI:core:clear}
Clear the display options, setting a ``standard'' view.

\paragraph{Parameters}
\begin{itemize}
\item \parameter{state} the name of a preset state.
\end{itemize}

\paragraph{Preset states}
\begin{itemize}
\item \keymap{natural}: azimuthal mount, atmosphere, landscape, no lines, labels or markers
\item \keymap{starchart}: equatorial mount, constellation lines, no landscape, atmosphere etc. labels and markers on.
\item \keymap{deepspace}: like starchart, but no planets, no eq.grid, no markers, no lines.
\item \keymap{galactic}: like deepspace, but in galactic coordinate system.
\end{itemize}

\paragraph{Default value}
Default value of parameter is \keymap{natural}.

\subsection{void core.debug(QString s)}
\label{sec:ScriptingAPI:core:debug}
Print a debugging message to the console.

\paragraph{Parameters}
\begin{itemize}
\item \parameter{s} the message to be displayed on the console.
\end{itemize}

\subsection{void core.dropSound(QString id)}
\label{sec:ScriptingAPI:core:dropSound}
Drop a sound from memory.

You should do this before the end of your script.
\paragraph{Parameters}
\begin{itemize}
\item \parameter{id} the message to be displayed on the consolethe identifier used when loadSound was called.
\end{itemize}

\subsection{void core.dropVideo(QString id)}
\label{sec:ScriptingAPI:core:dropVideo}
Drop a video from memory.

You should do this before the end of your script.

\paragraph{Parameters}
\begin{itemize}
\item \parameter{id} the identifier used when loadVideo() was called.
\end{itemize}

\subsection{void core.exit()}
\label{sec:ScriptingAPI:core:exit}
Stop the script.

\subsection{QStringList core.getAllSkyCultureIDs()}
\label{sec:ScriptingAPI:core:getAllSkyCultureIDs}
Get a list of Sky Culture IDs.

\paragraph{Returns}
A list of valid sky culture IDs.

\subsection{QString core.getAppLanguage()}
\label{sec:ScriptingAPI:core:getAppLanguage}
Get the current application language.

\paragraph{Returns}
Two letter language code, e.g. ``en'', or ``de'' and so on.

\subsection{QString core.getDate(QString spec)}
\label{sec:ScriptingAPI:core:getDate}
Get the simulation date and time as a string in ISO format, e.g. ``2008-03-24T13:21:01''

\paragraph{Parameters}
\begin{itemize}
\item \parameter{spec} if \keymap{utc}, the returned string's timezone is UTC, else it is local time.
\end{itemize}

\paragraph{Default value}
Default value of parameter is \keymap{utc}.

\subsection{QString core.getDeltaT()}
\label{sec:ScriptingAPI:core:getDeltaT}
Get the $\Delta T$ for the simulation date and time as a string in HMS format, e.g. ``0h1m68.2s''

\paragraph{Returns}
The $\Delta T$ for current simulation time.

\subsection{QString core.getDeltaTAlgorithm()}
\label{sec:ScriptingAPI:core:getDeltaTAlgorithm}
Get the $\Delta T$ equation name for the simulation date and time as a string.

\paragraph{Returns}
Name of the $\Delta T$ equation.

\subsection{bool core.getDiskViewport()}
\label{sec:ScriptingAPI:core:getDiskViewport}
Get the status of the disk viewport.

\paragraph{Returns}
A \keymap{true} if the disk view port is currently enabled.

\subsection{bool core.getFlagGravityLabels()}
\label{sec:ScriptingAPI:core:getFlagGravityLabels}
Get the current status of the gravity labels option.

\paragraph{Returns}
A \keymap{true} if gravity labels are enabled, else \keymap{false}.

\subsection{double core.getJDay()}
\label{sec:ScriptingAPI:core:getJDay}
Get the current date as Julian Day number.

\paragraph{Returns}
The Julian Day number.

\subsection{float core.getMaxFps()}
\label{sec:ScriptingAPI:core:getMaxFps}
Get the current maximum frames per second.

\paragraph{Returns}
The current maximum frames per secon setting.

\subsection{double core.getMilkyWayIntensity()}
\label{sec:ScriptingAPI:core:getMilkyWayIntensity}
Get Milky Way intensity.

\paragraph{Returns}
Value of Milky Way intensity, e.g. ``1.2''.

\subsection{float core.getMinFps()}
\label{sec:ScriptingAPI:core:getMinFps}
Get the current minimum frames per second.

\paragraph{Returns}
The current minimum frames per secon setting.

\subsection{double core.getMJDay()}
\label{sec:ScriptingAPI:core:getMJDay}
Get the current date as Modified Julian Day.

\paragraph{Returns}
The Modified Julian Day.

\subsection{QString core.getMountMode()}
\label{sec:ScriptingAPI:core:getMountMode}
Get the mount mode as a string.

\paragraph{Returns}
An ``equatorial'' or ``azimuthal''.

\subsection{bool core.getNightMode()}
\label{sec:ScriptingAPI:core:getNightMode}
Get the current status of Night Mode.

\paragraph{Returns}
A \texttt{true} if night mode is currently set.

\subsection{QVariantMap core.getObjectInfo(QString name)}
\label{sec:ScriptingAPI:core:getObjectInfo}
Fetch a map with data about an object's position, magnitude and so on.

\paragraph{Parameters}
\begin{itemize}
\item \parameter{name} is the English name of the object for which data will be returned.
\end{itemize}

\paragraph{Returns}
A map of object data. Keys:
\begin{itemize}
\item \keymap{altitude}: apparent altitude angle in decimal degrees;
\item \keymap{azimuth}: apparent azimuth angle in decimal degrees;
\item \keymap{altitude-geometric}: geometric altitude angle in decimal degrees;
\item \keymap{azimuth-geometric}: geometric azimuth angle in decimal degrees;
\item \keymap{ra}: right ascension angle (current date frame) in decimal degrees;
\item \keymap{dec}: declination angle (current date frame) in decimal degrees;
\item \keymap{raJ2000}: right ascension angle (J2000 frame) in decimal degrees;
\item \keymap{decJ2000}: declination angle (J2000 frame) in decimal degrees;
\item \keymap{glong}: galactic longitude in decimal degrees;
\item \keymap{glat}: galactic latitude in decimal degrees;
\item \keymap{elong}: ecliptic longitude in decimal degrees\footnote[1]{On Earth only!};
\item \keymap{elat}: ecliptic latitude in decimal degrees\footnotemark[1];
\item \keymap{elongJ2000}: ecliptic longitude (J2000 frame) in decimal degrees\footnotemark[1];
\item \keymap{elatJ2000}: ecliptic latitude (J2000 frame) in decimal degrees\footnotemark[1];
\item \keymap{vmag}: visual magnitude;
\item \keymap{vmage}: visual magnitude (extincted);
\item \keymap{size}: angular size in radians;
\item \keymap{size-dd}: angular size in decimal degrees;
\item \keymap{size-deg}: angular size in decimal degrees (formatted string);
\item \keymap{size-dms}: angular size in DMS format;
\item \keymap{localized-name}: localized name;
\item \keymap{distance}: distance to object in AU\footnote[2]{For Solar system objects only!};
\item \keymap{phase}: phase of object\footnotemark[2];
\item \keymap{illumination}: phase of object in percentages\footnotemark[2];
\item \keymap{phase-angle}: phase angle of object in radians\footnotemark[2];
\item \keymap{phase-angle-dms}: phase angle of object in DMS\footnotemark[2];
\item \keymap{phase-angle-deg}: phase angle of object in decimal degrees\footnotemark[2];
\item \keymap{elongation}: elongation of object in radians\footnotemark[2];
\item \keymap{elongation-dms}: elongation of object in DMS\footnotemark[2];
\item \keymap{elongation-deg}: elongation of object in decimal degrees\footnotemark[2].
\end{itemize}

\subsection{QString core.getObserverLocation()}
\label{sec:ScriptingAPI:core:getObserverLocation}
Get the ID of the current observer location.

\subsection{QVariantMap core.getObserverLocationInfo()}
\label{sec:ScriptingAPI:core:getObserverLocationInfo}
Get the info of the current observer location.

\paragraph{Returns}
A map of object data. Keys:
\begin{itemize}
\item \keymap{altitude}: altitude in meters;
\item \keymap{longitude}: longitude in decimal degrees;
\item \keymap{latitude}: latitude in decimal degrees;
\item \keymap{planet}: name of planet;
\item \keymap{location}: city and country;
\item \keymap{sidereal-year}: duration of the sidereal year on the planet in Earth's days;
\item \keymap{sidereal-day}: duration of the sidereal day on the planet in Earth's hours;
\item \keymap{solar-day}: duration of the mean solar day on the planet in Earth's hours;
\item \keymap{local-sidereal-time}: local sidereal time on the planet in hours;
\item \keymap{local-sidereal-time-hms}: local sidereal time on the planet in hours in HMS format.
\end{itemize}

\subsection{QString core.getProjectionMode()}
\label{sec:ScriptingAPI:core:getProjectionMode}
Get the current projection mode ID string.

\paragraph{Returns}
The string which identifies the current projection mode. For a list of possibl results, see section \ref{sec:ScriptingAPI:core:setProjectionMode()}.

\subsection{int core.getScreenHeight()}
\label{sec:ScriptingAPI:core:getScreenHeight}
Get the screen height (actually, height of Stellarium main view) in pixels.

\paragraph{Returns}
The screen height in pixels.

\subsection{int core.getScreenWidth()}
\label{sec:ScriptingAPI:core:getScreenWidth}
The screen width (actually, width of Stellarium main view) in pixels

\paragraph{Returns}
Get the screen width in pixels.

\subsection{QVariantMap core.getScreenXYFromAltAzi(QString altitude, QString azimuth)}
\label{sec:ScriptingAPI:core:getScreenXYFromAltAzi}
Get screen coordinates from some specified altitude and azimuth angles may be specified in a format recognised by \href{http://www.stellarium.org/doc/head/namespaceStelUtils.html#acd88f8194549c7d46656952428720427}{StelUtils::getDecAngle()}.

\paragraph{Parameters}
\begin{itemize}
\item \parameter{altitude} the altitude angle;
\item \parameter{azimuth} the azimuth angle;
\end{itemize}

\paragraph{Returns}
A map of object data. Keys:
\begin{itemize}
\item \keymap{x} x coordinate on the screen;
\item \keymap{y} y coordinate on the screen;
\end{itemize}

\subsection{double core.getScriptRate()}
\label{sec:ScriptingAPI:core:getScriptRate}
Get the script execution rate as a multiple of normal execution speed.

\paragraph{Returns}
The current script execution rate.

\subsection{QVariantMap core.getSelectedObjectInfo()}
\label{sec:ScriptingAPI:core:getSelectedObjectInfo}
Fetch a map with data about an latest selected object's position, magnitude and so on.
\paragraph{Returns}
A map of object data. Keys:
\begin{itemize}
\item \keymap{altitude}: apparent altitude angle in decimal degrees;
\item \keymap{azimuth}: apparent azimuth angle in decimal degrees;
\item \keymap{altitude-geometric}: geometric altitude angle in decimal degrees;
\item \keymap{azimuth-geometric}: geometric azimuth angle in decimal degrees;
\item \keymap{ra}: right ascension angle (current date frame) in decimal degrees;
\item \keymap{dec}: declination angle (current date frame) in decimal degrees;
\item \keymap{raJ2000}: right ascension angle (J2000 frame) in decimal degrees;
\item \keymap{decJ2000}: declination angle (J2000 frame) in decimal degrees;
\item \keymap{glong}: galactic longitude in decimal degrees;
\item \keymap{glat}: galactic latitude in decimal degrees;
\item \keymap{elong}: ecliptic longitude in decimal degrees\footnote[1]{On Earth only!};
\item \keymap{elat}: ecliptic latitude in decimal degrees\footnotemark[1];
\item \keymap{elongJ2000}: ecliptic longitude (J2000 frame) in decimal degrees\footnotemark[1];
\item \keymap{elatJ2000}: ecliptic latitude (J2000 frame) in decimal degrees\footnotemark[1];
\item \keymap{vmag}: visual magnitude;
\item \keymap{vmage}: visual magnitude (extincted);
\item \keymap{size}: angular size in radians;
\item \keymap{size-dd}: angular size in decimal degrees;
\item \keymap{size-deg}: angular size in decimal degrees (formatted string);
\item \keymap{size-dms}: angular size in DMS format;
\item \keymap{localized-name}: localized name;
\item \keymap{distance}: distance to object in AU\footnote[2]{For Solar system objects only!};
\item \keymap{phase}: phase of object\footnotemark[2];
\item \keymap{illumination}: phase of object in percentages\footnotemark[2];
\item \keymap{phase-angle}: phase angle of object in radians\footnotemark[2];
\item \keymap{phase-angle-dms}: phase angle of object in DMS\footnotemark[2];
\item \keymap{phase-angle-deg}: phase angle of object in decimal degrees\footnotemark[2];
\item \keymap{elongation}: elongation of object in radians\footnotemark[2];
\item \texttt{elongation-dms}: elongation of object in DMS\footnotemark[2];
\item \keymap{elongation-deg}: elongation of object in decimal degrees\footnotemark[2].
\end{itemize}

\subsection{QString core.getSkyCulture()}
\label{sec:ScriptingAPI:core:getSkyCulture}
Find out the current sky culture.

\paragraph{Returns}
The ID of the current sky culture (i.e. the name of the directory in which the curret sky cultures files are found, e.g. ``western'').

\subsection{QString core.getSkyCultureName()}
\label{sec:ScriptingAPI:core:getSkyCultureName}
Find out the current sky culture and get it English name.

\paragraph{Returns}
The English name of the current sky culture.

\subsection{QString core.getSkyCultureNameI18n()}
\label{sec:ScriptingAPI:core:getSkyCultureNameI18n}
Find out the current sky culture and get it localized name.

\paragraph{Returns}
The translated name of the current sky culture.

\subsection{QString core.getSkyLanguage()}
\label{sec:ScriptingAPI:core:getSkyLanguage}
Get the current sky language.

\paragraph{Returns}
Two letter language code, e.g. ``en'', or ``de'' and so on.

\section{ConstellationMgr}
\label{sec:ScriptingAPI:ConstellationMgr}
Display and manage the constellations. It can display constellations lines, names, art textures and boundaries. It also supports several different sky cultures.

\section{GridLinesMgr}
\label{sec:ScriptingAPI:GridLinesMgr}
The GridLinesMgr controls the drawing of the Azimuthal, Equatorial, Ecliptical and Galactic Grids, as well as the great circles: Meridian Line, Ecliptic Lines of J2000.0 and date, Equator Line (of J2000.0 and date), Precession Circles, and a special line marking conjunction or opposition in ecliptical longitude (of date).

\section{LabelMgr}
\label{sec:ScriptingAPI:LabelMgr}
Allows for creation of custom labels on objects or coordinates. Because this class is intended for use in scripting (although other uses are also fine), all label types and so on are specified by QString descriptions. The labels are painted very late, i.e. also sky object labels will be written over the landscape.

\section{LandscapeMgr}
\label{sec:ScriptingAPI:LandscapeMgr}
Manages all the rendering at the level of the observer's surroundings. This includes landscape textures, fog, atmosphere and cardinal points. I decided to put all these elements together in a single class because they are inherently linked, especially when we start moving the observer in altitude.

\section{SporadicMeteorMgr}
\label{sec:ScriptingAPI:SporadicMeteorMgr}
Simulates a sporadic meteor shower, with a random color and a random radiant.

\section{NebulaMgr}
\label{sec:ScriptingAPI:NebulaMgr}
Manage a collection of nebulae. This class is used to display the deep-sky catalog with information, and textures for some of them.

\section{ScreenImageMgr}
\label{sec:ScriptingAPI:ScreenImageMgr}
Module for managing images for scripting. Images are identified by a string ID which is passed to ScreenImageMgr members when it is necessary to specify an image to work with. Member functions in this class which modify the state of the class are all mediated through the signal/slot mechanism to ensure such operations happen in the main thread where images are drawn, and not in the script thread.

\section{SolarSystem}
\label{sec:ScriptingAPI:SolarSystem}
This StelObjectModule derivative is used to model SolarSystem bodies. This includes the Major Planets, Minor Planets and Comets.

\section{StarMgr}
\label{sec:ScriptingAPI:StarMgr}

Stores the star catalogue data.

Used to render the stars themselves, as well as determine the color table and render the labels of those stars with names for a given SkyCulture.

The celestial sphere is split into zones, which correspond to the triangular faces of a geodesic sphere. The number of zones (faces) depends on the level of sub-division of this sphere. The lowest level, 0, is an icosahedron (20 faces), subsequent levels, L, of sub-division give the number of zones, n as:

\begin{equation}
n=20 x 4^L
\end{equation}

Stellarium uses levels 0 to 7 in the existing star catalogues. Star Data Records contain the position of a star as an offset from the central position of the zone in which that star is located, thus it is necessary to determine the vector from the observer to the centre of a zone, and add the star's offsets to find the absolute position of the star on the celestial sphere.

This position for a star is expressed as a 3-dimensional vector which points from the observer (at the centre of the geodesic sphere) to the position of the star as observed on the celestial sphere.

\section{StelAudioMgr}
\label{sec:ScriptingAPI:StelAudioMgr}

\section{StelVideoMgr}
\label{sec:ScriptingAPI:StelVideoMgr}

A scriptable way to show videos embedded in the screen.

After experimental support with Qt4/Phonon library, this feature is back. Videos can be scaled, paused, placed and relocated (shifted) on screen. Setting opacity seems not to do much unless setting it to zero, the video is then simply invisible. Therefore smooth fading in/out or setting a semitransparent overlay does not work, but there is now an intro/end animation available: zooming out from a pixel position to a player frame position, and returning to that spot close to end of video playback.

However, support for multimedia content depends on the operating system, installed codecs, and completeness of the QtMultimedia system support, so some features or video formats may not work for you (test video and re-code it if necessary).

\section{StelMovementMgr}
\label{sec:ScriptingAPI:StelMovementMgr}

Manages the head movements and zoom operations.

\section{StelSkyDrawer}
\label{sec:ScriptingAPI:StelSkyDrawer}

Provide a set of methods used to draw sky objects taking into account eyes adaptation, zoom level, instrument model and artificially set magnitude limits.

\section{StelSkyLayerMgr}
\label{sec:ScriptingAPI:StelSkyLayerMgr}

Manage the sky background images, including DSS and deep sky objects images.
