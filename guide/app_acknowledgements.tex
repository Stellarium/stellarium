%% \chapterimage{chapter-t1-bg} % Chapter heading image
\chapter{Acknowledgements}

\section{Contributors}
\label{sec:Contributors}
When not listed as chapter authors, the following were the main contributors to the User Guide, mostly for versions before V0.15:

\noindent%
\begin{tabularx}{\textwidth}{l|X}
\toprule
Matthew Gates  & Primary author of the first User Guide (versions 0.9 to 0.12)\\
Paul Robinson  &Sky guide; exercise ideas\\
Andras Mohari  &Celestial sphere diagrams; numerous corrections\\
Rudy Gobits, Dirk Schwarzhans & Mac platform specifics\\
Barry Gerdes (\textdagger{}2014)  &Windows platform specifics \\
%              &Large parts of Ch.~\ref{ch:Phenomena} (Astronomical Phenomena)\\ %% Mentioned as Chapter Author
               &Customisation of .fab files\\
%              &Ch.~\ref{ch:Landscapes} Making a custom landscape (first version) \\ %% Mentioned as Chapter Author, and the old chapter has been superseded.
               &Documentation Maintainer up to V0.12\\
Sigma          &Japanese translation; many corrections\\ %% FOR FIRST EDITION. 
Richard Powell & Figure~\ref{fig:colourmag} (colour/magnitude diagram) is a modification of his diagram; 
                 he kindly granted permission for it to be distributed under the FDL\\
John Twin      & Many spelling corrections \\\midrule
Georg Zotti    & Lead author of the 0.15+ editions\\
Alexander Wolf & New layout, many contributions for the 0.15+ editions\\
The rest of the Stellarium developer team & You know who you are\ldots :-)\tabularnewline
\bottomrule
\end{tabularx}


Additional material has been incorporated into the guide from sources
that are published under the GNU FDL, including material from Wikipedia
and the Astronomy book at Wikibooks.

\section{How you can help}
\label{sec:HowYouCanHelp}

We especially welcome contributions, but also bug reports, feature requests and other feedback through the
usual channels (trackers, forums and so on):
\begin{description}
\item[Stellarium on Github] \url{https://github.com/Stellarium/}
%\item[Bug tracker] \url{https://bugs.launchpad.net/stellarium}
%\item[Questions for Stellarium (FAQ)\index{FAQ}] \url{https://answers.launchpad.net/stellarium}
\item[Questions for Stellarium (FAQ)\index{FAQ}] \url{https://github.com/Stellarium/stellarium/wiki/FAQ}
\item[Current issues (FAQ)\index{FAQ}] \url{https://github.com/Stellarium/stellarium/wiki/Common-Problems-for-the-current-version}
\item[Google Group] \url{https://groups.google.com/group/stellarium}
\item[Feedback forum] \url{https://sourceforge.net/p/stellarium/discussion/278769/}
%\item[Blueprints for Stellarium] \url{https://blueprints.launchpad.net/stellarium}
\end{description}


\section{Technical Articles}
\label{sec:ack:technical}
% 2018-02-07: Put used data from git here.

Stellarium would not look the same and work without these works from computer graphics and astronomy.

\begin{description}
	\item[The tone reproductor class]
		The class mainly performs a fast implementation of the algorithm
		from \citet{TumblinRushmeier:1993}, with more accurate values from \citet{DevlinChalmersWilkie:2002}. 
		The blue shift formula is taken from \citet{WannJensen:2000} and combined with the Scotopic vision formula from  \citet{Larson:1997}.
	\item[The skylight class]
		The class governs sky colors and is a fast implementation of the algorithm by \citet{Preetham:1999}.
	\item[The skybright class]
		The class governs physical sky brightness values based on Bradley Schaefer's  \program{VISLIMIT.BAS} basic source code \citep{Schaefer:1998}.
%		from Brad Schaefer's article on pages 57-60,  May 1998 _Sky & Telescope_,	"To the Visual Limits". 
		The basic sources are available on the Sky and Telescope web site (code ``offered as-is and without support'').
	\item[The $\Delta T$ calculations]
		For implementation of calculation routines for $\Delta T$ we used the following sources:
		\begin{enumerate}
		\item $\Delta T$ webpage by Rob van Gent: \url{http://www.staff.science.uu.nl/~gent0113/deltat/deltat.htm}
		\item \citetp{Espenak-Meeus:2006} (\url{http://eclipse.gsfc.nasa.gov/SEhelp/deltatpoly2004.html})
		\item \citetp{1948AJ.....53..169C}
		\item \citetp{1939MNRAS..99..541S} %``The Rotation of the Earth, and the Secular Accelerations of the Sun, Moon and Planets'' 
		\item \citetp{1979AcA....29..101S} %``Polynomial approximations for the correction delta T E.T.-U.T. in the period 1800-1975'' 
		\item \citetp{1988A&A...205L...8B} %"ELP 2000-85 and the dynamic time-universal time relation" 
		\item \citetp{1988AN....309..219S} %``Empirical Transformations from U.T. to E.T. for the Period 1800-1988''
		\item \citetp{2004JHA....35..327M} %``Historical values of the Earth's clock error DeltaT and the calculation of eclipses'' 
		     with Addendum \citep{2005JHA....36..339M}
		\item \citetp{2000JBAA..110..323M}% ``Polynomial approximations to Delta T, 1620-2000 AD''
		%% DISABLED NEXT: This only uses Schoch!
		%\item \citetp{2009ASPC..409..166H} %``Einstein's Theory of Relativity Confirmed by Ancient Solar Eclipses'', Henriksson G.,
		\item \citetp{Mucke-Meeus:1983}% ``Canon of Solar Eclipses''
		\item \citetp{1975grhe.conf..459M} %``The accelerations of the earth and moon from early astronomical observations''
		\item \citetp{1978tfer.conf....5S} %``Pre-Telescopic Astronomical Observations''%, Stephenson F.R.,
		\item \citetp{1984RSPTA.313...47S} %``Long-term changes in the rotation of the earth - 700 B.C. to A.D. 1980'' 
		\item \citetp{1995RSPTA.351..165S} %``Long-Term Fluctuations in the Earth's Rotation: 700 BC to AD 1990''
		\item \citetp{Stephenson:1997}  %``Historical Eclipses and Earth's Rotation''%by F. R. Stephenson (1997)
		\item \citetp{AstronomicalAlgorithms:1998} %``Astronomical Algorithms''
		\item \citetp{Montenbruck-Pfleger:2000} % ``Astronomy on the Personal Computer'' 
		\item \citetp{Reingold-Dershowitz:2001} %``Calendrical Calculations''
		\item $\Delta T$ webpage by V. Reijs: \url{http://www.iol.ie/~geniet/eng/DeltaTeval.htm}
		\end{enumerate}
	\item[Precession:] \citetp{2011AA:Vondrak} with correction \citep{2012AA:Vondrak} 
	\item[Nutation:]   \citetp{Nutation:IAU2000B} %Dennis D. McCarthy and Brian J. Luzum: An Abridged Model of the Precession-Nutation of the Celestial Pole.
		This model provides accuracy better than 1 milli-arcsecond in the
		time 1995-2050. It is applied for years -4000..+8000 only.
        \item[Martian polar caps:] Based on data from \citet{MarsPoles:2009} and \citet{MarsRotation:2015}.\newFeature{0.22.0}
\end{description}		
		
\section{Included Source Code}
\label{sec:ack:code}

\begin{itemize}		
  \item Some computation of the sidereal time (\file{sidereal\_time.h/c}) and Pluto
      orbit contains code from the libnova library (LGPL) by Liam Girdwood.
  \item The \file{orbit.cpp/h} and \file{solve.h} files were directly borrowed from
      \program{Celestia} (Chris Laurel; GPL license). They have now evolved a bit.
  \item Several implementations of IMCCE theories for planet and satellite movement by Johannes Gajdosik 
       (MIT-style license, see the corresponding files for the license text)
  \item The tesselation algorithms were originally extracted from the glues 
      library version 1.4 by Mike Gorchak \url{<mike@malva.ua>} (SGI FREE SOFTWARE LICENSE B).
  \item OBJ loader in the Scenery3D plugin based on glObjViewer (c) 2007 dhpoware
  \item Parts of the code to work with DE430 and DE431 data files have been taken from Project Pluto (GPL license).
  \item The \file{SpoutLibrary.dll} and header from the SpoutSDK version 2.005 available at \url{http://spout.zeal.co} (BSD license).
\end{itemize}

\section{Data}
\label{sec:ack:data}

\begin{enumerate}
\item The Hipparcos star catalog
    From ESA (European Space Agency) and the Hipparcos mission. ref. ESA, 1997, The Hipparcos and Tycho Catalogues, ESA SP-1200 \url{http://cdsweb.u-strasbg.fr/ftp/cats/I/239}
	% TODO: NEXT ENTRY should be changed to ExplanSuppl./WGRE
\item The solar system data mainly comes from IMCCE and partly from Celestia.
\item Polynesian constellations are based on diagrams from the Polynesian Voyaging Society
% TODO: MAKE SURE THIS IS STILL VCALID WITH THE NEW VERSION 2018.
\item Chinese constellations are based on diagrams from the Hong Kong Space Museum
\item Egyptian constellations are based on the work of Juan Antonio Belmonte, Instituto de Astrofisica de Canarias
\item The Tycho-2 Catalogue of the 2.5 Million Brightest Stars Hog E., Fabricius C., Makarov V.V., Urban S., Corbin T., Wycoff G., Bastian U., Schwekendiek P., Wicenec A.
    Astron. Astrophys. 355, L27 (2000)
    \url{http://cdsweb.u-strasbg.fr/ftp/cats/I/259}
\item Naval Observatory Merged Astrometric Dataset (NOMAD) version 1 (\url{http://www.nofs.navy.mil/nomad})
    Norbert Zacharias writes:
	\begin{quotation}
    ``There are no fees, both UCAC and NOMAD are freely available with the only requirement that the source of the data (U.S.
    Naval Observatory) and original product name need to be provided with any distribution, as well as a description about any
    changes made to the data, if at all.''
	\end{quotation}
    The changes made to the data are:
	\begin{itemize}
    \item try to compute visual magnitude and color from the b,v,r values
    \item compute $\mathrm{nr\_of\_measurements} = \mathrm{the\ number\ of\ valid\ b,v,r\ values}$
    \item throw away or keep stars (depending on magnitude, nr\_of\_measurements, combination of flags, tycho2 number)
    \item add all stars from Hipparcos (incl.\ component solutions), and tycho2+1st supplement
    \item reorganize the stars in several brigthness levels and triangular zones according to position and magnitude
	\end{itemize}
    The programs that are used to generate the star files are called
    \program{MakeCombinedCatalogue}, \program{ParseHip}, \program{ParseNomad}, and can be
    found in the util subdirectory in source code. The position,
    magnitudes, and proper motions of the stars coming from NOMAD
    are unchanged, except for a possible loss of precision,
    especially in magnitude. When there is no V magnitude, it is
    estimated from R or B magnitude.  When there is no B or V
    magnitude, the color B-V is estimated from the other magnitudes.
    Also proper motions of faint stars are neglected at all.
\item Stellarium's Catalog of Variable Stars based on General Catalog of Variable Stars (GCVS) version 2013Apr. \url{http://www.sai.msu.su/gcvs/gcvs/}
    Samus N.N., Durlevich O.V., Kazarovets E V., Kireeva N.N., Pastukhova E.N., Zharova A.V., et al., General Catalogue of Variable Stars (Samus+ 2007-2012)
    \url{http://cdsarc.u-strasbg.fr/viz-bin/Cat?cat=B%2Fgcvs&}
\item Consolidated DSO catalog was created from various data:
	\begin{enumerate}
     \item NGC/IC data taken from SIMBAD Astronomical Database \url{http://simbad.u-strasbg.fr}
     \item Distance to NGC/IC data taken from NED                (NASA/IPAC EXTRAGALACTIC DATABASE)  \url{http://ned.ipac.caltech.edu}
     \item Catalogue of HII Regions                              (Sharpless, 1959)     (\url{http://vizier.u-strasbg.fr/viz-bin/VizieR?-source=VII/20})
     \item H-$\alpha$ emission regions in the Southern Milky Way (Rodgers+, 1960)      (\url{http://vizier.u-strasbg.fr/viz-bin/VizieR?-source=VII/216})
     \item Catalogue of Reflection Nebulae                       (Van den Bergh, 1966) (\url{http://vizier.u-strasbg.fr/viz-bin/VizieR?-source=VII/21})
     \item Lynds' Catalogue of Dark Nebulae (LDN)                (Lynds, 1962)         (\url{http://vizier.u-strasbg.fr/viz-bin/VizieR?-source=VII/7A})
     \item Lynds' Catalogue of Bright Nebulae                    (Lynds, 1965)         (\url{http://vizier.u-strasbg.fr/viz-bin/VizieR?-source=VII/9})
     \item Catalog of bright diffuse Galactic nebulae            (Cederblad, 1946)     (\url{http://vizier.u-strasbg.fr/viz-bin/VizieR?-source=VII/231})
     \item Barnard's Catalogue of 349 Dark Objects in the Sky    (Barnard, 1927)       (\url{http://vizier.u-strasbg.fr/viz-bin/VizieR?-source=VII/220A})
     \item A Catalogue of Star Clusters shown on Franklin-Adams Chart Plates (Melotte, 1915) from NASA ADS (\url{http://adsabs.harvard.edu/abs/1915MmRAS..60..175M})
     \item On Structural Properties of Open Galactic Clusters and their Spatial Distribution. Catalog of Open Galactic Clusters (Collinder, 1931)
	       from NASA ADS (\url{http://adsabs.harvard.edu/abs/1931AnLun...2....1C})
     \item The Collinder Catalog of Open Star Clusters. An Observer’s Checklist. Edited by Thomas Watson 
	       from CloudyNights (\url{http://www.cloudynights.com/page/articles/cat/articles/the-collinder-catalog-updated-r2467})
	\end{enumerate}
\item Cross-identification of objects in consolidated DSO catalog was maked with:
	\begin{enumerate}
    \item  SIMBAD Astronomical Database \url{http://simbad.u-strasbg.fr}
    \item  Merged catalogue of reflection nebulae (Magakian, 2003) \url{http://vizier.u-strasbg.fr/viz-bin/VizieR-3?-source=J/A+A/399/141}
    \item  Messier  Catalogue from Wikipedia (includes morphological classification and distances) \url{https://en.wikipedia.org/wiki/List_of_Messier_objects}
    \item  Caldwell Catalogue from Wikipedia (includes morphological classification and distances) \url{https://en.wikipedia.org/wiki/Caldwell_catalogue}
	\end{enumerate}
\item Morphological classification and magnitudes (partially) for Melotte catalogue from DeepSkyPedia \url{http://deepskypedia.com/wiki/List:Melotte}

\end{enumerate}


\section{Image Credits}
\label{sec:ack:images}

% TODO: FIX SPELLINGS ETC. Placed image credits from git here. 
All graphics are copyrighted by the Stellarium's Team (GPL) except the ones mentioned below:
\begin{itemize}		
\item The ``earthmap'' texture was created by NASA (Reto Stockli, NASA Earth
	  Observatory) using data from the MODIS instrument aboard the
	  Terra satellite (Public Domain). See section~\ref{sec:ack:credits:stockli} for details.
\item Moon texture map \file{moon.png} and normal map was combined
  from maps by USGS Astrogeology Research Program,
  (\url{http://astrogeology.usgs.gov}; Public Domain, DFSG-free) and
  by Lunar surface textures from Celestia, based on Clementine data
  (Public Domain).  The new 4k texture is from NASA's Scientific
  Visualization Studio (\url{https://svs.gsfc.nasa.gov/4720}).
\item Saturn map and ring textures created by Björn Jónsson:
	  ``All the planetary maps available on these pages are publicly
	  available. You do not need a special permission to use them but if
	  you do then please mention their origin in your work [..]''
\item Jupiter map created by James Hastings-Trew from Cassini data. ``The 
	  maps are free to download and use as source material or resource
	  in artwork or rendering (CGI or real time).''
\item The Venus, Amalthea, Proteus, Iapetus and Phoebe maps and rings of Uranus and Neptune are from Celestia (\url{http://shatters.net/celestia/})
	  under the GNU General Purpose License, version 2 or any later version:
	  \begin{itemize}
	     \item Venus is from Björn Jónsson and modified by RVS.
	     \item Amalthea is a shaded relief map by Phil Stooke, colored by Wm. Robert Johnston 
	       (\url{http://www.johnstonsarchive.net/spaceart/cylmaps.html}),
	       and further modified by Jens Meyer and Grant Hutchison.
	     \item The Proteus map is from Phil Stooke.
	     \item Triton is probably from David Seal's site (see below), modified by Chris Laurel and Grant Hutchison.
	     \item Iapetus and Phoebe maps are from Dr. Fridger Schrempp (t00fri).
	  \end{itemize}
\item Mercury map is produced by NASA from Messenger data
	  (\url{https://astrogeology.usgs.gov/search/map/Mercury/Messenger/Global/Mercury_MESSENGER_MDIS_Basemap_BDR_Mosaic_Global_166m})
	  and modified and colored by RVS. License: public domain.
\item Europa, Io and Callisto maps are created by John van Vliet from PDS data and modified by RVS. License: cc-by-sa.
\item Ganymede map is from USGS and modified by RVS. Public domain.
\item Tethys, Dione, Rhea, Enceladus and Mimas maps are created by NASA (CICLOPS team) from Cassini data, colored by RVS. Public domain.
\item Hyperion map created by John van Vliet from PDS data, modified by RVS. License: cc-by-sa.
\item Triton: Image selection, radiometric calibration, geographic registration and photometric correction, and final mosaic assembly 
	  were performed by Dr. Paul Schenk at the Lunar and Planetary Institute, Houston, Texas. Image data from Voyager 2 (NASA, JPL).
	  Original texture has ``white spots'', which was filled by Dizel777 (\url{http://spaceengine.org/forum/19-563-25069-16-1409101585}).
\item Pluto map is produced by NASA from New Horizons data and colored by RVS. (Courtesy NASA/Johns Hopkins University Applied Physics
	  Laboratory/Southwest Research Institute, \url{http://photojournal.jpl.nasa.gov/catalog/PIA19858}).
\item Charon map is produced by NASA from New Horizons data and colored by RVS. (Courtesy NASA/Johns Hopkins University Applied Physics
	  Laboratory/Southwest Research Institute (\url{http://photojournal.jpl.nasa.gov/catalog/PIA19866}).
\item All other planet maps from David Seal's site: \url{http://maps.jpl.nasa.gov/}; see license in section~\ref{sec:ack:credits:jpl}.
\item The fullsky milky way panorama has been created by Axel Mellinger, University of Potsdam, Germany. 
      Further information and more pictures available from \url{http://home.arcor-online.de/axel.mellinger/}.
      License: permission given to ``Modify and redistribute this image if proper credit to the original image is given.''
\item All messiers nebula pictures except those mentioned below from the
	  Grasslands Observatory: ``Images courtesy of Tim Hunter and James McGaha, Grasslands Observatory at \url{http://www.3towers.com}.''
	  License: permission given to ``use the image freely'' (including right to modify and redistribute) ``as long as it is credited.''
\item M31, and the Pleiades pictures come from Herm Perez: \url{http://home.att.net/~hermperez/default.htm}
	  License: ``Feel free to use these images, if you use them in a commercial setting please attribute the source.''
\item Images of M8, M33, NGC253, NGC1499, NGC2244 from Jean-Pierre Bousquet
\item Images of M1, M15, M16, M27, M42, M57, M97, NGC6946 from Stephane Dumont
\item Images of M17, M44, NGC856, NGC884 from Maxime Spano
\item Constellation art, GUI buttons, logo created by Johan Meuris (Jomejome) (\url{jomejome (at) users.sourceforge.net})
	  \url{http://www.johanmeuris.eu/}.
	  License: released under the Free Art License (\url{http://artlibre.org/licence.php/lalgb.html}).
		 
	  Icon created by Johan Meuris, License: Creative Commons Attribution-ShareAlike 3.0 Unported.
\item The ``earth-clouds'' texture includes imagery owned by NASA.
	  See NASA's Visible Earth project at \url{http://visibleearth.nasa.gov/}.

	  License: \begin{enumerate}
	                  \item The imagery is free of licensing fees
		              \item NASA requires that they be provided a credit as the owners of the imagery.
                \end{enumerate}
	  The cloud texturing was taken from Celestia (GPL; \url{http://www.shatters.net/celestia/}).
\item The folder icon derived from the Tango Desktop Project, used under the terms of the Creative Commons Attribution Share-Alike license.
\item Images of NGC281, NGC5139, NGC6543, NGC6960, NGC7023, NGC7317, NGC7319, NGC7320
	  from Andrey Kuznetsov, Kepler Observatory (\url{http://kepler-observatorium.ru}). 
	  License: Creative Commons Attribution 3.0 Unported.
\item Images of NGC891, NGC1333, NGC2903, NGC3185, NGC3187, NGC3189, NGC3190, NGC3193, NGC3718, NGC3729, NGC4490, NGC5981, NGC5982, NGC5985, NGC7129 
	  from Oleg Bryzgalov (\url{http://olegbr.astroclub.kiev.ua/}). License: Creative Commons Attribution 3.0 Unported.
\item Image of solar corona from eclipse 2008-08-01 and image of $\eta$ Carinae by Georg Zotti \url{http://www.waa.at/}. License: Creative Commons Attribution 3.0 Unported.
\item Images of IC1805, IC1848, NGC6888 from Steve Tuttle (\url{http://www.stuttle1.com/}).
\item Images of IC4628, M20, M21, M47, NGC2467, IC2948, NGC3324, NGC3293, NGC7590, RCW158 from Trevor Gerdes (\url{http://www.sarcasmogerdes.com/}).
\item Images of IC2118, NGC1532 from users of Ice In Space (\url{http://www.iceinspace.com.au/}).
\item Image of IC5146 from James A Weier.
\item Images of SMC, LMC (Magellanic Clouds) and $\rho$ Oph from Albert Van Donkelaar.
\item Images of NGC55, NGC300, NGC1365, NGC3628, NGC4945, NGC5128, NGC6726, NGC6744, NGC6752, NGC6822, NGC7293, NGC2070
	  from Philip Montgomery (\url{http://www.kenthurst.bigpondhosting.com/}).
\item The Vesta and Ceres map was taken from USGS website \url{https://astrogeology.usgs.gov/} and colored by RVS. License: public domain.
\item Images of NGC7318, NGC7331, M3, M13, M51, M63, M64, M74, M78, M81, M82, M96, M101, M105, Barnard 22, Barnard 142, Barnard 173, IC405,
	  IC443, NGC1514, NGC1961, NGC2371, NGC2403, NGC246, NGC2841, NGC3310, NGC3938, NGC4559, NGC7008, NGC7380, NGC7479, NGC7635, Sh2-101
	  from Peter Vasey, Plover Hill Observatory (\url{http://www.madpc.co.uk/~peterv/}).
\item Image of IC434 from Marc Aragnou.
\item Images of IC59, IC63, IC410, NGC2359, Sadr region from Carole Pope (\url{https://sites.google.com/site/caroleastroimaging/}).
\item Images of NGC3690, NGC5257, NGC6050, IC883, UGC8335 and UGC9618 from NASA, ESA, the Hubble Heritage (STScI/AURA)-ESA/Hubble Collaboration.
	  License: public domain; (\url{http://hubblesite.org/copyright/})
\item Images of NGC40, NGC4631, NGC4627, NGC4656, NGC4657 from Steven Bellavia.
\item Images of Barnard Loop, IC342 from Sun Shuwei. License: public domain.
\item Images of M77, Sh2-264, Sh2-308 from Wang Lingyi. License: public domain.
\item Images of IC10 from Lowell Observatory (\url{http://www2.lowell.edu/}). License: public domain.
\item Images of IC2177, NGC6334, NGC6357, Fornax Cluster from ESO/Digitized Sky Survey 2 (\url{http://eso.org/public/}).
	  License: Creative Commons Attribution 4.0 International. 
\item Images of NGC3603 from ESO/La Silla Observatory (\url{http://eso.org/public/}). License: Creative Commons Attribution 4.0 International. 
\item Images of NGC4244 from Ole Nielsen. License: Creative Commons Attribution-Share Alike 2.5 Generic. 
\item Images of NGC7000 from NASA. License: public domain.
\item Images of IC1396 from Giuseppe Donatiello. License: Creative Commons CC0 1.0 Universal Public Domain Dedication.
\item Images of Sh2-155 from Hewholooks (\url{https://commons.wikimedia.org/wiki/}). License: Creative Commons Attribution-Share Alike 3.0 Unported.
\item Images of NGC4565 from Ken Crawford (\url{http://www.imagingdeepsky.com/}).
	  License: ``This work is free and may be used by anyone for any purpose.
	  If you wish to use this content, you do not need to request permission
	  as long as you follow any licensing requirements mentioned on this page.''
\item Images of Coma Cluster from Caelum Observatory (\url{http://www.caelumobservatory.com/}). License: Creative Commons Attribution-Share Alike 3.0 Unported.
\item Images of NGC1316 from ESO (\url{http://eso.org/public/}). License: Creative Commons Attribution 4.0 International.
\end{itemize}


\subsection{Full credits for ``earthmap'' texture}
\label{sec:ack:credits:stockli}
\begin{description}
\item[Author:] Reto Stockli, NASA Earth Observatory,
		\url{rstockli (at) climate.gsfc.nasa.gov}
\item[Address of correspondence:]
        \begin{minipage}[t]{\textwidth}
		Reto Stockli\\
		ETH/IAC (NFS Klima) \& NASA/GSFC Code 913 (SSAI)\\
		University Irchel\\
		Building 25 Room J53\\
		Winterthurerstrasse 190\\
		8057 Zurich, Switzerland
		\end{minipage}
\item[Phone:]  +41 (0)1 635 5209
\item[Fax:]    +41 (0)1 362 5197
\item[Email:]  \url{rstockli (at) climate.gsfc.nasa.gov}
\item[URL:]	\url{http://earthobservatory.nasa.gov},
	\url{http://www.iac.ethz.ch/staff/stockli}
\item[Supervisors:]
		Fritz Hasler and David Herring, NASA/Goddard Space Flight Center
\item[Funding:]
		This project was realized under the SSAI subcontract 2101-01-027 (NAS5-01070)
\item[License:]
		``Any and all materials published on the Earth Observatory are
		freely available for re-publication or re-use, except where
		copyright is indicated.''
\end{description}

\subsection{License for the JPL planets images}
\label{sec:ack:credits:jpl}

From \url{http://www.jpl.nasa.gov/images/policy/index.cfm}:
\begin{quotation}
    \noindent Unless otherwise noted, images and video on JPL public web sites (public
    sites ending with a \texttt{jpl.nasa.gov} address) may be used for any purpose
    without prior permission, subject to the special cases noted below.
    Publishers who wish to have authorization may print this page and retain
    it for their records; JPL does not issue image permissions on an image
    by image basis.  By electing to download the material from this web site
    the user agrees:
	\begin{enumerate}
    \item that Caltech makes no representations or warranties with respect to
       ownership of copyrights in the images, and does not represent others
       who may claim to be authors or owners of copyright of any of the
       images, and makes no warranties as to the quality of the images.
       Caltech shall not be responsible for any loss or expenses resulting
       from the use of the images, and you release and hold Caltech harmless
       from all liability arising from such use.
    \item to use a credit line in connection with images. Unless otherwise
       noted in the caption information for an image, the credit line should
       be ``Courtesy NASA/JPL-Caltech.''
    \item that the endorsement of any product or service by Caltech, JPL or
       NASA must not be claimed or implied.
    \end{enumerate}
    Special Cases:
	\begin{itemize}
    \item Prior written approval must be obtained to use the NASA insignia logo
      (the blue ``meatball'' insignia), the NASA logotype (the red ``worm''
      logo) and the NASA seal. These images may not be used by persons who
      are not NASA employees or on products (including Web pages) that are
      not NASA sponsored. In addition, no image may be used to explicitly
      or implicitly suggest endorsement by NASA, JPL or Caltech of
      commercial goods or services. Requests to use NASA logos may be
      directed to Bert Ulrich, Public Services Division, NASA Headquarters,
      Code POS, Washington, DC 20546, telephone (202) 358-1713, fax (202)
      358-4331, email \url{bert.ulrich (at) hq.nasa.gov}.
    \item Prior written approval must be obtained to use the JPL logo (stylized
      JPL letters in red or other colors). Requests to use the JPL logo may
      be directed to the Television/Imaging Team Leader, Media Relations
      Office, Mail Stop 186-120, Jet Propulsion Laboratory, Pasadena CA
      91109, telephone (818) 354-5011, fax (818) 354-4537.
    \item If an image includes an identifiable person, using the image for
      commercial purposes may infringe that person's right of privacy or
      publicity, and permission should be obtained from the person. NASA
      and JPL generally do not permit likenesses of current employees to
      appear on commercial products. For more information, consult the NASA
      and JPL points of contact listed above.
    \item JPL/Caltech contractors and vendors who wish to use JPL images in
      advertising or public relation materials should direct requests to the
      Television/Imaging Team Leader, Media Relations Office, Mail Stop
      186-120, Jet Propulsion Laboratory, Pasadena CA 91109, telephone
      (818) 354-5011, fax (818) 354-4537.
    \item Some image and video materials on JPL public web sites are owned by
      organizations other than JPL or NASA. These owners have agreed to
      make their images and video available for journalistic, educational
      and personal uses, but restrictions are placed on commercial uses.
      To obtain permission for commercial use, contact the copyright owner
      listed in each image caption.  Ownership of images and video by
      parties other than JPL and NASA is noted in the caption material
      with each image.
	\end{itemize}
\end{quotation}

\subsection{DSS}
\label{sec:ack:credits:dss}

From \url{http://archive.stsci.edu/dss/acknowledging.html}: %, as of 2018-02-06.
\begin{quotation}
\noindent The Digitized Sky Surveys were produced at the Space Telescope Science Institute under U.S. Government grant NAG W-2166. 
The images of these surveys are based on photographic data obtained using the Oschin Schmidt Telescope on Palomar Mountain and the UK Schmidt Telescope. 
The plates were processed into the present compressed digital form with the permission of these institutions.

The National Geographic Society - Palomar Observatory Sky Atlas (POSS-I) was made by the 
California Institute of Technology with grants from the National Geographic Society.

The Second Palomar Observatory Sky Survey (POSS-II) was made by the California Institute of Technology with 
funds from the National Science Foundation, the National Geographic Society, the Sloan Foundation, 
the Samuel Oschin Foundation, and the Eastman Kodak Corporation.

The Oschin Schmidt Telescope is operated by the California Institute of Technology and Palomar Observatory.

The UK Schmidt Telescope was operated by the Royal Observatory Edinburgh, with funding from the 
UK Science and Engineering Research Council (later the UK Particle Physics and Astronomy Research Council), 
until 1988 June, and thereafter by the Anglo-Australian Observatory. 
The blue plates of the southern Sky Atlas and its Equatorial Extension (together known as the SERC-J), 
as well as the Equatorial Red (ER), and the Second Epoch [red] Survey (SES) were all taken with the UK Schmidt.

All data are subject to the copyright given in the copyright summary\footnote{\url{http://archive.stsci.edu/dss/copyright.html}}. 
Copyright information specific to individual plates is provided in the downloaded FITS headers.

Supplemental funding for sky-survey work at the ST ScI is provided by the European Southern Observatory. 
\end{quotation}


%%% Local Variables: 
%%% mode: latex
%%% TeX-master: "guide"
%%% End: 
