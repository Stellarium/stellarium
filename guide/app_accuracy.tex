%% Stellarium user guide.
%% State: 2015-12 wiki->Guide. Was called Precision, but this is wrong.
%% 2016-04 GZ Typofixes, made proper chapter from this. 
%% TODO: (GZ) I am not sure about the 1-arcsecond accuracy of VSOP! More should be written with due diligence.
%% 2016-07-28 GZ Extended chapter with more than just VSOP notes.
%% 2017-12-10 GZ Extended with more warnings for "ignorant historians".
%% 2021-07-13 GZ Extended with numerical comparison results
%\chapterimage{chapter-t1-bg} % Chapter heading image


\chapter{Accuracy}
\label{ch:Accuracy}

Stellarium originally was developed to present a beautiful simulation
of the night sky, mostly to understand what is visible in the sky when
you leave your house, i.e., for present times. To save computation
time, some concessions were made in astronomical accuracy by using
simplified models which seemed acceptable at that time which was also
close to the standard epoch of J2000.0. However, time is progressing,
precession becomes noticeable when pointing telescopes to J2000.0 coordinates, and some 
users also started to overstress Stellarium's capabilities to simulate the
historical sky of many centuries in the past, and found --- or even
worse: did not find, but simply published --- the resulting
inconsistencies and errors.  Unfortunately, celestial motions are
indeed more complicated than a simple clockwork, and the process of
retrofitting detailed and accurate models which started around v0.11.0
(in 2011) has only recently (2021) \newFeature{0.21.2} come to a
state where we as developers are beginning to be satisfied. Now, after
implementation of annual aberration, Stellarium appears to deviate
from reference solutions in the present time only by fractions of
arcseconds, and by a few arcseconds in the remote past.  Greater
accuracy, including relativistic effects or the deflection of light by
the Solar gravity field, may still be required for critical
observations of stellar occultations by asteroids, grazing
occultations of stars by the Moon or similar things. Stellarium does
not compute these effects, and is definitely not usable for detailed
planning of your spacecraft missions to the planets.
There are more specialized programs for these topics, please use them.

Therefore, when using Stellarium for scientific work like eclipse
simulation to illustrate records found in Cuneiform tablets, always
also use some other reference to compare, and of course specify the
program version you used for your work. (Use at least version 25.1
of Stellarium, which introduced the much improved star catalog!)  
You can of course contact us if you bring a solid background
in fundamental astronomy and are willing and able to help improving
Stellarium's accuracy even further!

\section{Date Range}
\label{sec:Accuracy:DateRange}

The calendar dates are limited to a very wide interval of years, 
\newFeature{25.3}$-200,000\ldots+200,000$\footnote{Before version 25.3, the date range was $-100,000\ldots+100,000$.}. 
If you want to observe the other planets, this is way beyond what current ephemeris models are able to show, 
as explained in section~\ref{sec:Accuracy:Planets}! 
Observe for example the Moon: you will find that it seems to run 
on a polar orbit around +78.000 and even retrograde on a highly elongated orbit a few millennia after that. 
A quick check in -180.000 places it in a highly elongated orbit beyond Jupiter 
that takes it through the solar system at up to more than 10\% the speed of light.
This is obvious nonsense, caused by excessively extrapolating mathematical models in inappropriate ways. 

The extended date range is only useful to visualize the effect of stellar proper motions and the wobbling 
motion of Earth's axis known as precession. All other components of Stellarium have not made for this extended date range.

\section{Stellar Astrometry}
\label{sec:Accuracy:Astrometry}

Even though you can follow movements of stars on the sky for tens of thousands of years, 
please note that the precision of these predictions are limited by the precision of the data we have at 
the present time provided by Hipparcos for bright stars and Gaia.\newFeature{25.1} Most of the stars brighter than magnitude 10.5 
have 3D position and 3D velocity data (i.e., 6D astrometry) in our star catalogs, thus their 6D astrometry are computed 
at any epoch by assuming linear motion with constant velocity. These calculation provides arcseconds level accuracy
of stellar position, arcseconds per year level accuracy of proper motions, arcseconds level accuracy of parallax and km/s 
level accuracy of radial velocity in the previous and next tens of thousands of years. For stars fainter than magnitude 10.5 but brighter
than magnitude 16.75, 2D computation is performed such that RA/DEC positions are computed from RA/DEC proper motions separately and assume
constant proper motion. For stars fainter than magnitude 16.75, no astrometry propagation is performed hence their RA/DEC stay constant at any epoch.

The study of \citet{deLorenzis:2018} emphasized sky position errors with bright high proper motion stars like Arcturus, 
Dubhe, Sirius, Toliman and Vega which deviate noticeably from their correct locations even at around 2500BC with Stellarium 
0.17.0 released on 21 Dec 2017. The star catalogs and astrometry propagation algorithms have been significantly improved since Stellarium 25.1, which has 
solved the problem outlined in the study. 

% From Github PR#4500:
A quick estimate: For Barnard's Star which is nearby and moving fast, the sky position uncertainty on the sky 
would be $\approx0.05^\circ$ (200000~mas) at year 200,000 using Gaia DR3 astrometries and associated uncertainies. 
The 2015 DR3 sky position uncertainty is $\approx 0.04$~mas.

\section{Planetary Positions}
\label{sec:Accuracy:Planets}

Stellarium uses by default the widespread VSOP87 theory~\citep{1988A&A...202..309B} %\footnote{\url{http://vizier.cfa.harvard.edu/viz-bin/ftp-index?/ftp/cats/VI/81}}
to calculate the positions of the planets over time.
This is an analytical ephemeris modeled to match the numerical
integration run DE200 from NASA JPL, which by itself gives positions for 1850\ldots2050. 
Its use is recommended for the years $-4000\ldots+8000$. You can observe the sun leaving the ``ecliptic
of date'' and running on the ``ecliptic J2000'' outside this date
range. This is obviously a mathematical trick to keep
continuity. Still, positions may be somewhat useful outside this
range, but don't expect anything reliable 50,000 years in the past! 

The optionally usable JPL DE431 and DE441 (see \ref{sec:ExtraData:ephemerides}) deliver planet positions strictly for
-13000\ldots+17000 only, and nothing outside, but likely with higher accuracy. 
Outside of this range, positions from VSOP87 will be shown again.

As far as Stellarium is concerned, the user should bear in mind the
limits of the VSOP87 and other models (Table~\ref{tab:Accuracy:Planets}).
Accuracy values for VSOP87 are heliocentric as given by \citet{1988A&A...202..309B}.

\begin{table}[tb]
\begin{tabularx}{\textwidth}{X|l|X}
\toprule
\emph{Object(s)} & \emph{Method} & \emph{Notes}\tabularnewline
\midrule
Mercury, Venus, Earth-Moon barycenter, Mars & VSOP87 & Accuracy is 1 arc-second from 2000 B.C. -- 6000 A.D. \\%\midrule
Jupiter, Saturn                             & VSOP87 & Accuracy is 1 arc-second from 0 A.D. -- 4000 A.D.    \\%\midrule
Uranus, Neptune                             & VSOP87 & Accuracy is 1 arc-second from 4000 B.C. -- 8000 A.D. \\%\midrule
Pluto                                       & ?      & Pluto's position is valid only from 1885 A.D. -- 2099 A.D.\\%\midrule
Earth's Moon                                & ELP2000-82B & Fit against JPL DE200, which provides date range 1850--2050. 
                                                            Positions should be usable for some (unspecified) longer interval. \\%\midrule
Galilean satellites                         & L2     & Valid from 500 A.D. -- 3500 A.D.\\ 
\bottomrule
\end{tabularx}
\caption{Orbit-centric accuracy of used algorithms}
\label{tab:Accuracy:Planets}
\end{table}

\section{Minor Bodies}
\label{sec:Accuracy:MinorBodies}

Positions for the Minor Bodies (Dwarf Planets, Asteroids, Comets) are
computed with standard algorithms found in astronomical text
books. The generally used method of orbital elements allows computing
the positions of the respective object on an undisturbed Kepler orbit
around the sun. However, gravitational, and in the case of comets,
non-gravitational (outgassing) disturbances slowly change these
orbital elements. Therefore an \emph{epoch} is given for such
elements, and computation of positions for times far from this epoch
will lead to positional errors. Therefore, when searching for
asteroids or comets, always update your orbital elements (See section~\ref{sec:plugins:SolarSystemEditor}), 
and use elements with an epoch as close to your time of observation as
possible! Stellarium does not simulate gravitational perturbances and
orbital changes of minor bodies passing major planets.

\section{Precession and Nutation}
\label{sec:Accuracy:Precession}

Since v0.14.0, Stellarium computes the orientation of earth's axis
according to the IAU2006 Precession in a long-time variant developed
by \citet{2011AA:Vondrak, 2012AA:Vondrak}, and IAU2000B Nutation~\citep{Nutation:IAU2000B}.
This also now allows proper depiction of the changes in ecliptic obliquity
and display of ``instantaneous precession circles'' around the
ecliptic poles. These circles are indeed varying according to
ecliptical obliquity. Nutation is only computed for about 6000 years
around J2000.0. Nobody could have observed it before 1609, and it is
unclear for how long this model is applicable.

\section{Planet Axes}
\label{sec:Accuracy:PlanetAxes}

Where available, Stellarium uses \newFeature{0.21} axis orientation
and rotation data from the \indexterm{IAU} Working Group on
Cartographic Coordinates and Rotational Elements (\indexterm{WGCCRE}).
Orientation for some small moons is still simplified and
based on the model originally available. (See section~\ref{sec:ssystem.ini:Planet})

\section{Eclipses}
\label{sec:Accuracy:Eclipses}

The moon's motion is very complicated, and eclipse computations can be
tricky.  One aspect which every student of history, prehistory and
archaeology should know, but as it seems not every does, is at least a
basic understanding of the irregular slow-down of earth's rotation
known as $\Delta T$ (see section~\ref{sec:Concepts:DeltaT}).

Eclipse records on cuneiform tablets go back to the 8th century BC, some Chinese 
records go back a bit further. Experts on Earth rotation have provided models 
for $\Delta T$ based on such observations. These are usually given as parabolic fit, 
with some recommended time span. Extrapolating parabolic fits to erratic curves too far into the past is dangerous, 
pointless and inevitably leads to errors or invalid results. Models differ by many hours (exceeding a whole day!)
when applied too far in the past. This means, there was certainly an eclipse, and latitude will be OK, 
but you cannot say which longitude the eclipse was covering. The probability to have seen a total 
eclipse at a certain interesting location will be very small and should by itself not be used as positive argument 
for any reasonable statement. This is not a problem of Stellarium, 
but of current knowledge. To repeat, you may find a solar eclipse in the 6th millennium BC, 
but you cannot even be sure which side of Earth could have observed it!

\textbf{WANTED:} A model for $\Delta T$ that works in the Mesolithic! 

\section{The Calendar}
\label{sec:Accuracy:Calendar}

Stellarium follows the widely accepted convention that dates before the Gregorian calendar 
reform are given in the Julian calendar, introduced by \name[Julius]{Caesar} in 46~BC. 
This calendar introduces a leap day every fourth year, which is a bit too much. 
If you want to show summer solstice in the year $-2500$, you will notice it is not on June 21st. 
This is not an error of Stellarium -- the Gregorian reform had a reason! In prehistoric times, 
it may be more useful to work with solar ecliptic longitudes. For example, summer solstice is
when the Sun is at $\lambda=90\degree$.

\name[Julius]{Caesar} introduced a Solar calendar to the Roman empire
in 46~BC but was assassinated in March 44~BC=$-43$ (see below).  His
simple 4-year leap rule was then misapplied by the high priests of
Rome by adding a leap day every third year.  \name{Augustus} fixed the
calendar by omitting a few leap days, and only 8~AD was the first
correctly applied leap year as we know them \citep[\S186,
  \S189]{Ginzel:ChronologieII}. Be careful if you are working in
Augustean and earlier times in the Roman world, maybe even related to
named dates in the pre-Julian Roman calendar: Stellarium, also its
Calendars plugin (see section \ref{sec:plugin:Calendars}), does not
model these issues!


Some authors (esp.\ for Mesoamerican dates) give dates in the
proleptic Gregorian calendar (which means that they apply the Gregorian
leap year rules before 1582). This is a convention, they may of course
do that, but you must do the conversion to the Julian calendar to
visualize the sky with Stellarium.  If they also give Julian Day
numbers (JD), you can use them with Stellarium as JD numbers are
independent from calendar systems. Otherwise, you can utilize the Calendars
plugin (see section \ref{sec:plugin:Calendars}).

Some users have noted and expressed confusion about a difference of one between dates given as BC (or BCE) versus
Stellarium's negative years.  
Note that historical chronology does not know a year Zero! (Try to write this date in Roman numbers!)  
Year AD~1 (1~CE) was preceded by year 1~BC (1~BCE). 
Astronomers want to simplify computations and therefore use a year $0$ (equivalent to 1~BC). 
It is simple: 1~BC=$0$, 2~BC=$-1$, \ldots 100~BC=$-99$ (Fig.~\ref{fig:Accuracy:Calendar}). 
Stellarium is an astronomy program and uses astronomical counting. 
Other programs may behave differently in this respect, 
e.g.\ may even use negative signs to indicate years BC. 
However, the Calendars plugin (see section \ref{sec:plugin:Calendars}) 
even allows displaying both counting systems.

\begin{figure}[htb]\centering\small
  \begin{tabular}{l|ccccccccc}
    & \multicolumn{4}{r}{$\Longleftarrow$ negative}&
    \multicolumn{1}{c}{0}&
    \multicolumn{4}{l}{positive $\Longrightarrow$}\\
    Astronomical & \ldots & -3   & -2   & -1   & 0                        & 1    & 2    & 3    & \ldots\\\hline
    Historical   & \ldots & 4~BC & 3~BC & 2~BC &\multicolumn{1}{c||}{1~BC}& AD~1 & AD~2 & AD~3 & \ldots\\
    & \multicolumn{5}{r||}{$\Longleftarrow$ Before Christ}&
      \multicolumn{4}{l}{After Christ $\Longrightarrow$}
  \end{tabular}
  \caption{Counting years around the beginning of the Common Era.}
  \label{fig:Accuracy:Calendar}
\end{figure}
  

\section{Comparison to Reference Data}
\label{sec:Accuracy:JPLcomparison}

Tables~\ref{tab:accuracy:UT} and \ref{tab:accuracy:TT} \newFeature{0.21.2}
provide comparison data between JPL
Horizons\footnote{\url{https://ssd.jpl.nasa.gov/horizons.cgi}} and
Stellarium. The JPL data were fetched in July 2021 by our
attentive user Atque. These are apparent geocentric coordinates
provided by JPL's solutions DE440/DE441 in J2000.0 equatorial
coordinates, including JPL's corrections for nutation, light time,
aberration, light deflection in the Solar gravity field, and other
relativistic effects. The right part provides the respective values
computed by Stellarium from DE440/DE441 files, with the current state of corrections for
nutation, light time and aberration. The $\Delta T$ model was set to
``JPL Horizons'', but when using DE* files a slightly different model
specialized to the DE ephemeris is used\footnote{For experts: The value
  for secular acceleration of the Moon in ELP2000-82B, $\dot{n}=-23.8946$,
  is replaced by $\dot{n}=-25.8$.}, resulting in differences in the effective 
  value of $\Delta T$.  Among the causes for differences we must take into account:
\begin{itemize}
\item Stellarium does not provide corrections for diurnal aberration (should not matter for geocentric positions)
\item Stellarium does not provide corrections for light deflection in the Sun's gravitational field
%\item Stellarium provides nutation only for 6000 years around J2000.0. It is undocumented for how long the underlying model is valid.
\item JPL Horizons uses slightly different models than Stellarium for precession and nutation\footnote{\url{https://ssd.jpl.nasa.gov/?horizons_doc#specific_quantities}}
\end{itemize}
The first item should not make a difference. The next two may cause
differences in the range of a few arcseconds. The apparent drift that
we can observe in the remote past may stem from different assumptions
regarding $\Delta T$ (see \ref{sec:Concepts:DeltaT}) or from other
reasons yet to be identified. Competent help welcome!

\noindent
\colorbox{light-gray}{\fbox{\parbox[t]{0.975\linewidth}{
\begin{center}WANTED\end{center}

If you can trace down hitherto unknown reasons for remaining inaccuracies and either provide a fix or a decisive hint to us,  
we can offer a reward in the EUR200--500 range from our donation fund. We, the developers, will judge and evaluate relevance, legal recourse is excluded.\\

}}}



\scriptsize
\begin{longtable}{r@{\,}r|r@{h}r@{m}r<{s}|r@{°}r@{'}r<{''}||r@{h}r@{m}r<{s}|r@{°}r@{'}r<{''}}

\multicolumn{2}{c}{}&\multicolumn{6}{c||}{JPL Horizons} & \multicolumn{6}{c}{Stellarium 0.21.2$\beta$}\\ 
UT Date   & Time   & \multicolumn{3}{c}{RA J2000.0} & \multicolumn{3}{c||}{DE2000.0} & \multicolumn{3}{c}{$\alpha_{J2000.0}$} & \multicolumn{3}{c}{$\delta_{J2000.0}$} \\
\endhead
  
SUN\\
b2500-Jan-01 & 00:00  &   17 & 24 & 24.70 & -23 & 43 & 11.8 & 17&24&23.01 & -23&43&03.7\\ % -2499-01-01T00:00:00
b2500-Jun-02 & 00:00  &   03 & 06 & 17.57 & +17 & 54 & 00.7 & 03&06&16.20 & +17&53&49.0\\ % -2499-06-02T00:00:00
b2500-Nov-01 & 00:00  &   13 & 11 & 48.12 & -07 & 48 & 17.2 & 13&11&46.48 & -07&48&04.0\\ % -2499-11-01T00:00:00
b2499-Apr-02 & 00:00  &   23 & 26 & 28.14 & -03 & 42 & 34.9 & 23&26&26.37 & -03&42&45.1\\ % -2498-04-02T00:00:00
b0500-Jan-01 & 00:00  &   18 & 25 & 28.83 & -23 & 37 & 22.2 & 18&25&27.51 & -23&37&31.3\\ % -499-01-01T00:00:00
b0500-Jun-02 & 00:00  &   04 & 09 & 13.48 & +21 & 17 & 20.8 & 04&09&12.37 & +21&17&26.0\\ % -499-06-02T00:00:00
b0500-Nov-01 & 00:00  &   14 & 04 & 26.18 & -12 & 48 & 35.8 & 14&04&25.24 & -12&48&36.0\\ % -499-11-01T00:00:00
b0499-Apr-02 & 00:00  &   00 & 21 & 10.51 & +02 & 19 & 31.0 & 00&21&09.80 & +02&19&27.2\\ % -498-04-02T00:00:00
 0500-Jan-01 & 00:00  &   18 & 53 & 06.08 & -23 & 04 & 11.0 & 18&53&05.15 & -23&04&05.0\\ % 500-01-01T00:00:00
 0500-Jun-01 & 00:00  &   04 & 39 & 16.40 & +22 & 19 & 46.7 & 04&39&15.56 & +22&19&38.5\\ % 500-06-01T00:00:00
 0500-Oct-31 & 00:00  &   14 & 29 & 26.81 & -14 & 52 & 19.2 & 14&29&25.97 & -14&52&10.5\\ % 500-10-31T00:00:00
 0501-Apr-01 & 00:00  &   00 & 46 & 23.08 & +05 & 01 & 37.5 & 00&46&22.09 & +05&01&29.6\\ % 501-04-01T00:00:00
 0501-Aug-31 & 00:00  &   10 & 41 & 17.14 & +08 & 22 & 56.8 & 10&41&15.90 & +08&23&02.3\\ % 501-08-31T00:00:00
 1600-Jan-01 & 00:00  &   18 & 43 & 28.07 & -23 & 06 & 56.6 & 18&43&28.33 & -23&06&56.5\\ % 1600-01-01T00:00:00
 1600-Jun-01 & 00:00  &   04 & 35 & 37.18 & +22 & 04 & 25.0 & 04&35&37.43 & +22&04&25.6\\ % 1600-06-01T00:00:00
 1600-Oct-31 & 00:00  &   14 & 21 & 56.45 & -14 & 09 & 38.9 & 14&21&56.68 & -14&09&40.1\\ % 1600-10-31T00:00:00
 1601-Apr-01 & 00:00  &   00 & 41 & 18.48 & +04 & 27 & 19.4 & 00&41&18.70 & +04&27&20.8\\ % 1601-04-01T00:00:00
 1601-Aug-31 & 00:00  &   10 & 36 & 30.29 & +08 & 48 & 10.8 & 10&36&30.52 & +08&48&09.6\\ % 1601-08-31T00:00:00
 2021-Jan-01 & 00:00  &   18 & 46 & 53.59 & -22 & 59 & 57.2 & 18&46&53.58 & -22&59&57.2\\ % 2021-01-01T00:00:00
 2021-Jun-02 & 00:00  &   04 & 40 & 40.91 & +22 & 11 & 09.1 & 04&40&40.90 & +22&11&09.1\\ % 2021-06-02T00:00:00
 2021-Nov-01 & 00:00  &   14 & 25 & 34.78 & -14 & 25 & 31.0 & 14&25&34.78 & -14&25&30.9\\ % 2021-11-01T00:00:00
 2022-Apr-02 & 00:00  &   00 & 45 & 00.07 & +04 & 50 & 03.8 & 00&45&00.07 & +04&50&03.7\\ % 2022-04-02T00:00:00
MERCURY  \\
b2500-Jan-01 & 00:00  &   17 & 07 & 28.00 & -25 & 22 & 36.2 & 17&07&24.06 & -25&22&24.1\\ % -2499-01-01T00:00:00
b2500-Jun-02 & 00:00  &   04 & 57 & 29.05 & +22 & 59 & 25.2 & 04&57&27.89 & +22&59&23.8\\ % -2499-06-02T00:00:00
b2500-Nov-01 & 00:00  &   11 & 56 & 15.20 & +03 & 10 & 47.5 & 11&56&13.79 & +03&10&59.4\\ % -2499-11-01T00:00:00
b2499-Apr-02 & 00:00  &   23 & 03 & 07.53 & -07 & 14 & 26.7 & 23&03&03.05 & -07&14&59.7\\ % -2498-04-02T00:00:00
b0500-Jan-01 & 00:00  &   18 & 40 & 28.21 & -25 & 26 & 38.0 & 18&40&26.23 & -25&26&48.3\\ % -499-01-01T00:00:00
b0500-Jun-02 & 00:00  &   05 & 38 & 48.42 & +22 & 19 & 54.2 & 05&38&48.02 & +22&20&05.8\\ % -499-06-02T00:00:00
b0500-Nov-01 & 00:00  &   12 & 56 & 45.19 & -03 & 57 & 14.5 & 12&56&44.15 & -03&57&09.1\\ % -499-11-01T00:00:00
b0499-Apr-02 & 00:00  &   00 & 21 & 09.78 & +01 & 53 & 27.6 & 00&21&08.27 & +01&53&16.7\\ % -498-04-02T00:00:00
 0500-Jan-01 & 00:00  &   18 & 23 & 39.86 & -25 & 14 & 04.0 & 18&23&38.57 & -25&13&56.9\\ % 500-01-01T00:00:00
 0500-Jun-01 & 00:00  &   06 & 29 & 34.30 & +24 & 32 & 58.8 & 06&29&33.26 & +24&32&53.9\\ % 500-06-01T00:00:00
 0500-Oct-31 & 00:00  &   13 & 28 & 10.47 & -06 & 51 & 07.8 & 13&28&10.12 & -06&51&03.8\\ % 500-10-31T00:00:00
 0501-Apr-01 & 00:00  &   23 & 55 & 15.72 & -02 & 45 & 50.1 & 23&55&14.26 & -02&46&00.2\\ % 501-04-01T00:00:00
 0501-Aug-31 & 00:00  &   12 & 03 & 40.27 & -01 & 48 & 53.7 & 12&03&38.96 & -01&48&43.7\\ % 501-08-31T00:00:00
 1600-Jan-01 & 00:00  &   19 & 16 & 55.37 & -24 & 27 & 04.8 & 19&16&55.69 & -24&27&04.5\\ % 1600-01-01T00:00:00
 1600-Jun-01 & 00:00  &   05 & 27 & 00.75 & +22 & 06 & 49.9 & 05&27&00.89 & +22&06&49.8\\ % 1600-06-01T00:00:00
 1600-Oct-31 & 00:00  &   13 & 24 & 30.87 & -06 & 45 & 17.1 & 13&24&31.13 & -06&45&18.8\\ % 1600-10-31T00:00:00
 1601-Apr-01 & 00:00  &   00 & 45 & 30.34 & +03 & 58 & 05.0 & 00&45&30.64 & +03&58&07.2\\ % 1601-04-01T00:00:00
 1601-Aug-31 & 00:00  &   12 & 08 & 26.65 & -04 & 19 & 14.8 & 12&08&26.83 & -04&19&16.2\\ % 1601-08-31T00:00:00
 2021-Jan-01 & 00:00  &   19 & 17 & 59.43 & -24 & 20 & 58.1 & 19&17&59.41 & -24&20&58.2\\ % 2021-01-01T00:00:00
 2021-Jun-02 & 00:00  &   05 & 35 & 35.59 & +22 & 40 & 50.7 & 05&35&35.59 & +22&40&50.7\\ % 2021-06-02T00:00:00
 2021-Nov-01 & 00:00  &   13 & 25 & 55.55 & -06 & 47 & 18.8 & 13&25&55.54 & -06&47&18.7\\ % 2021-11-01T00:00:00
 2022-Apr-02 & 00:00  &   00 & 43 & 05.35 & +03 & 21 & 44.7 & 00&43&05.34 & +03&21&44.7\\ % 2022-04-02T00:00:00
VENUS    \\
b2500-Jan-01 & 00:00  &   14 & 38 & 40.01 & -10 & 33 & 46.3 & 14&38&39.62 & -10&33&38.6\\ % -2499-01-01T00:00:00
b2500-Jun-02 & 00:00  &   01 & 31 & 36.91 & +08 & 30 & 50.3 & 01&31&35.00 & +08&30&34.0\\ % -2499-06-02T00:00:00
b2500-Nov-01 & 00:00  &   14 & 04 & 33.57 & -13 & 23 & 07.7 & 14&04&31.36 & -13&22&49.0\\ % -2499-11-01T00:00:00
b2499-Apr-02 & 00:00  &   02 & 07 & 22.90 & +16 & 14 & 49.7 & 02&07&21.01 & +16&14&32.7\\ % -2498-04-02T00:00:00
b0500-Jan-01 & 00:00  &   15 & 15 & 48.26 & -13 & 58 & 19.4 & 15&15&47.07 & -13&58&20.9\\ % -499-01-01T00:00:00
b0500-Jun-02 & 00:00  &   02 & 41 & 50.72 & +14 & 49 & 02.0 & 02&41&49.47 & +14&49&02.0\\ % -499-06-02T00:00:00
b0500-Nov-01 & 00:00  &   15 & 14 & 41.57 & -18 & 52 & 17.3 & 15&14&40.38 & -18&52&19.3\\ % -499-11-01T00:00:00
b0499-Apr-02 & 00:00  &   03 & 10 & 27.71 & +21 & 27 & 17.2 & 03&10&26.78 & +21&27&20.3\\ % -498-04-02T00:00:00
 0500-Jan-01 & 00:00  &   20 & 18 & 43.11 & -15 & 30 & 09.1 & 20&18&42.96 & -15&30&05.4\\ % 500-01-01T00:00:00
 0500-Jun-01 & 00:00  &   02 & 11 & 10.44 & +11 & 10 & 10.0 & 02&11&09.60 & +11&10&01.3\\ % 500-06-01T00:00:00
 0500-Oct-31 & 00:00  &   14 & 33 & 32.68 & -14 & 39 & 03.4 & 14&33&31.70 & -14&38&53.8\\ % 500-10-31T00:00:00
 0501-Apr-01 & 00:00  &   03 & 03 & 13.09 & +18 & 47 & 05.5 & 03&03&11.96 & +18&46&55.3\\ % 501-04-01T00:00:00
 0501-Aug-31 & 00:00  &   09 & 10 & 52.83 & +08 & 18 & 44.4 & 09&10&52.19 & +08&18&42.3\\ % 501-08-31T00:00:00
 1600-Jan-01 & 00:00  &   17 & 09 & 26.57 & -18 & 08 & 55.1 & 17&09&26.71 & -18&08&55.3\\ % 1600-01-01T00:00:00
 1600-Jun-01 & 00:00  &   02 & 28 & 53.12 & +12 & 50 & 48.1 & 02&28&53.37 & +12&50&49.4\\ % 1600-06-01T00:00:00
 1600-Oct-31 & 00:00  &   14 & 54 & 13.53 & -16 & 15 & 03.3 & 14&54&13.79 & -16&15&04.6\\ % 1600-10-31T00:00:00
 1601-Apr-01 & 00:00  &   03 & 18 & 37.64 & +20 & 06 & 01.5 & 03&18&37.90 & +20&06&02.6\\ % 1601-04-01T00:00:00
 1601-Aug-31 & 00:00  &   07 & 48 & 52.93 & +16 & 06 & 13.5 & 07&48&53.14 & +16&06&13.3\\ % 1601-08-31T00:00:00
 2021-Jan-01 & 00:00  &   17 & 18 & 28.43 & -22 & 26 & 01.6 & 17&18&28.42 & -22&26&01.5\\ % 2021-01-01T00:00:00
 2021-Jun-02 & 00:00  &   05 & 57 & 01.46 & +24 & 21 & 39.2 & 05&57&01.45 & +24&21&39.2\\ % 2021-06-02T00:00:00
 2021-Nov-01 & 00:00  &   17 & 40 & 41.73 & -27 & 05 & 05.3 & 17&40&41.73 & -27&05&05.3\\ % 2021-11-01T00:00:00
 2022-Apr-02 & 00:00  &   21 & 52 & 25.36 & -12 & 04 & 41.5 & 21&52&25.35 & -12&04&41.5\\ % 2022-04-02T00:00:00
MOON   \\
b2500-Jan-01 & 00:00  &   00 & 13 & 40.60 & -00 & 43 & 59.3 & 00&13&13.21 & -00&47&46.8\\ % -2499-01-01T00:00:00
b2500-Jun-02 & 00:00  &   12 & 48 & 51.83 & -04 & 50 & 11.6 & 12&48&20.54 & -04&45&51.2\\ % -2499-06-02T00:00:00
b2500-Nov-01 & 00:00  &   02 & 49 & 20.95 & +19 & 57 & 17.4 & 02&48&50.55 & +19&54&16.7\\ % -2499-11-01T00:00:00
b2499-Apr-02 & 00:00  &   15 & 45 & 48.47 & -25 & 00 & 31.4 & 15&45&11.28 & -24&57&59.1\\ % -2498-04-02T00:00:00
b0500-Jan-01 & 00:00  &   02 & 14 & 28.99 & +14 & 10 & 37.4 & 02&14&19.51 & +14&10&07.3\\ % -499-01-01T00:00:00
b0500-Jun-02 & 00:00  &   15 & 02 & 34.83 & -16 & 05 & 08.5 & 15&02&24.27 & -16&04&46.0\\ % -499-06-02T00:00:00
b0500-Nov-01 & 00:00  &   05 & 18 & 41.70 & +19 & 00 & 42.7 & 05&18&32.25 & +19&00&50.6\\ % -499-11-01T00:00:00
b0499-Apr-02 & 00:00  &   18 & 10 & 41.75 & -18 & 33 & 49.1 & 18&10&30.48 & -18&34&06.2\\ % -498-04-02T00:00:00
 0500-Jan-01 & 00:00  &   05 & 33 & 09.71 & +28 & 18 & 25.7 & 05&33&03.58 & +28&18&17.9\\ % 500-01-01T00:00:00
 0500-Jun-01 & 00:00  &   19 & 39 & 46.24 & -24 & 36 & 20.7 & 19&39&38.32 & -24&36&40.8\\ % 500-06-01T00:00:00
 0500-Oct-31 & 00:00  &   08 & 41 & 09.22 & +19 & 42 & 14.7 & 08&41&02.31 & +19&42&43.4\\ % 500-10-31T00:00:00
 0501-Apr-01 & 00:00  &   22 & 26 & 23.71 & -07 & 51 & 05.2 & 22&26&15.08 & -07&52&03.8\\ % 501-04-01T00:00:00
 0501-Aug-31 & 00:00  &   11 & 28 & 24.37 & -00 & 41 & 25.5 & 11&28&16.66 & -00&40&29.8\\ % 501-08-31T00:00:00
 1600-Jan-01 & 00:00  &   07 & 04 & 55.02 & +24 & 04 & 37.3 & 07&04&57.90 & +24&04&31.5\\ % 1600-01-01T00:00:00
 1600-Jun-01 & 00:00  &   19 & 54 & 15.59 & -20 & 26 & 36.3 & 19&54&17.86 & -20&26&28.9\\ % 1600-06-01T00:00:00
 1600-Oct-31 & 00:00  &   10 & 11 & 24.53 & +07 & 13 & 09.4 & 10&11&25.16 & +07&13&06.3\\ % 1600-10-31T00:00:00
 1601-Apr-01 & 00:00  &   22 & 46 & 21.63 & -02 & 54 & 30.7 & 22&46&21.77 & -02&54&30.0\\ % 1601-04-01T00:00:00
 1601-Aug-31 & 00:00  &   12 & 52 & 10.20 & -10 & 52 & 36.0 & 12&52&10.35 & -10&52&37.7\\ % 1601-08-31T00:00:00
 2021-Jan-01 & 00:00  &   08 & 23 & 36.67 & +23 & 01 & 21.2 & 08&23&37.89 & +23&01&16.5\\ % 2021-01-01T00:00:00
 2021-Jun-02 & 00:00  &   22 & 46 & 54.03 & -13 & 27 & 21.8 & 22&46&53.95 & -13&27&20.3\\ % 2021-06-02T00:00:00
 2021-Nov-01 & 00:00  &   11 & 18 & 43.58 & +09 & 57 & 42.8 & 11&18&42.68 & +09&57&47.0\\ % 2021-11-01T00:00:00
 2022-Apr-02 & 00:00  &   01 & 21 & 51.21 & +05 & 40 & 00.7 & 01&21&49.88 & +05&39&52.4\\ % 2022-04-02T00:00:00
MARS       \\
 1601-Jan-01 & 00:00  &   19 & 26 & 56.21 & -23 & 01 & 42.7 & 19&26&56.45 & -23&01&42.4\\ % 1601-01-01T00:00:00
 1601-Jun-02 & 00:00  &   02 & 59 & 06.93 & +16 & 41 & 21.5 & 02&59&07.14 & +16&41&22.6\\ % 1601-06-02T00:00:00
 1601-Nov-01 & 00:00  &   09 & 48 & 48.55 & +15 & 01 & 38.0 & 09&48&48.75 & +15&01&37.2\\ % 1601-11-01T00:00:00
 1602-Apr-02 & 00:00  &   10 & 25 & 38.57 & +13 & 17 & 29.7 & 10&25&38.70 & +13&17&29.0\\ % 1602-04-02T00:00:00
 2021-Jan-01 & 00:00  &   01 & 40 & 18.14 & +11 & 20 & 41.2 & 01&40&18.14 & +11&20&41.1\\ % 2021-01-01T00:00:00
 2021-Jun-02 & 00:00  &   07 & 45 & 03.73 & +22 & 37 & 21.0 & 07&45&03.72 & +22&37&21.0\\ % 2021-06-02T00:00:00
 2021-Nov-01 & 00:00  &   13 & 55 & 54.00 & -11 & 22 & 08.1 & 13&55&54.00 & -11&22&08.1\\ % 2021-11-01T00:00:00
 2022-Apr-02 & 00:00  &   21 & 31 & 33.53 & -15 & 56 & 58.5 & 21&31&33.53 & -15&56&58.5\\ % 2022-04-02T00:00:00
JUPITER          \\
 1601-Jan-01 & 00:00  &   11 & 33 & 23.43 & +04 & 18 & 09.3 & 11&33&23.58 & +04&18&08.4\\ % 1601-01-01T00:00:00
 1601-Jun-02 & 00:00  &   10 & 59 & 49.52 & +07 & 52 & 18.0 & 10&59&49.67 & +07&52&17.1\\ % 1601-06-02T00:00:00
 1601-Nov-01 & 00:00  &   12 & 39 & 41.32 & -03 & 02 & 14.3 & 12&39&41.47 & -03&02&15.3\\ % 1601-11-01T00:00:00
 1602-Apr-02 & 00:00  &   13 & 08 & 44.17 & -05 & 35 & 33.6 & 13&08&44.30 & -05&35&34.4\\ % 1602-04-02T00:00:00
 2021-Jan-01 & 00:00  &   20 & 20 & 44.62 & -20 & 00 & 44.3 & 20&20&44.62 & -20&00&44.4\\ % 2021-01-01T00:00:00
 2021-Jun-02 & 00:00  &   22 & 15 & 51.65 & -11 & 42 & 45.3 & 22&15&51.65 & -11&42&45.3\\ % 2021-06-02T00:00:00
 2021-Nov-01 & 00:00  &   21 & 41 & 29.27 & -15 & 01 & 08.9 & 21&41&29.27 & -15&01&08.9\\ % 2021-11-01T00:00:00
 2022-Apr-02 & 00:00  &   23 & 30 & 29.23 & -04 & 17 & 08.2 & 23&30&29.24 & -04&17&08.2\\ % 2022-04-02T00:00:00
SATURN           \\
 1750-Jan-01 & 00:00  &   15 & 46 & 12.27 & -17 & 54 & 37.7 & 15&46&12.37 & -17&54&38.1\\ % 1750-01-01T00:00:00
 1750-Jun-02 & 00:00  &   15 & 44 & 23.78 & -17 & 34 & 00.9 & 15&44&23.87 & -17&34&01.3\\ % 1750-06-02T00:00:00
 1750-Nov-01 & 00:00  &   15 & 59 & 19.00 & -18 & 50 & 41.0 & 15&59&19.10 & -18&50&41.4\\ % 1750-11-01T00:00:00
 1751-Apr-02 & 00:00  &   16 & 49 & 11.72 & -20 & 39 & 30.2 & 16&49&11.82 & -20&39&30.5\\ % 1751-04-02T00:00:00
 2021-Jan-01 & 00:00  &   20 & 15 & 50.41 & -20 & 10 & 28.7 & 20&15&50.41 & -20&10&28.7\\ % 2021-01-01T00:00:00
 2021-Jun-02 & 00:00  &   21 & 04 & 25.24 & -17 & 24 & 46.6 & 21&04&25.24 & -17&24&46.6\\ % 2021-06-02T00:00:00
 2021-Nov-01 & 00:00  &   20 & 39 & 27.72 & -19 & 14 & 59.7 & 20&39&27.72 & -19&14&59.7\\ % 2021-11-01T00:00:00
 2022-Apr-02 & 00:00  &   21 & 39 & 14.50 & -15 & 01 & 42.2 & 21&39&14.50 & -15&01&42.2\\ % 2022-04-02T00:00:00
URANUS      \\
 1600-Jan-01 & 00:00  &   01 & 41 & 51.34 & +10 & 01 & 26.1 & 01&41&51.49 & +10&01&26.9\\ % 1600-01-01T00:00:00
 1600-Jun-01 & 00:00  &   02 & 05 & 21.58 & +12 & 13 & 45.6 & 02&05&21.73 & +12&13&46.4\\ % 1600-06-01T00:00:00
 1600-Oct-31 & 00:00  &   02 & 04 & 09.95 & +12 & 05 & 39.9 & 02&04&10.09 & +12&05&40.7\\ % 1600-10-31T00:00:00
 1601-Apr-01 & 00:00  &   02 & 07 & 08.63 & +12 & 25 & 31.5 & 02&07&08.78 & +12&25&32.4\\ % 1601-04-01T00:00:00
 1601-Aug-31 & 00:00  &   02 & 28 & 09.10 & +14 & 11 & 23.3 & 02&28&09.24 & +14&11&24.1\\ % 1601-08-31T00:00:00
 2021-Jan-01 & 00:00  &   02 & 18 & 27.82 & +13 & 21 & 27.5 & 02&18&27.83 & +13&21&27.5\\ % 2021-01-01T00:00:00
 2021-Jun-02 & 00:00  &   02 & 40 & 37.83 & +15 & 11 & 44.7 & 02&40&37.84 & +15&11&44.7\\ % 2021-06-02T00:00:00
 2021-Nov-01 & 00:00  &   02 & 42 & 31.60 & +15 & 19 & 13.5 & 02&42&31.60 & +15&19&13.5\\ % 2021-11-01T00:00:00
 2022-Apr-02 & 00:00  &   02 & 42 & 25.31 & +15 & 22 & 19.7 & 02&42&25.31 & +15&22&19.7\\ % 2022-04-02T00:00:00
NEPTUNE     \\
 1801-Jan-01 & 00:00  &   15 & 07 & 04.41 & -15 & 44 & 47.2 & 15&07&04.49 & -15&44&47.6\\ % 1801-01-01T00:00:00
 1801-Jun-02 & 00:00  &   15 & 01 & 57.08 & -15 & 18 & 05.9 & 15&01&57.15 & -15&18&06.2\\ % 1801-06-02T00:00:00
 1801-Nov-01 & 00:00  &   15 & 07 & 02.36 & -15 & 46 & 26.5 & 15&07&02.43 & -15&46&26.9\\ % 1801-11-01T00:00:00
 1802-Apr-02 & 00:00  &   15 & 17 & 01.46 & -16 & 20 & 15.8 & 15&17&01.54 & -16&20&16.2\\ % 1802-04-02T00:00:00
 2021-Jan-01 & 00:00  &   23 & 19 & 18.38 & -05 & 33 & 12.4 & 23&19&18.39 & -05&33&12.4\\ % 2021-01-01T00:00:00
 2021-Jun-02 & 00:00  &   23 & 36 & 13.26 & -03 & 46 & 42.9 & 23&36&13.26 & -03&46&42.9\\ % 2021-06-02T00:00:00
 2021-Nov-01 & 00:00  &   23 & 27 & 30.35 & -04 & 46 & 19.7 & 23&27&30.35 & -04&46&19.7\\ % 2021-11-01T00:00:00
 2022-Apr-02 & 00:00  &   23 & 38 & 17.59 & -03 & 34 & 19.0 & 23&38&17.59 & -03&34&19.0\\ % 2022-04-02T00:00:00
IO(JI)    \\
 1601-Jan-01 & 00:00  &   11 & 33 & 30.36 & +04 & 17 & 19.7 & 11&33&30.51 & +04&17&18.9\\ % 1601-01-01T00:00:00
 1601-Jun-02 & 00:00  &   10 & 59 & 55.45 & +07 & 51 & 35.7 & 10&59&55.61 & +07&51&34.8\\ % 1601-06-02T00:00:00
 1601-Nov-01 & 00:00  &   12 & 39 & 41.63 & -03 & 02 & 20.8 & 12&39&41.81 & -03&02&22.0\\ % 1601-11-01T00:00:00
 1602-Apr-02 & 00:00  &   13 & 08 & 39.94 & -05 & 35 & 12.3 & 13&08&40.11 & -05&35&13.4\\ % 1602-04-02T00:00:00
 2021-Jan-01 & 00:00  &   20 & 20 & 41.43 & -20 & 00 & 57.7 & 20&20&41.43 & -20&00&57.7\\ % 2021-01-01T00:00:00
 2021-Jun-02 & 00:00  &   22 & 15 & 44.01 & -11 & 43 & 33.2 & 22&15&44.01 & -11&43&33.2\\ % 2021-06-02T00:00:00
 2021-Nov-01 & 00:00  &   21 & 41 & 21.65 & -15 & 01 & 52.3 & 21&41&21.65 & -15&01&52.3\\ % 2021-11-01T00:00:00
 2022-Apr-02 & 00:00  &   23 & 30 & 27.92 & -04 & 17 & 20.4 & 23&30&27.93 & -04&17&20.4\\ % 2022-04-02T00:00:00
EUROPA(JII)      \\
 1601-Jan-01 & 00:00  &   11 & 33 & 16.17 & +04 & 19 & 07.1 & 11&33&16.29 & +04&19&06.3\\ % 1601-01-01T00:00:00
 1601-Jun-02 & 00:00  &   10 & 59 & 53.24 & +07 & 51 & 59.3 & 10&59&53.37 & +07&51&58.7\\ % 1601-06-02T00:00:00
 1601-Nov-01 & 00:00  &   12 & 39 & 49.09 & -03 & 03 & 11.4 & 12&39&49.26 & -03&03&12.3\\ % 1601-11-01T00:00:00
 1602-Apr-02 & 00:00  &   13 & 08 & 40.13 & -05 & 35 & 20.4 & 13&08&40.28 & -05&35&21.3\\ % 1602-04-02T00:00:00
 2021-Jan-01 & 00:00  &   20 & 20 & 34.16 & -20 & 01 & 24.9 & 20&20&34.16 & -20&01&24.9\\ % 2021-01-01T00:00:00
 2021-Jun-02 & 00:00  &   22 & 15 & 55.57 & -11 & 42 & 23.5 & 22&15&55.57 & -11&42&23.5\\ % 2021-06-02T00:00:00
 2021-Nov-01 & 00:00  &   21 & 41 & 41.68 & -14 & 59 & 58.2 & 21&41&41.68 & -14&59&58.2\\ % 2021-11-01T00:00:00
 2022-Apr-02 & 00:00  &   23 & 30 & 29.10 & -04 & 17 & 03.2 & 23&30&29.11 & -04&17&03.3\\ % 2022-04-02T00:00:00
GANYMEDE(JIII)\\
 1601-Jan-01 & 00:00  &   11 & 33 & 29.35 & +04 & 17 & 17.1 & 11&33&29.49 & +04&17&16.2\\ % 1601-01-01T00:00:00
 1601-Jun-02 & 00:00  &   11 & 00 & 04.63 & +07 & 50 & 38.0 & 11&00&04.78 & +07&50&37.2\\ % 1601-06-02T00:00:00
 1601-Nov-01 & 00:00  &   12 & 39 & 41.91 & -03 & 02 & 07.8 & 12&39&42.07 & -03&02&08.7\\ % 1601-11-01T00:00:00
 1602-Apr-02 & 00:00  &   13 & 08 & 24.28 & -05 & 33 & 17.7 & 13&08&24.40 & -05&33&18.6\\ % 1602-04-02T00:00:00
 2021-Jan-01 & 00:00  &   20 & 20 & 33.16 & -20 & 01 & 30.9 & 20&20&33.15 & -20&01&30.9\\ % 2021-01-01T00:00:00
 2021-Jun-02 & 00:00  &   22 & 15 & 57.19 & -11 & 42 & 07.4 & 22&15&57.19 & -11&42&07.5\\ % 2021-06-02T00:00:00
 2021-Nov-01 & 00:00  &   21 & 41 & 47.81 & -14 & 59 & 25.0 & 21&41&47.82 & -14&59&25.0\\ % 2021-11-01T00:00:00
 2022-Apr-02 & 00:00  &   23 & 30 & 31.22 & -04 & 17 & 01.0 & 23&30&31.23 & -04&17&01.0\\ % 2022-04-02T00:00:00
CALLISTO(JIV) \\                                                                          
 1601-Jan-01 & 00:00  &   11 & 33 & 15.59 & +04 & 19 & 21.5 & 11&33&15.73 & +04&19&20.6\\ % 1601-01-01T00:00:00
 1601-Jun-02 & 00:00  &   10 & 59 & 22.98 & +07 & 55 & 21.8 & 10&59&23.12 & +07&55&20.9\\ % 1601-06-02T00:00:00
 1601-Nov-01 & 00:00  &   12 & 39 & 17.32 & -02 & 59 & 27.6 & 12&39&17.47 & -02&59&28.8\\ % 1601-11-01T00:00:00
 1602-Apr-02 & 00:00  &   13 & 08 & 09.98 & -05 & 32 & 01.2 & 13&08&10.10 & -05&32&02.1\\ % 1602-04-02T00:00:00
 2021-Jan-01 & 00:00  &   20 & 20 & 42.72 & -20 & 00 & 56.3 & 20&20&42.73 & -20&00&56.3\\ % 2021-01-01T00:00:00
 2021-Jun-02 & 00:00  &   22 & 15 & 55.40 & -11 & 42 & 15.1 & 22&15&55.41 & -11&42&15.1\\ % 2021-06-02T00:00:00
 2021-Nov-01 & 00:00  &   21 & 41 & 58.24 & -14 & 58 & 23.9 & 21&41&58.25 & -14&58&23.9\\ % 2021-11-01T00:00:00
 2022-Apr-02 & 00:00  &   23 & 30 & 53.15 & -04 & 14 & 15.0 & 23&30&53.15 & -04&14&15.0\\ % 2022-04-02T00:00:00

\end{longtable}

% Part of Stellarium User Guide.
% This table compares results from JPL Horizons to Stellarium. 
% We compare results in TT to avoid issues with DeltaT.

\begin{longtable}{r@{\,}r|r@{h}r@{m}r<{s}|r@{°}r@{'}r<{''}|r<{s}||r@{h}r@{m}r<{s}|r@{°}r@{'}r<{''}}

\caption{An ephemeris comparison between Stellarium and JPL Horizons, based on JPL DE440/441. 
Both are showing geocentric apparent coordinates (with slightly 
different models for precession, nutation and aberration; JPL results are also 
corrected for light deflection). This table shows ephemeris comparison in Dynamical time (TT), to avoid showing divergences which are only caused by differences in $\Delta T$ handling. 
(JPL's values for $\Delta T$ are listed for reference, but not used.)}
\label{tab:accuracy:TT}\\
\multicolumn{2}{c}{}&\multicolumn{7}{c||}{JPL Horizons} & \multicolumn{6}{c}{Stellarium 0.21.2$\beta$}\\ 
TT Date   & Time & \multicolumn{3}{c}{RA J2000.0} & \multicolumn{3}{c}{DE J2000.0} & \multicolumn{1}{c||}{$\Delta T$}& \multicolumn{3}{c}{$\alpha_{J2000.0}$} & \multicolumn{3}{c}{$\delta_{J2000.0}$} \\
\midrule\endfirsthead

\multicolumn{2}{c}{}&\multicolumn{7}{c||}{JPL Horizons} & \multicolumn{6}{c}{Stellarium 0.21.2$\beta$}\\ 
TT Date   & Time & \multicolumn{3}{c}{RA J2000.0} & \multicolumn{3}{c}{DE J2000.0} & \multicolumn{1}{c||}{$\Delta T$}& \multicolumn{3}{c}{$\alpha_{J2000.0}$} & \multicolumn{3}{c}{$\delta_{J2000.0}$}  \\
\midrule\endhead
  
SUN\\
b2500-Jan-01 &00:00   &  17 & 21 & 25.18 & -23 & 40 & 30.1  & 58893.015756 & 17&21&26.91 & -23&40&34.2 \\ % -2499-01-01T00:00:00
b2500-Jun-02 &00:00   &  03 & 03 & 40.51 & +17 & 43 & 04.0  & 58881.693980 & 03&03&42.09 & +17&43&12.5 \\ % -2499-06-02T00:00:00
b2500-Nov-01 &00:00   &  13 & 09 & 12.65 & -07 & 32 & 06.5  & 58870.373294 & 13&09&14.12 & -07&32&16.5 \\ % -2499-11-01T00:00:00
b2499-Apr-02 &00:00   &  23 & 24 & 04.53 & -03 & 58 & 17.6  & 58859.053698 & 23&24&06.02 & -03&58&08.4 \\ % -2498-04-02T00:00:00
b0500-Jan-01 &00:00   &  18 & 24 & 36.63 & -23 & 37 & 53.7  & 16988.259557 & 18&24&37.49 & -23&37&54.4 \\ % -499-01-01T00:00:00
b0500-Jun-02 &00:00   &  04 & 08 & 26.19 & +21 & 15 & 14.4  & 16982.002779 & 04&08&26.99 & +21&15&17.6 \\ % -499-06-02T00:00:00
b0500-Nov-01 &00:00   &  14 & 03 & 40.03 & -12 & 44 & 27.5  & 16975.747099 & 14&03&40.76 & -12&44&32.1 \\ % -499-11-01T00:00:00
b0499-Apr-02 &00:00   &  00 & 20 & 28.76 & +02 & 14 & 57.0  & 16969.492518 & 00&20&29.48 & +02&15&01.7 \\ % -498-04-02T00:00:00
 0500-Jan-01 &00:00   &  18 & 52 & 48.93 & -23 & 04 & 32.9  &  5608.981515 & 18&52&49.49 & -23&04&32.9 \\ % 500-01-01T00:00:00
 0500-Jun-01 &00:00   &  04 & 39 & 00.52 & +22 & 19 & 15.9  &  5605.028887 & 04&39&01.07 & +22&19&17.7 \\ % 500-06-01T00:00:00
 0500-Oct-31 &00:00   &  14 & 29 & 11.35 & -14 & 51 & 03.8  &  5601.076499 & 14&29&11.84 & -14&51&06.7 \\ % 500-10-31T00:00:00
 0501-Apr-01 &00:00   &  00 & 46 & 09.14 & +05 & 00 & 08.5  &  5597.124355 & 00&46&09.61 & +05&00&11.6 \\ % 501-04-01T00:00:00
 0501-Aug-31 &00:00   &  10 & 41 & 02.87 & +08 & 24 & 23.2  &  5593.172456 & 10&41&03.36 & +08&24&20.5 \\ % 501-08-31T00:00:00
 1600-Jan-01 &00:00   &  18 & 43 & 27.73 & -23 & 06 & 56.9  &   110.462079 & 18&43&27.90 & -23&06&57.0 \\ % 1600-01-01T00:00:00
 1600-Jun-01 &00:00   &  04 & 35 & 36.87 & +22 & 04 & 24.3  &   109.804702 & 04&35&37.03 & +22&04&24.8 \\ % 1600-06-01T00:00:00
 1600-Oct-31 &00:00   &  14 & 21 & 56.15 & -14 & 09 & 37.4  &   109.148794 & 14&21&56.30 & -14&09&38.2 \\ % 1600-10-31T00:00:00
 1601-Apr-01 &00:00   &  00 & 41 & 18.21 & +04 & 27 & 17.7  &   108.494367 & 00&41&18.35 & +04&27&18.6 \\ % 1601-04-01T00:00:00
 1601-Aug-31 &00:00   &  10 & 36 & 30.02 & +08 & 48 & 12.4  &   107.841429 & 10&36&30.16 & +08&48&11.7 \\ % 1601-08-31T00:00:00
MERCURY\\
b2500-Jan-01 &00:00   &  17 & 02 & 08.10 & -25 & 14 & 54.4  & 58893.015756 & 17&02&09.83 & -25&14&59.3 \\ % -2499-01-01T00:00:00
b2500-Jun-02 &00:00   &  04 & 55 & 00.41 & +23 & 03 & 30.5  & 58881.693980 & 04&55&02.12 & +23&03&35.5 \\ % -2499-06-02T00:00:00
b2500-Nov-01 &00:00   &  11 & 54 & 04.49 & +03 & 27 & 59.5  & 58870.373294 & 11&54&05.97 & +03&27&49.8 \\ % -2499-11-01T00:00:00
b2499-Apr-02 &00:00   &  22 & 57 & 59.14 & -07 & 54 & 02.4  & 58859.053698 & 22&58&00.66 & -07&53&53.6 \\ % -2498-04-02T00:00:00
b0500-Jan-01 &00:00   &  18 & 38 & 56.17 & -25 & 28 & 13.0  & 16988.259557 & 18&38&57.04 & -25&28&13.3 \\ % -499-01-01T00:00:00
b0500-Jun-02 &00:00   &  05 & 38 & 38.91 & +22 & 22 & 48.0  & 16982.002779 & 05&38&39.75 & +22&22&49.6 \\ % -499-06-02T00:00:00
b0500-Nov-01 &00:00   &  12 & 55 & 48.65 & -03 & 50 & 09.1  & 16975.747099 & 12&55&49.37 & -03&50&13.9 \\ % -499-11-01T00:00:00
b0499-Apr-02 &00:00   &  00 & 19 & 39.97 & +01 & 41 & 26.3  & 16969.492518 & 00&19&40.68 & +01&41&31.2 \\ % -498-04-02T00:00:00
 0500-Jan-01 &00:00   &  18 & 23 & 11.76 & -25 & 14 & 05.2  &  5608.981515 & 18&23&12.34 & -25&14&05.6 \\ % 500-01-01T00:00:00
 0500-Jun-01 &00:00   &  06 & 29 & 15.56 & +24 & 33 & 46.9  &  5605.028887 & 06&29&16.13 & +24&33&47.2 \\ % 500-06-01T00:00:00
 0500-Oct-31 &00:00   &  13 & 28 & 09.82 & -06 & 51 & 31.3  &  5601.076499 & 13&28&10.29 & -06&51&34.4 \\ % 500-10-31T00:00:00
 0501-Apr-01 &00:00   &  23 & 54 & 49.90 & -02 & 49 & 01.2  &  5597.124355 & 23&54&50.38 & -02&48&58.1 \\ % 501-04-01T00:00:00
 0501-Aug-31 &00:00   &  12 & 03 & 21.92 & -01 & 46 & 20.3  &  5593.172456 & 12&03&22.39 & -01&46&23.4 \\ % 501-08-31T00:00:00
 1600-Jan-01 &00:00   &  19 & 16 & 54.82 & -24 & 27 & 05.8  &   110.462079 & 19&16&54.99 & -24&27&05.7 \\ % 1600-01-01T00:00:00
 1600-Jun-01 &00:00   &  05 & 27 & 00.83 & +22 & 06 & 51.2  &   109.804702 & 05&27&01.00 & +22&06&51.5 \\ % 1600-06-01T00:00:00
 1600-Oct-31 &00:00   &  13 & 24 & 30.45 & -06 & 45 & 14.3  &   109.148794 & 13&24&30.59 & -06&45&15.2 \\ % 1600-10-31T00:00:00
 1601-Apr-01 &00:00   &  00 & 45 & 29.79 & +03 & 58 & 00.7  &   108.494367 & 00&45&29.92 & +03&58&01.7 \\ % 1601-04-01T00:00:00
 1601-Aug-31 &00:00   &  12 & 08 & 26.49 & -04 & 19 & 13.1  &   107.841429 & 12&08&26.63 & -04&19&14.0 \\ % 1601-08-31T00:00:00
VENUS\\ 
b2500-Jan-01 &00:00   &  14 & 37 & 07.30 & -10 & 25 & 08.5  & 58893.015756 & 14&37&08.81 & -10&25&17.6 \\ % -2499-01-01T00:00:00
b2500-Jun-02 &00:00   &  01 & 28 & 34.06 & +08 & 11 & 02.7  & 58881.693980 & 01&28&35.54 & +08&11&12.6 \\ % -2499-06-02T00:00:00
b2500-Nov-01 &00:00   &  14 & 01 & 18.60 & -13 & 03 & 41.8  & 58870.373294 & 14&01&20.11 & -13&03&51.3 \\ % -2499-11-01T00:00:00
b2499-Apr-02 &00:00   &  02 & 04 & 39.28 & +15 & 58 & 24.8  & 58859.053698 & 02&04&40.81 & +15&58&34.2 \\ % -2498-04-02T00:00:00
b0500-Jan-01 &00:00   &  15 & 15 & 08.22 & -13 & 55 & 20.5  & 16988.259557 & 15&15&08.98 & -13&55&24.4 \\ % -499-01-01T00:00:00
b0500-Jun-02 &00:00   &  02 & 40 & 55.02 & +14 & 44 & 15.5  & 16982.002779 & 02&40&55.77 & +14&44&19.8 \\ % -499-06-02T00:00:00
b0500-Nov-01 &00:00   &  15 & 13 & 42.04 & -18 & 47 & 53.8  & 16975.747099 & 15&13&42.82 & -18&47&57.7 \\ % -499-11-01T00:00:00
b0499-Apr-02 &00:00   &  03 & 09 & 40.90 & +21 & 23 & 39.0  & 16969.492518 & 03&09&41.68 & +21&23&43.0 \\ % -498-04-02T00:00:00
 0500-Jan-01 &00:00   &  20 & 18 & 48.52 & -15 & 30 & 50.9  &  5608.981515 & 20&18&49.05 & -15&30&49.8 \\ % 500-01-01T00:00:00
 0500-Jun-01 &00:00   &  02 & 10 & 53.15 & +11 & 08 & 35.0  &  5605.028887 & 02&10&53.64 & +11&08&38.0 \\ % 500-06-01T00:00:00
 0500-Oct-31 &00:00   &  14 & 33 & 13.65 & -14 & 37 & 22.9  &  5601.076499 & 14&33&14.14 & -14&37&25.8 \\ % 500-10-31T00:00:00
 0501-Apr-01 &00:00   &  03 & 02 & 54.73 & +18 & 45 & 37.5  &  5597.124355 & 03&02&55.25 & +18&45&40.1 \\ % 501-04-01T00:00:00
 0501-Aug-31 &00:00   &  09 & 10 & 56.37 & +08 & 17 & 52.4  &  5593.172456 & 09&10&56.87 & +08&17&50.6 \\ % 501-08-31T00:00:00
 1600-Jan-01 &00:00   &  17 & 09 & 26.64 & -18 & 08 & 55.7  &   110.462079 & 17&09&26.80 & -18&08&56.1 \\ % 1600-01-01T00:00:00
 1600-Jun-01 &00:00   &  02 & 28 & 52.77 & +12 & 50 & 46.3  &   109.804702 & 02&28&52.92 & +12&50&47.1 \\ % 1600-06-01T00:00:00
 1600-Oct-31 &00:00   &  14 & 54 & 13.16 & -16 & 15 & 01.5  &   109.148794 & 14&54&13.31 & -16&15&02.3 \\ % 1600-10-31T00:00:00
 1601-Apr-01 &00:00   &  03 & 18 & 37.29 & +20 & 05 & 59.9  &   108.494367 & 03&18&37.45 & +20&06&00.6 \\ % 1601-04-01T00:00:00
 1601-Aug-31 &00:00   &  07 & 48 & 52.75 & +16 & 06 & 13.4  &   107.841429 & 07&48&52.90 & +16&06&13.2 \\ % 1601-08-31T00:00:00
MOON\\ 
b2500-Jan-01 &00:00   &  23 & 43 & 54.13 & -04 & 46 & 15.6  & 58893.015756 & 23&43&55.61 & -04&46&06.1 \\ % -2499-01-01T00:00:00
b2500-Jun-02 &00:00   &  12 & 15 & 08.68 & -00 & 12 & 02.6  & 58881.693980 & 12&15&10.14 & -00&12&12.5 \\ % -2499-06-02T00:00:00
b2500-Nov-01 &00:00   &  02 & 16 & 07.77 & +16 & 35 & 31.3  & 58870.373294 & 02&16&09.31 & +16&35&40.6 \\ % -2499-11-01T00:00:00
b2499-Apr-02 &00:00   &  15 & 07 & 11.31 & -22 & 11 & 27.1  & 58859.053698 & 15&07&12.93 & -22&11&35.4 \\ % -2498-04-02T00:00:00
b0500-Jan-01 &00:00   &  02 & 04 & 52.71 & +13 & 33 & 43.2  & 16988.259557 & 02&04&53.45 & +13&33&47.8 \\ % -499-01-01T00:00:00
b0500-Jun-02 &00:00   &  14 & 51 & 07.34 & -15 & 30 & 50.9  & 16982.002779 & 14&51&08.09 & -15&30&55.1 \\ % -499-06-02T00:00:00
b0500-Nov-01 &00:00   &  05 & 08 & 19.28 & +18 & 56 & 30.3  & 16975.747099 & 05&08&20.10 & +18&56&32.5 \\ % -499-11-01T00:00:00
b0499-Apr-02 &00:00   &  17 & 59 & 21.65 & -18 & 38 & 44.9  & 16969.492518 & 17&59&22.47 & -18&38&46.1 \\ % -498-04-02T00:00:00
 0500-Jan-01 &00:00   &  05 & 29 & 38.64 & +28 & 17 & 23.7  &  5608.981515 & 05&29&39.22 & +28&17&24.9 \\ % 500-01-01T00:00:00
 0500-Jun-01 &00:00   &  19 & 35 & 39.76 & -24 & 49 & 55.2  &  5605.028887 & 19&35&40.33 & -24&49&54.6 \\ % 500-06-01T00:00:00
 0500-Oct-31 &00:00   &  08 & 38 & 03.41 & +19 & 57 & 58.9  &  5601.076499 & 08&38&03.95 & +19&57&57.5 \\ % 500-10-31T00:00:00
 0501-Apr-01 &00:00   &  22 & 23 & 01.61 & -08 & 15 & 47.6  &  5597.124355 & 22&23&02.11 & -08&15&45.1 \\ % 501-04-01T00:00:00
 0501-Aug-31 &00:00   &  11 & 25 & 37.06 & -00 & 20 & 06.1  &  5593.172456 & 11&25&37.53 & -00&20&09.0 \\ % 501-08-31T00:00:00
 1600-Jan-01 &00:00   &  07 & 04 & 50.63 & +24 & 04 & 49.5  &   110.462079 & 07&04&50.78 & +24&04&49.4 \\ % 1600-01-01T00:00:00
 1600-Jun-01 &00:00   &  19 & 54 & 11.52 & -20 & 26 & 52.9  &   109.804702 & 19&54&11.67 & -20&26&52.7 \\ % 1600-06-01T00:00:00
 1600-Oct-31 &00:00   &  10 & 11 & 21.07 & +07 & 13 & 32.6  &   109.148794 & 10&11&21.20 & +07&13&32.0 \\ % 1600-10-31T00:00:00
 1601-Apr-01 &00:00   &  22 & 46 & 17.79 & -02 & 54 & 56.8  &   108.494367 & 22&46&17.91 & -02&54&56.1 \\ % 1601-04-01T00:00:00
 1601-Aug-31 &00:00   &  12 & 52 & 06.59 & -10 & 52 & 15.0  &   107.841429 & 12&52&06.72 & -10&52&15.8 \\ % 1601-08-31T00:00:00
MARS\\ 
 1601-Jan-01 &00:00   &  19 & 26 & 55.96 & -23 & 01 & 43.2  &   108.881677 & 19&26&56.12 & -23&01&43.1 \\ % 1601-01-01T00:00:00
 1601-Jun-02 &00:00   &  02 & 59 & 06.71 & +16 & 41 & 20.6  &   108.227856 & 02&59&06.87 & +16&41&21.3 \\ % 1601-06-02T00:00:00
 1601-Nov-01 &00:00   &  09 & 48 & 48.40 & +15 & 01 & 38.7  &   107.575530 & 09&48&48.55 & +15&01&38.1 \\ % 1601-11-01T00:00:00
 1602-Apr-02 &00:00   &  10 & 25 & 38.61 & +13 & 17 & 29.7  &   106.924708 & 10&25&38.76 & +13&17&28.9 \\ % 1602-04-02T00:00:00
JUPITER\\ 
 1601-Jan-01 &00:00   &  11 & 33 & 23.43 & +04 & 18 & 09.3  &   108.881677 & 11&33&23.57 & +04&18&08.4 \\ % 1601-01-01T00:00:00
 1601-Jun-02 &00:00   &  10 & 59 & 49.50 & +07 & 52 & 18.1  &   108.227856 & 10&59&49.64 & +07&52&17.3 \\ % 1601-06-02T00:00:00
 1601-Nov-01 &00:00   &  12 & 39 & 41.26 & -03 & 02 & 14.0  &   107.575530 & 12&39&41.40 & -03&02&14.9 \\ % 1601-11-01T00:00:00
 1602-Apr-02 &00:00   &  13 & 08 & 44.21 & -05 & 35 & 33.8  &   106.924708 & 13&08&44.34 & -05&35&34.7 \\ % 1602-04-02T00:00:00
SATURN\\ 
 1750-Jan-01 &00:00   &  15 & 46 & 12.26 & -17 & 54 & 37.7  &    17.322273 & 15&46&12.36 & -17&54&38.1 \\ % 1750-01-01T00:00:00
 1750-Jun-02 &00:00   &  15 & 44 & 23.78 & -17 & 34 & 00.9  &    17.413029 & 15&44&23.88 & -17&34&01.3 \\ % 1750-06-02T00:00:00
 1750-Nov-01 &00:00   &  15 & 59 & 18.99 & -18 & 50 & 41.0  &    17.503706 & 15&59&19.09 & -18&50&41.4 \\ % 1750-11-01T00:00:00
 1751-Apr-02 &00:00   &  16 & 49 & 11.73 & -20 & 39 & 30.2  &    17.594270 & 16&49&11.82 & -20&39&30.5 \\ % 1751-04-02T00:00:00
URANUS\\  
 1600-Jan-01 &00:00   &  01 & 41 & 51.35 & +10 & 01 & 26.1  &   110.462079 & 01&41&51.49 & +10&01&26.9 \\ % 1600-01-01T00:00:00
 1600-Jun-01 &00:00   &  02 & 05 & 21.56 & +12 & 13 & 45.5  &   109.804702 & 02&05&21.71 & +12&13&46.3 \\ % 1600-06-01T00:00:00
 1600-Oct-31 &00:00   &  02 & 04 & 09.96 & +12 & 05 & 40.0  &   109.148794 & 02&04&10.11 & +12&05&40.8 \\ % 1600-10-31T00:00:00
 1601-Apr-01 &00:00   &  02 & 07 & 08.61 & +12 & 25 & 31.4  &   108.494367 & 02&07&08.76 & +12&25&32.3 \\ % 1601-04-01T00:00:00
 1601-Aug-31 &00:00   &  02 & 28 & 09.10 & +14 & 11 & 23.3  &   107.841429 & 02&28&09.25 & +14&11&24.1 \\ % 1601-08-31T00:00:00
NEPTUNE\\ 
 1801-Jan-01 &00:00   &  15 & 07 & 04.41 & -15 & 44 & 47.2  &    18.261920 & 15&07&04.49 & -15&44&47.6 \\ % 1801-01-01T00:00:00
 1801-Jun-02 &00:00   &  15 & 01 & 57.08 & -15 & 18 & 05.9  &    18.107889 & 15&01&57.15 & -15&18&06.3 \\ % 1801-06-02T00:00:00
 1801-Nov-01 &00:00   &  15 & 07 & 02.35 & -15 & 46 & 26.5  &    17.952783 & 15&07&02.43 & -15&46&26.9 \\ % 1801-11-01T00:00:00
 1802-Apr-02 &00:00   &  15 & 17 & 01.46 & -16 & 20 & 15.8  &    17.797431 & 15&17&01.54 & -16&20&16.2 \\ % 1802-04-02T00:00:00
IO(JI)\\   
 1601-Jan-01 &00:00   &  11 & 33 & 30.35 & +04 & 17 & 19.8  &   108.881677 & 11&33&30.50 & +04&17&18.9 \\ % 1601-01-01T00:00:00
 1601-Jun-02 &00:00   &  10 & 59 & 55.42 & +07 & 51 & 35.9  &   108.227856 & 10&59&55.57 & +07&51&35.0 \\ % 1601-06-02T00:00:00
 1601-Nov-01 &00:00   &  12 & 39 & 41.55 & -03 & 02 & 20.3  &   107.575530 & 12&39&41.71 & -03&02&21.4 \\ % 1601-11-01T00:00:00
 1602-Apr-02 &00:00   &  13 & 08 & 39.95 & -05 & 35 & 12.3  &   106.924708 & 13&08&40.11 & -05&35&13.4 \\ % 1602-04-02T00:00:00
EUROPA(JII)\\  
 1601-Jan-01 &00:00   &  11 & 33 & 16.18 & +04 & 19 & 07.0  &   108.881677 & 11&33&16.31 & +04&19&06.2 \\ % 1601-01-01T00:00:00
 1601-Jun-02 &00:00   &  10 & 59 & 53.24 & +07 & 51 & 59.3  &   108.227856 & 10&59&53.38 & +07&51&58.6 \\ % 1601-06-02T00:00:00
 1601-Nov-01 &00:00   &  12 & 39 & 49.02 & -03 & 03 & 11.0  &   107.575530 & 12&39&49.17 & -03&03&11.8 \\ % 1601-11-01T00:00:00
 1602-Apr-02 &00:00   &  13 & 08 & 40.14 & -05 & 35 & 20.5  &   106.924708 & 13&08&40.29 & -05&35&21.4 \\ % 1602-04-02T00:00:00
GANYMEDE(JIII)\\  
 1601-Jan-01 &00:00   &  11 & 33 & 29.32 & +04 & 17 & 17.3  &   108.881677 & 11&33&29.46 & +04&17&16.4 \\ % 1601-01-01T00:00:00
 1601-Jun-02 &00:00   &  11 & 00 & 04.62 & +07 & 50 & 38.1  &   108.227856 & 11&00&04.77 & +07&50&37.3 \\ % 1601-06-02T00:00:00
 1601-Nov-01 &00:00   &  12 & 39 & 41.87 & -03 & 02 & 07.5  &   107.575530 & 12&39&42.01 & -03&02&08.4 \\ % 1601-11-01T00:00:00
 1602-Apr-02 &00:00   &  13 & 08 & 24.32 & -05 & 33 & 17.9  &   106.924708 & 13&08&24.46 & -05&33&18.9 \\ % 1602-04-02T00:00:00
CALLISTO(JIV)\\  
 1601-Jan-01 &00:00   &  11 & 33 & 15.60 & +04 & 19 & 21.4  &   108.881677 & 11&33&15.74 & +04&19&20.5 \\ % 1601-01-01T00:00:00
 1601-Jun-02 &00:00   &  10 & 59 & 22.96 & +07 & 55 & 21.9  &   108.227856 & 10&59&23.10 & +07&55&21.0 \\ % 1601-06-02T00:00:00
 1601-Nov-01 &00:00   &  12 & 39 & 17.27 & -02 & 59 & 27.3  &   107.575530 & 12&39&17.41 & -02&59&28.3 \\ % 1601-11-01T00:00:00
 1602-Apr-02 &00:00   &  13 & 08 & 10.01 & -05 & 32 & 01.4  &   106.924708 & 13&08&10.14 & -05&32&02.4 \\ % 1602-04-02T00:00:00
\end{longtable}

\normalsize




%%% Local Variables: 
%%% mode: latex
%%% TeX-PDF-mode: t
%%% TeX-master: "guide"
%%% End: 

