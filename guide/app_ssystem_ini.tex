%% Part of Stellarium User Guide
%% Status:
%% 2016-05-16 GZ new file. This file and its handling is so complicated, it deserves proper docs.
%% 2017-06-18 GZ for v0.16.0 we have split the config file. Adapted docs.
%% TODO: document the more obscure entries. Add links to the respective chapters.

%\chapterimage{chapter-t1-bg} % Chapter heading image

\section{Solar System Configuration Files}
\label{sec:ssystem.ini}

The files\footnote{Before v0.16, there was a single file, \file{ssystem.ini}, with slightly different rules.}
 \file{ssystem\_major.ini} and  \file{ssystem\_minor.ini}  
(in the \file{data/} subdirectory of the program directory, 
contain orbital and rotational data from which Stellarium configures the Solar
System objects. 

The minor bodies can be modified by placing a privately modified 
copy of \file{ssystem\_minor.ini} into your own
\file{data/} directory (see Chapter~\ref{sec:FilesAndDirectories}). 
You can edit the file either manually or with the Solar System Editor plugin (see
Section~\ref{sec:plugins:SolarSystemEditor}). The private user file is automatically created by the Solar System Editor plugin.


%Deprecated parameters are marked by gray background. 
%Possible new parameters are marked by yellow background.
%% GZ This is no longer true. Reactivate somehow? 

Each object's data are described in its own section which is typically
named after the object name. Some section names (e.g.\ those using
diacriticals, spaces or other problematic characters) appear a bit mangled.

We list here examples for the major planets, larger moons with special
coordinate functions, minor moons with generic orbital elements, minor
planets and comets with elliptical and parabolic orbit elements.

All elements are stored in alphabetic order in the files, however it
makes more sense to present the elements in another sequence which
better reflects the meaning. The actual order of the objects in the
files, and order of entries inside an object section, is irrelevant,
but in case of duplicate entries the later entry is used.

\subsection{File ssystem\_major.ini}
\label{sec:ssystem.ini:major}


\subsubsection{Planet section}
\label{sec:ssystem.ini:Planet}

Example:
\begin{configfile}
[jupiter]
name=Jupiter
type=planet
coord_func=jupiter_special
orbit_Period=4331.87

atmosphere=1
albedo=0.51
color=1., 0.983, 0.934
absolute_magnitude=-9.40

radius=71492
oblateness=0.064874
tex_map=jupiter.png #texture courtesy of Bj\xf6rn J\xf3nsson

# deprecated rotational elements
rot_equator_ascending_node=-22.203
rot_obliquity=2.222461
rot_periode=9.92491
rot_rotation_offset=-1 # use JupiterGRS patch

# WGCCRE rotational elements
rot_pole_ra=268.056595
rot_pole_ra1=-0.006499
rot_pole_de=64.495303
rot_pole_de1=0.002413
rot_pole_w0=284.95
rot_pole_w1=870.5360000

\end{configfile}


where 
\begin{description}
\item[name] English name of the planet. May appear translated. 
\item[type] Mandatory for planets: \value{planet}
\item[parent]=Sun. The body which this object is running around. Default: \texttt{Sun}
\item[coord\_func] The planet positions are all computed with a dedicated function (VSOP87 or DE43x). 
\item[orbit\_Period] number of (earth) days for how
  long the orbit should be made visible. Typically Stellarium shows
  one orbit line. The orbit slowly drifts, however.

\item[atmosphere] (0 or 1) flag to indicate whether observer locations should enable atmosphere drawing.

\item[radius] Equator radius, km.
\item[oblateness] Flattening of the polar diameter. ($1-r_{pole}/r_{eq}$)
\item[albedo] total albedo (reflectivity) of the planet. Used for ``generic''
  magnitude computation, but the major planets usually have dedicated
  magnitude formulas, so the value is not evaluated unless you are
  observing from a location outside Earth.
\item[color] Used to colorize halo. At least one of the components should be 1.
\item[tex\_map] File name of texture map in \file{textures} folder.
\item[halo] Should be true to draw a halo (simple light disk) when object too small to draw a sphere. Default: \texttt{true}
\end{description}

\paragraph{Elements for the Physical Ephemeris}

Where known, Stellarium can properly model axis rotation with the
\texttt{rot\_\ldots} entries in the element sets.  Versions prior to
0.21 had rotational elements defined with respect to the equator of
the parent object. While writing this documentation the origin and
accuracy of this model is unfortunately untraceable.  However, the IAU
Working Group on Cartographic Coordinates and Rotational Elements
(WGCCRE) and modern sources like \citet{ESAA:2013} use rotational
elements with respect to the ICRF (see section
\ref{sec:Concepts:Equatorial}). Therefore, starting with version 0.21,\newFeature{0.21}
Stellarium uses these more modern data and can now also show axes with
precessional or other motions. Data for some planet moons are
unfortunately still not available in this WGCCRE format, and therefore
the old format can still be used, but should not be used for new
entries.

\begin{description}
\item[rot\_equator\_ascending\_node] \emph{deprecated}
\item[rot\_obliquity] \emph{deprecated}
\item[rot\_rotation\_offset] \emph{deprecated} longitude of prime
  meridian. For Planets, this used to be counted from the ascending
  node of the equator over the ecliptical plane at J2000.0. For moons,
  it is the longitude of the prime meridian counted from the ascending
  node of the moon's equator over the planet equator.  A special value
  here indicates special treatment for the Great Red Spot.
\item[rot\_periode] \emph{deprecated} Duration of one sidereal rotation, in earth hours.

\item[rot\_pole\_ra]  constant of axis right ascension in ICRF, degrees
\item[rot\_pole\_ra1] change per century of axis right ascension in ICRF, degrees
\item[rot\_pole\_de] constant of axis declination in ICRF, degrees
\item[rot\_pole\_de1] change per century of axis declination in ICRF, degrees
\item[rot\_pole\_w0] longitude of the prime meridian counted from the
  ascending node of the objects's equator through the ICRF equator, degrees
\item[rot\_pole\_w1] change per day of axis rotation, degrees
\end{description}

\noindent In many cases, the attitude/rotation formulae for the North (or
``positive'') pole $(\alpha_0, \delta_0)$ are given like
\begin{eqnarray}
  \label{eq:PlanetOrientation}
  \alpha_0 &=& \mathtt{rot\_pole\_ra} + T * \mathtt{rot\_pole\_ra1}\\
  \delta_0 &=& \mathtt{rot\_pole\_de} + T * \mathtt{rot\_pole\_de1}\\
  W        &=& \mathtt{rot\_pole\_w0} + d * \mathtt{rot\_pole\_w1}
\end{eqnarray}

\noindent These cases are fully covered by the data in
\file{ssystem\_major.ini}. Corrections for solar system objects given
by \citet{ESAA:2013, WGCCRE2009, WGCCRE2009:corr, WGCCRE2015,
  WGCCRE2015:corr} have been implemented in the program.


\subsubsection{Moon section}
\label{sec:ssystem.ini:Moon}

All planet moons are defined only in \file{ssystem\_major.ini}.

Moons are special in that they orbit another planet. Therefore, the
rotational elements used to be specified relative to the equatorial
plane of the parent planet, and \texttt{orbit\_SemiMajorAxis} are
specified in kilometers.  However, \newFeature{0.21} as for the
planets (see section \ref{sec:ssystem.ini:Planet} above), current
\indexterm{IAU} reference material gives axis orientation with right
ascension and declination values for the pole in ICRF coordinates,
with some of them in motion. So again, if one of the
\texttt{rot\_pole\_...} values exist, we assume the current
standard, but keep the old elements available until we find new data.
For more complicated motion, again some special functions
are applied.

%% TODO: CHANGE THIS WHEN AXES HAVE BEEN REMADE!

\begin{configfile}
[amalthea]
name=Amalthea
type=moon
parent=Jupiter
iau_moon_number=JV

orbit_AscendingNode=141.5521520794674
orbit_Eccentricity=0.006175744402949701
orbit_Epoch=2454619.50000
orbit_Inclination=0.3864576103404582
orbit_LongOfPericenter=245.4222355150120000
orbit_MeanLongitude=224.7924893552550000
orbit_Period=0.5016370462116355
orbit_SemiMajorAxis=181994.8658358799

absolute_magnitude=7.4
albedo=0.09
color=1., 0.627, 0.492
radius=83.5
tex_map=amalthea.png
model=j5amalthea_MLfix.obj


rot_equator_ascending_node=213.7
rot_obliquity=15.5
# rot_periode=12.039289109079252
rot_rotation_offset=235.50
rot_pole_ra=268.05
rot_pole_ra1=-0.009
rot_pole_de=64.49
rot_pole_de1=0.003
rot_pole_w0=231.67
rot_pole_w1=722.6314560
\end{configfile}

where
\begin{description}
  \item[name] English name of planet moon. No number, just the name. May be translated.
  \item[type] \texttt{moon}

  \item[parent] English name of planet or parent body.
  \item[iau\_moon\_number] a short label (string) consisting of the planet's initial and the moon's Roman number in order of discovery.\newFeature{0.16.1}

  \item[coord\_func] Must be \texttt{kepler\_orbit} (default, can be left away) for moons with orbital elements given, or \texttt{<name>\_special} for 
  \item[orbit\_AscendingNode] $\Omega$
  \item[orbit\_Eccentricity] $e$
  \item[orbit\_Epoch] 
  \item[orbit\_Inclination] $i$ [degrees]
  \item[orbit\_LongOfPericenter] 
  \item[orbit\_MeanLongitude]
  \item[orbit\_Period] [days]
  \item[orbit\_SemiMajorAxis] $a$ [km]

  \item[radius] Equator radius, km.
  \item[oblateness] Flattening of the polar
    diameter. ($1-r_{pole}/r_{eq}$) Bodies with non-ellipsoid shape,
    e.g. tri-axial geometry, cannot be modelled with this simple
    approach, but a 3D solid model can be shown, see below.
    
  \item[albedo] total reflectivity [0..1]
  \item[color] for drawing halo (default: 1,1,1)
  \item[halo] [=true|false] to draw a simple diffuse dot when zoomed out. Default: \texttt{true}
  \item[tex\_map] name of spherical texture map. Many moons have been mapped by visiting spacecraft! For many other moons, Stellarium applies an inverted Lunar texture image.
  \item[model] \newFeature{0.16.0}(optional) name of a 3D model for a non-spherical body in the \file{model} subdirectory of the program directory. 

  \item[rot\_equator\_ascending\_node] \emph{deprecated}
  \item[rot\_obliquity] \emph{deprecated}
  \item[rot\_rotation\_offset] \emph{deprecated} longitude of prime meridian at J2000.0. 
  \item[rot\_periode]  \emph{deprecated}  Duration of one sidereal rotation, in earth hours. For moons in bound rotation 
       (which always show one face towards their parent planet), it is best to omit this value: it defaults to \texttt{orbit\_Period * 24}. 

  \item[rot\_pole\_ra]  constant of axis right ascension in ICRF, degrees
  \item[rot\_pole\_ra1] change per century of axis right ascension in ICRF, degrees
  \item[rot\_pole\_de] constant of axis declination in ICRF, degrees
  \item[rot\_pole\_de1] change per century of axis declination in ICRF, degrees
  \item[rot\_pole\_w0] longitude of the prime meridian counted from the
    ascending node of the objects's equator through the ICRF equator, degrees
  \item[rot\_pole\_w1] change per day of axis rotation, degrees  
\end{description}



\subsubsection{Observers}
\label{sec:ssystem.ini:Observers}

Stellarium is great for excursions to the surface of any object with
known orbital elements.  Configuring a viewpoint away from a planet
requires a special kind of location.

\paragraph{Solar System Observer}
\label{sec:ssystem.ini:SolarSystemObserver}

The Solar System Observer has
been provided in earlier versions as a neutral view location high above the North pole of
the Solar System. Meanwhile \newFeature{0.19.2} the location can be changed by keyboard interaction. See section \ref{sec:gui:location:observers}.
\begin{configfile}
[solar_system_observer]
name=Solar System Observer
parent=Sun
halo=false
hidden=true
orbit_SemiMajorAxis=30
type=observer
\end{configfile}
Note that it is \texttt{hidden} and has no \texttt{halo}. 
The observer is attached to the Sun, therefore distance \texttt{orbit\_semiMajorAxis} is given in AU.

\paragraph{Planet Observers}
\label{sec:ssystem.ini:PlanetObserver}

There are other ``observer'' locations for all planets \newFeature{0.19.2} which have moons: Earth, Mars, Jupiter, Saturn, Uranus, and Neptune.
For example:
\begin{configfile}
[earth_observer]
name=Earth Observer
parent=Earth
halo=false
hidden=true
orbit_SemiMajorAxis=149600000
type=observer
\end{configfile}

The distance \texttt{orbit\_semiMajorAxis} has been preconfigured \newFeature{23.1} with a
distance that gives a good view of the planet with its moons. 
Like for moons, the units are \km\ from the planet. 

\subsection{File ssystem\_minor.ini}
\label{sec:ssystem.ini:minor}

Orbital elements for minor bodies are always given for a particular
\indexterm{equinox} like J2000.0 (which indicates the coordinate
reference system) and \indexterm{epoch} (date). The other planets pull
on the objects and change their orbits, most usually leading to the
objects appearing too early or too late in their orbits, but also
changing the other orbital elements. In addition, outgassing events of
comets can act like jet propulsion and change orbits in unpredictable
ways.  It is heavily recommended to update the orbital elements on a
regular basis ($2\times$/year?), or at least before you go out and are
actually observing minor bodies.  Use the Solar System Editor plugin
for this task (section~\ref{sec:plugins:SolarSystemEditor}).

You may find element sets for different equinoxes, like B1950.0. These
have to be converted to equinox J2000.0 data before being useful in
Stellarium.

\begin{description}
\item[coord\_func] is \texttt{kepler\_orbit} by
  default\newFeature{0.20}. For compatibility and interoperability
  with older versions we will keep the old default of
  \texttt{comet\_orbit} for some longer time.\footnote{Previous
    editions used the name \texttt{comet\_orbit}. Now the
    \texttt{coord\_func} line can be left away, defaulting to
    \texttt{kepler\_orbit}.}
\item[parent] defaults to \texttt{Sun} and can be omitted.

\item[orbit\_Epoch] JDE when these elements are valid. Defaults to
  2451545.0 = the J2000.0 standard epoch.

\item[orbit\_good] can be given in days to limit computation of the
  object to the time range
  $\mathtt{epoch}\pm\mathtt{orbit\_good}$. If\newFeature{0.21.3}
  specified as \texttt{0} there is no check for out-of-range dates. If
  specified as \texttt{-1}, half the orbital period is used. The main
  purpose of this parameter is to avoid an element clash or the
  inadvertent use of outdated comet orbit elements when a periodic
  comet reappears.

\item[orbit\_ArgOfPericenter] $\omega$ [degrees]
\item[orbit\_AscendingNode] $\Omega$  [degrees]
\item[orbit\_Eccentricity] $e=0$ circular, $0<e<1$ elliptic, $e=1$ parabolic, $e>1$ hyperbolic
\item[orbit\_Inclination] $i$ [degrees] inclination against J2000 ecliptic
\item[orbit\_SemiMajorAxis] $a$ [AU]
\end{description}
The other parameters are like those for the major planets. 

\subsubsection{Minor Planet section}
\label{sec:ssystem.ini:MinorPlanet}

\begin{configfile}
[4vesta]
type=asteroid  
minor_planet_number=4
name=Vesta

coord_func=comet_orbit
parent=Sun
orbit_Epoch=2457000.5
orbit_MeanAnomaly=20.86389
orbit_MeanMotion=0.27154465

orbit_ArgOfPericenter=151.19843
orbit_AscendingNode=103.85141
orbit_Eccentricity=0.0887401
orbit_Inclination=7.14043
orbit_SemiMajorAxis=2.3617933
orbit_good=1325.46

color=1., 1., 1.
halo=true
oblateness=0.0

albedo=0.423
radius=280
absolute_magnitude=3.2
slope_parameter=0.32
tex_map=vesta.png
model=4vesta_21_MLfix.obj
\end{configfile}

\begin{description}
\item[type] can be \texttt{asteroid, dwarf planet, cubewano,
  plutino, scattered disc object, Oort cloud object}. With the
exception of Pluto (which is included in \file{ssystem\_major.ini} and cannot be changed), 
all positions for minor bodies are computed with the orbiting elements given in this way. 
\end{description}

Minor planets further specify
\begin{description}
\item[orbit\_Epoch], JDE when these elements are valid. Defaults to J2000.0, i.e., 2451545.0. 
\item[orbit\_MeanAnomaly] $M$ mean anomaly, degrees. 
\item[orbit\_MeanMotion] $n$ mean motion [degrees/day].
\end{description}

Visual magnitude is modelled from
\begin{description}
\item[absolute\_magnitude] $H$
\item[slope\_parameter] $G$.
\end{description}

Elements for rotational axis may be given just like for planets when
they are known. It is recommended to use the modern specification
(elements \texttt{rot\_pole\_...}).

\begin{description}
\item[model]
A few asteroids have been visited by spacecraft, \newFeature{0.16.0}
and for many other asteroids visual observations of stellar occultations 
by asteroids and light curve measurements have enabled researchers to derive 
3D shape models of asteroids. If a model is available in the \file{models} 
subdirectory of the program directory, this can be configured with a \texttt{model} entry.
\end{description}


\subsubsection{Comet section}
\label{sec:ssystem.ini:Comet}

Comets are tiny, and their outgassing and close approaches to the
major planets cause fast changes in their orbital elements, so that
each apparition should be specified with a dedicated section in
\file{ssystem\_minor.ini}.

\paragraph{Naming schemes}
Over just the past decades, the naming of comets has changed considerably. 
Starting in the 19th century, comets are usually named after their discoverer(s). 
In addition, they are given a catalog number. 
Traditionally, comets observations published by scientific organisations were sorted and labelled with Roman numerals after date of perihelion. 
When scientific communication became faster, comet discoveries were circulated by telegraphs, and a preliminary immediate labelling by sequence of discovery was used. 
After the year was over, the comet received again the Roman number indicating the position in the sequence of perihelon dates in that year. 

In 1994, a new scheme has been introduced\footnote{\url{https://minorplanetcenter.net/iau/lists/CometResolution.html}}: 
When a comets is discovered, the discovery date governs a unique label: 
The 12 months of the year are split into half-months, for which 24 of the 26 letters of the alphabet are used (I and Z are omitted). 
After the letter, discoveries are enumerated in sequence. 
A letter before this number indicates what kind of comet the element set describes: \textbf{C} indicates a regular comet, 
\textbf{P} a periodic (recurring) comet, \textbf{D} is defunct (lost). \textbf{X} is one for which no reliable orbital elements can be given. 
\textbf{I} labels an interstellar object (an object on an orbit which seems not to be permanently attached to the sun).
After a return, periodic comets receive a serial number before the P.

This new scheme is also applied for old comets. However, for the sake of compatibility with older literature, 
Stellarium can use several optional entries in addition to the name, for example:\newFeature{23.2}
\begin{description}
\item[name] 1P/Halley (First periodic comet, named after \name[Edmund]{Halley}, the astronomer who identified its recurrent nature)
\item[iau\_designation] 1982 U1 (First comet discovered in 2nd half of October 1982)
\item[discovery\_code] 1982i (9th comet discovered in 1982)
\item[perihelion\_code] 1986 III (Third comet to go through perihelion in 1986)
\end{description}

There is unfortunately divergent information about use of the new IAU discovery code for periodical comets like 1P/Halley. 
Some lists provide such codes for previous observations, while others use the modern code for Halley's own observation (P/1682 Q1) for every apparition, 
in which case identification of the proper apparition from just the name becomes impossible. 
IAU simply uses a periodic number with P and optional name (for comet Halley, ``1P/Halley''. Almost all periodic comets have names) to identify such a recurring object. 
While this can be used for predicted future passages through the inner Solar system, 
there cannot be a modern ``discovery code'' based label when the comet was at least in theory followed through aphelion.
But also this format, using only periodic number, prevents disambiguation between apparitions. 
To make life simpler for us, we prefer to use a name entry like ``1P/Halley (1986)'' with year of perihel passage in brackets. 
A numbered comet may come without a discoverer name. In this case it is just shown with perihel year like ``396P (2019)''.
The ini file should include a periodic number in front of the P in the name, however some data files are incomplete. 
Unnumbered comets follow the new IAU code, e.g., ``C/1996 B2 (Hyakutake)''. 
You can also provided separate \texttt{name} and \texttt{iau\_designation} entries, in which case they will be combined. 


Most other elements are similar to minor planets. But note the specification of
\begin{description}
\item[orbit\_PericenterDistance] $q$ [AU]
\item[orbit\_TimeAtPericenter] $T$  JDE of closest approach to the Sun
\item[orbit\_Epoch] JDE when these elements are valid (optional).
\end{description}
which is more typical for comets which may have no orbital period when
they are on parabolic or hyperbolic orbits. Frequently orbital
elements are given for a date very close to perihelion so that the
perihel date $T$ is used wherever 
\texttt{orbit\_Epoch} is not given. However, this assumption does not always hold,
so we\newFeature{0.21.3} recommend to state \texttt{orbit\_Epoch} explicitly.

\begin{description}
\item[ref] Reference, e.g.\ some MPC Circular (optional). \newFeature{0.21.3} May be used in future versions.
\end{description}

Comet brightness is evaluated from
\begin{equation}
  \label{eq:comet_magnitudes}
  \mathrm{mag}=\mathtt{absolute\_magnitude}+5\cdot\log{\mathrm{distance}} + 2.5\cdot\mathtt{slope\_parameter}\cdot\log(\mathrm{CometSunDistance})
\end{equation}

The term \texttt{slope\_parameter} may be a misnomer in case of
comets. From the literature \citep{AstronomicalAlgorithms:1998} (equation 33.13) we find
\begin{equation}
  \label{eq:comet_magnitudes_Meeus}
  \mathrm{mag}=g+5\log\Delta + \kappa\log r
\end{equation}
from which $\kappa=2\cdot\mathtt{slope\_parameter}$. In any case, $\kappa$ is typically [5\ldots15] and specific for each comet.

You can finetune the appearance of the comet, both when zooming in all the way to the core, or how the tails are represented. 
\begin{description}
\item[albedo] (default: 0.075) is used to set the brightness for rendering the body, if you are close enough. 
\item[model] Some comets have been visited by spacecraft so that shape models of their cores 
	may be available and can be configured with this entry.\newFeature{0.16.0}
\item[outgas\_intensity] (default:0.1) Intensity of a pseudo-outgas effect on the core, based on an inverse exponential Lambert shading, with the light at the viewing position.
\item[outgas\_falloff] (default:0.1)  Exponent for falloff of outgas effect, should probably stay $< 1$.
\item[dust\_brightnessfactor] (default: 1.5) brightness of dust tail relative to the gas tail.
\item[dust\_lengthfactor] (default: 0.4) length of dust tail relative to the gas tail.
\item[dust\_widthfactor] (default: 1.5) width of dust tail relative to the gas tail.
\end{description}
Some historical depictions show huge comets. If painted correctly, you can try to recreate such paintings.

A large number of elements for historical comets is provided in the
file \file{ssystem\_1000comets.ini} in the installation directory. You
can copy\&paste what you need into your
\file{ssystem\_minor.ini} or add all with the Solar System Editor plugin (section~\ref{sec:plugins:SolarSystemEditor}). 


With a clever combination of elements and \texttt{orbit\_good}
entries, \newFeature{0.21.3} it is possible to specify several sets of
orbital elements for different epochs of one apparition.

\paragraph{Periodic Comet}
\label{sec:ssystem.ini:Comet:Periodic}

\begin{configfile}
[1phalley]
type=comet  
name=1P/Halley
iau_designation=1982 U1
discovery_code=1982i
perihelion_code=1986 III

orbit_ArgOfPericenter=111.7154
orbit_AscendingNode=58.8583
orbit_Eccentricity=0.968004
orbit_Inclination=162.2453
orbit_PericenterDistance=0.57136
orbit_TimeAtPericenter=2446463.12979167
orbit_good=780

color=1.0, 1.0, 1.0
dust_brightnessfactor=1.5
dust_lengthfactor=0.4
dust_widthfactor=1.5

albedo=0.1
radius=5
absolute_magnitude=5.5
slope_parameter=3.2
model=1682q1halley_MLfix.obj [optional]
\end{configfile}

Note a rather short duration of
\texttt{orbit\_good}, which means the comet is only displayed 780 days
before and after perihelion. (Actually, before and after
\texttt{orbit\_Epoch}, but this is not given explicitly, so it
defaults to \texttt{orbit\_TimeAtPericenter}.)


\paragraph{Parabolic/Hyperbolic Comet}
\label{sec:ssystem.ini:Comet:Parabolic}

\begin{configfile}
[c2013us10%28catalina%29]
type=comet
name=Catalina
iau_designation=C/2013 US10

orbit_ArgOfPericenter=340.3533
orbit_AscendingNode=186.141
orbit_Eccentricity=1.000372
orbit_Inclination=148.8766
orbit_PericenterDistance=0.822958
orbit_TimeAtPericenter=2457342.20748843
orbit_good=1000

...

absolute_magnitude=4.4
slope_parameter=4
\end{configfile}

This has basically the same format. Note that eccentricity is larger than 1,
this means the comet is following a slightly hyperbolic
orbit. Stellarium shows data for this comet for almost 3 years
(\texttt{orbit\_good=1000} days) from the epoch (defaults to pericenter time).




%%% Local Variables: 
%%% mode: latex
%%% TeX-PDF-mode: t
%%% TeX-master: "guide"
%%% End: 

