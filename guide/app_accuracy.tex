%% Stellarium user guide.
%% State: 2015-12 wiki->Guide. Was called Precision, but this is wrong.
%% 2016-04 GZ Typofixes, made proper chapter from this. 
%% TODO: (GZ) I am not sure about the 1-arcsecond accuracy of VSOP! More should be written with due diligence.
%% 2016-07-28 GZ Extended chapter with more than just VSOP notes.
%% 2017-12-10 GZ Extended with more warnings for "ignorant historians".
%\chapterimage{chapter-t1-bg} % Chapter heading image


\chapter{Accuracy}
\label{ch:Accuracy}

Stellarium originally was developed to present a beautiful simulation
of the night sky, mostly to understand what is visible in the sky when
you leave your house, i.e., for present times. To save computation
time, some concessions were made in astronomical accuracy by using
simplified models which seemed acceptable at that time. However, many
users started to overstress Stellarium's capabilities to simulate the
historical sky of many centuries in the past, and found --- or even worse: 
did not find, but simply published --- the resulting inconsistencies and errors.  
Unfortunately, celestial motions are indeed more
complicated than a simple clockwork, and the process of retrofitting
detailed and accurate models which started around v0.11.0 is not
completed yet. Therefore, when using Stellarium for scientific work
like eclipse simulation to illustrate records found in Cuneiform
tablets, always also use some other reference to compare, and of 
course specify the program version you used for your work. (The next version 
may be more accurate or at least provide different results!) You can of
course contact us if you are willing and able to help improving
Stellarium's accuracy!

\section{Date Range}
\label{sec:Accuracy:DateRange}

The calendar dates are limited to a very wide interval of years, -100.000\ldots100.000. 
This is mostly useful to visualize the effect of stellar proper motions or the wobbling 
motion of Earth's axis known as precession. 
If you observe the other planets, this is way beyond what current models are able to show! 
Observe for example the Moon: you will find that it seems to run 
on a polar orbit around +78.000 and even retrograde on a highly elongated orbit a few millennia after that. 
This is obvious nonsense, caused by extrapolating mathematical models in inappropriate ways. 

\section{Stellar Proper Motion}
\label{sec:Accuracy:ProperMotion}

Even though you can follow stellar proper motion for tens of thousands of years, 
please note that the computation only takes the linear components $\Delta\alpha$, $\Delta\delta$ into account. 
For times far from today, a true 3D computation would be required, 
so that also changes in distance (and thus, brighness) could be simulated. 
Still, the simulation here should give a good impression which close and/or fast stars 
will move over time, and thus how constellations become deformed.

Note that due to the way the stars are stored in catalogs, zooming in on fast stars like Arcturus 
may even cease to display such stars when their original area in the sky is no longer in view.

Also, binary stars seem to break apart because Stellarium does not store them as systems 
revolving about their combined center of gravity.

These known deficits were also identified in a recent study
\citep{deLorenzis:2018}, which emphasized errors with Arcturus, Dubhe,
Sirius, Toliman and Vega which already deviate noticably from their
correct locations around 2500BC.  If you need accurate stellar
positions for earlier prehistory from a free and open-source program,
it seems only \program{Cartes du Ciel} currently can be recommended as
on a par with the best available solutions.


\section{Planetary Positions}
\label{sec:Accuracy:Planets}

Stellarium uses the VSOP87 theory~\citep{1988A&A...202..309B} %\footnote{\url{http://vizier.cfa.harvard.edu/viz-bin/ftp-index?/ftp/cats/VI/81}}
to calculate the positions of the planets over time.
This is an analytical ephemeris modeled to match the numerical
integration run DE200 from NASA JPL, which by itself gives positions for 1850\ldots2050. 
Its use is recommended for the years -4000\ldots+8000. You can observe the sun leaving the ``ecliptic
of date'' and running on the ``ecliptic J2000'' outside this date
range. This is obviously a mathematical trick to keep
continuity. Still, positions may be somewhat useful outside this
range, but don't expect anything reliable 50,000 years in the past! 

The optionally usable JPL DE431 delivers planet positions strictly for
-13000\ldots+17000 only, and nothing outside, but likely with higher accuracy. 
Outside of this range, positions from VSOP87 will be shown again.

As far as Stellarium is concerned, the user should bear in mind the
limits of the VSOP87 and other models (Table~\ref{tab:Accuracy:Planets}).
Accuracy values for VSOP87 are heliocentric as given by \citet{1988A&A...202..309B}.

\begin{table}[tb]
\begin{tabularx}{\textwidth}{X|l|X}
\toprule
\emph{Object(s)} & \emph{Method} & \emph{Notes}\tabularnewline
\midrule
Mercury, Venus, Earth-Moon barycenter, Mars & VSOP87 & Accuracy is 1 arc-second from 2000 B.C. -- 6000 A.D. \\%\midrule
Jupiter, Saturn                             & VSOP87 & Accuracy is 1 arc-second from 0 A.D. -- 4000 A.D.    \\%\midrule
Uranus, Neptune                             & VSOP87 & Accuracy is 1 arc-second from 4000 B.C. -- 8000 A.D. \\%\midrule
Pluto                                       & ?      & Pluto's position is valid only from 1885 A.D. -- 2099 A.D.\\%\midrule
Earth's Moon                                & ELP2000-82B & Fit against JPL DE200, which provides date range 1850--2050. 
                                                            Positions should be usable for some (unspecified) longer interval. \\%\midrule
Galilean satellites                         & L2     & Valid from 500 A.D. -- 3500 A.D.\\ 
\bottomrule
\end{tabularx}
\caption{Orbit-centric accuracy of used algorithms}
\label{tab:Accuracy:Planets}
\end{table}

\noindent Note that Stellarium models light-time correction (if enabled), but does not yet model aberration of light. 
Therefore seasons' beginnings are off by a few minutes.

\section{Minor Bodies}
\label{sec:Accuracy:MinorBodies}

Positions for the Minor Bodies (Dwarf Planets, Asteroids, Comets) are
computed with standard algorithms found in astronomical text
books. The generally used method of orbital elements allows to compute
the positions of the respective object on an undisturbed Kepler orbit
around the sun. However, gravitational, and in the case of comets,
non-gravitational (outgassing) disturbances slowly change these
orbital elements. Therefore an \emph{epoch} is given for such
elements, and computation of positions for times far from this epoch
will lead to positional errors. Therefore, when searching for
asteroids or comets, always update your orbital elements (See section~\ref{sec:plugins:SolarSystemEditor}), 
and use elements with an epoch as close to your time of observation as
possible! Stellarium does not simulate gravitational perturbances and
orbital changes of minor bodies passing major planets.

\section{Precession and Nutation}
\label{sec:Accuracy:Precession}

Since v0.14.0, Stellarium computes the orientation of earth's axis
according to the IAU2006 Precession in a long-time variant developed
by \citet{2011AA:Vondrak}, and IAU2000B Nutation~\citep{Nutation:IAU2000B}.
This also now allows proper depiction of the changes in ecliptic obliquity
and display of ``instantaneous precession circles'' around the
ecliptic poles. These circles are indeed varying according to
ecliptical obliquity. Nutation is only computed for about 500 years
around J2000.0. Nobody could have observed it before 1609, and it is
unclear for how long this model is applicable.

\section{Planet Axes}
\label{sec:Accuracy:PlanetAxes}

Orientation for the other planets is still simplified. Future versions
should implement modern IAU-approved models.\index{IAU} 

\section{Eclipses}
\label{sec:Accuracy:Eclipses}

The moon's motion is very complicated, and eclipse computations can be tricky. 
One aspect which every student of history, prehistory and archaeology should 
know but as it seems not every does, is at least a basic understanding of the 
irregular slow-down of earth's rotation known as $\Delta T$ (see section~\ref{sec:Concepts:DeltaT}). 

Eclipse records on cuneiform tablets go back to the 8th century BC, some Chinese 
records go back a bit further. Experts on Earth rotation have provided models 
for $\Delta T$ based on such observations. These are usually given as parabolic fit, 
with some recommended time span. Extrapolating parabolic fits to erratic curves too far into the past is dangerous, 
pointless and inevitably leads to errors or invalid results. Models differ by many hours (exceeding a whole day!)
when applied too far in the past. This means, there was certainly an eclipse, and latitude will be OK, 
but you cannot say which longitude the eclipse was covering. The probability to have seen a total 
eclipse at a certain interesting location will be very small and should by itself not be used as positive argument 
for any reasonable statement. This is not a problem of Stellarium, 
but of current knowledge. To repeat, you may find a solar eclipse in the 6th millennium BC, 
but you cannot even be sure which side of Earth could observe it!

\textbf{WANTED:} A model for $\Delta T$ that works in the Mesolithic! 

\section{The Calendar}
\label{sec:Accuracy:Calendar}

Stellarium follows the widely accepted convention that dates before the Gregorian calendar 
reform are given in the Julian calendar, introduced by \name[Julius]{Caesar} in 46~BC. 
This calendar introduces a leap day every fourth year, which is a bit too much. 
If you want to show summer solstice in the year -2500, you will notice it is not on June 21st. 
This is not an error of Stellarium -- the Gregorian reform had a reason! In prehistoric times, 
it may be more useful to work with solar ecliptic longitudes. For example, summer solstice is
when the Sun is at $\lambda=90\degree$.

\name[Julius]{Caesar} introduced a Solar calendar to the Roman empire
in 46~BC but was assassinated in March 44~BC=-43 (see below).  His
simple 4-year leap rule was then misapplied by the high priests of
Rome by adding a leap day every third year.  \name{Augustus} fixed the
calendar by omitting a few leap days, and only 8~AD was the first correctly applied leap year as
we know them \citep[\S186, \S189]{Ginzel:ChronologieII}. Be careful if
you are working in Augustean and earlier times in the Roman world,
maybe even related to named dates in the pre-Julian Roman calendar:
Stellarium does not model these issues! Better find independent
conversion methods to Julian Day Number (JD), and use that.

Some authors (esp.\ for Mesoamerican dates) give dates in the proleptic Gregorian calendar 
(which means, they apply the Gregorian leap year rules before 1582). This is a convention, they may of course do that, 
but you must do the conversion to the Julian calendar to visualize the sky with Stellarium. 
Hopefully they also give Julian Day numbers (JD), you can use them with Stellarium. 
JD numbers are independent from calendar systems. 

Some have noted a difference of one between dates given as BC versus Stellarium's negative years. 
Note that historical chronology does not know a year Zero! (Try to write this date in Roman numbers!) 
Year 1~AD was preceded by year 1~BC. Astronomers want to simplify computations and therefore 
use a year 0 (equivalent to 1~BC). It is simple: 1~BC=0, 2~BC=-1, \ldots. 100~BC=-99. 
Stellarium is an astronomy program. Just accept this.





%%% Local Variables: 
%%% mode: latex
%%% TeX-PDF-mode: t
%%% TeX-master: "guide"
%%% End: 

