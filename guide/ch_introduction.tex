% Status info:
% M. Gates	2006-2009
% A. Wolf	2011-2015
% ArdWar	2012
% B. Gerdes	2013
% Additions inserted from wiki 2015-12-26
% Content OK for 0.14+.
% 2016-07: Historical notes added by Fabien.
% TODO: typo&grammar check


\chapter{Introduction}
\label{ch:Introduction}

\emph{Stellarium} is a software project that allows people to use their
home computer as a virtual planetarium. It calculates the positions of
the Sun and Moon, planets and stars, and draws how the sky would look to
an observer depending on their location and the time. It can also draw
the constellations and simulate astronomical phenomena such as meteor
showers or comets, and solar or lunar eclipses.

Stellarium may be used as an educational tool for teaching about the
night sky, as an observational aid for amateur astronomers wishing to
plan a night's observing or even drive their telescopes to observing
targets, or simply as a curiosity (it's fun!). Because of the high
quality of the graphics that Stellarium produces, it is used in some
real planetarium projector products and museum projection setups. Some
amateur astronomy groups use it to create sky maps for describing
regions of the sky in articles for newsletters and magazines, and the
exchangeable sky cultures feature invites its use in the field of
Cultural Astronomy research and outreach.

Stellarium is under continuous development, and by the time you read
this guide, a newer version may have been released with even more
features than those documented here. Check for updates to Stellarium at
the Stellarium website\footnote{\url{https://stellarium.org}}.

If you have questions and/or comments about this guide, or about
Stellarium itself, visit the Stellarium site at
GitHub\footnote{\url{https://github.com/Stellarium/stellarium}} or our
Google Groups
forum\footnote{\url{https://groups.google.com/forum/#!forum/stellarium}}.


\section{Historical notes}
\label{sec:Introduction:HistoricalNotes}

Fabien Ch\'ereau started the project during the summer 2000, and throughout
the years found continuous support by a small team of enthusiastic developers.

Here is a list of past and present major contributors sorted roughly by date of
arrival on the project:
\newpage
\begin{description}
\item[Fabien Ch\'ereau] original creator, maintainer, general development
\item[Matthew Gates] maintainer, original user guide, user support, general development
\item[Johannes Gajdosik] astronomical computations, large star catalogs support
\item[Johan Meuris] GUI design, website creation, drawings of our 88 Western constellations
\item[Nigel Kerr] Mac OSX port
\item[Rob Spearman] funding for planetarium support
\item[Barry Gerdes] user support, tester, Windows support. Barry
  passed away in October 2014 at age 80. He was a major contributor on
  the forums, wiki pages and mailing list where his good will and
  enthusiasm is strongly missed. \ifthenelse{\equal{\StelVersion}{0.15.0}}{Version 0.15 of Stellarium is
  dedicated in his memory.}{} RIP Barry.
\item[Timothy Reaves] ocular plugin
\item[Bogdan Marinov] GUI, telescope control, other plugins
\item[Diego Marcos] SVMT plugin
\item[Guillaume Ch\'ereau] display, optimization, Qt upgrades, HiPS surveys
\item[Alexander Wolf] maintainer, DSO catalogs, AstroCalc module, user guide, general development
\item[Georg Zotti] astronomical computations, Scenery 3D plugin, ArchaeoLines and Calendars plugins, general development, user guide, user support
\item[Marcos Cardinot] MeteorShowers plugin
\item[Florian Schaukowitsch] Scenery 3D plugin, Remote Control plugin, RemoteSync plugin, OBJ rendering, Qt/OpenGL internals
\item[Teresa Huertas Rold\'an] Planetary nomenclature
\item[Jocelyn Girod] Observing Lists
\item[Ruslan Kabatsayev] ShowMySky Skylight model (based on Bruneton's model) 
\item[Worachate Boonplod] Eclipse computations
\end{description}

Unfortunately time is evolving, and most members of the original
development team are no longer able to devote most of their spare time
to the project (some are still available for limited work
which requires specific knowledge about the project).

As of 2017, the project's maintainer is Alexander Wolf, doing most
maintenance and regular releases. He has also introduced the AstroCalc module.
Other new features are contributed
mostly by Georg Zotti and his team focusing on extensions of
Stellarium's applicability in the fields of historical and cultural
astronomy research (which means Stellarium is getting more accurate)
and outreach (making it usable for museum installations), but also on
graphic items like comet tails, light pollution artwork or the Zodiacal Light. 

\vspace{1\baselineskip}
\noindent A detailed track of development can be found in the
\file{ChangeLog} file in the installation folder. A few important
milestones for the project:
\begin{description}
\item[2000] first lines of code for the project
\item[2001-06] first public mention (and user feedbacks!) of the
  software on the French newsgroup \texttt{fr.sci.astronomie.amateur} 
  \footnote{\url{https://groups.google.com/d/topic/fr.sci.astronomie.amateur/OT7K8yogRlI/discussion}}
\item[2003-01] Stellarium reviewed by Astronomy magazine
\item[2003-07] funding for developing planetarium features (fisheye projection and other features)
\item[2005-12] use accurate (and fast) planetary model
\item[2006-05] Stellarium ``Project Of the Month'' on SourceForge
\item[2006-08] large stars catalogs
\item[2007-01] funding by ESO for development of professional astronomy extensions (VirGO)
\item[2007-04] developers' meeting near Munich, Germany
\item[2007-05] switch to the Qt4 library as main GUI and general purpose library
\item[2009-09] plugin system, enabling a lot of new development
\item[2010-07] Stellarium ported on Maemo mobile device
\item[2010-11] artificial satellites plugin
\item[2014-06] high quality satellites and Saturn rings shadows, normal mapping for moon craters
\item[2014-07] v0.13.0: adapt to OpenGL evolutions in the Qt5 framework, now requires more modern graphic hardware than earlier versions
\item[2015-04] v0.13.3: Scenery 3D plugin
\item[2015-10] v0.14.0: Accurate precession
\item[2016-07] v0.15.0: Remote Control plugin
\item[2016-12] v0.15.1: DE430\&DE431, AstroCalc, DSS layer, and Stellarium acting as SpoutSender
\item[2017-06] v0.16.0: Remote Sync plugin, polygonal OBJ models for minor bodies, RTS2 telescope support
\item[2017-09] v0.16.1: Standard and extended DSO catalog, new subcatalogues for DSO
\item[2017-12] v0.17.0: Nomenclature labels for planets and moons, INDI telescope support
\item[2018-03] v0.18.0: Multiple image surveys
\item[2019-12] v0.19.3: ASCOM telescope support 
\item[2020-09] v0.20.3: Accurate seasons' beginnings
\item[2021-03] v0.21.0: Accurate planet rotation (Libration, central meridians, subsolar points, \ldots) 
\item[2021-09] v0.21.2: Annual aberration, DE440\&441. Accuracy goals reached.
\item[2022-10] \textbf{Stellarium 1.0}: Switch to Qt6, ShowMySky skylight model, Eclipse details. 
\item[2025-03] v25.1: New star catalog, parallax and revolving binary stars, new Skyculture file format. 
\end{description}

\section{Version Numbers}
Since its inception, Stellarium's version number had started with 0 to
indicate ``work in progress''. The numbers after that have followed a
year.release convention with approximately seasonal releases since
2018. (For example, 0.18.2 was the third release in 2018 which
appeared around autumn equinox, after 0.18.0 and 0.18.1.) 
If needed urgently, there can be additional releases which however break the simple season count.

A software version number of 1.0 signifies a milestone of some sort,
like completion of a particular original feature set, usability, or
stability. With the completion of the original accuracy issues in 2021
we felt it was time to finally change version number to 1. However, a
major technical update in the underlying Qt framework also forced some
technical changes upon us to ensure Stellarium will remain working in
the later 2020s, and therefore we decided to base the 1.* series on
the new version of Qt.

Qt6 does not support old hardware, particularly 32-bit systems will
not be able to run 1.* versions.  Therefore, we are going to keep both
series alive for those with older hardware. Thanks to only minor
differences in the underlying frameworks, both series should retain
feature equivalence. The only notable difference for now lied in the
Scripting functionality (see chapter~\ref{ch:scripting}).
Our ``internal'' version numbers therefore indicated which version of Qt was used:
series 0.* continued to be based on Qt5, and series 1.* was based on Qt6.

In 2023 we've changed the numbering scheme again to avoid new misunderstandings. 
Our ``internal'' version has a standard of 3 components\footnote{On Windows 
it has 4 components, where the last one is always 0.}, where the first indicates 
the two last digits of the year of release, the second is the release within 
that year (0 is used before first release, from January to March) and the last 
one can be 0 for releases, or the day number of the current year for snapshots. 
Our short (``public'') version has 2 components: year 
and number of the release. Example: the first release in year 2023 has 
version 23.1.0 and short (public) version 23.1, the series has number 23.0.

In addition to the version number we are adding the name component \texttt{qtX}, 
where \texttt{X} is \texttt{5} or \texttt{6} respectively and marks the 
major version of Qt which is used as the base for the package.

\section{Acknowledgements}
Stellarium has been kindly supported by ESA in their Summer of Code in
Space initiatives, which so far has resulted in better planetary rendering
(2012), the Meteor Showers plugin (2013), the web-based remote
control and an alternative solution for planetary positions based on
the DE430/DE431 ephemeris (2015), the RemoteSync plugin and OBJ models (2016), 
and the planet nomenclature labels (2017). Some of Georg's work of 2015--2023 was
supported by the Ludwig Boltzmann Institute for Archaeological
Prospection and Virtual Archaeology, Vienna, Austria. 

Stellarium receives funding on OpenCollective. We thank all our sponsors and backers 
who help us maintaining the software and expanding it with yet more features to come.

\section{Scientific use}

Stellarium has gained wide popularity in the research areas of
cultural astronomy. Please note limitations of Stellarium mentioned in
Appendix~\ref{ch:Accuracy}. If you used Stellarium in scientific work,
please cite our overview paper \citep{Zotti-etal:JSA2020.6.2} or those
mentioned in the respective feature descriptions.

\section{About this User Guide}
This guide is based on the user guide written by Matthew Gates for
version 0.10 around 2008. The guide was then ported to the Stellarium
wiki and continuously updated by Barry Gerdes and Alexander Wolf up to version 0.12. 
Unfortunately, some new features were not properly documented in the wiki, and generally, 
without Barry the wiki started to fall out of sync with
the actual program.  In late 2015 we (Alexander and Georg) started porting the texts
back to \LaTeX{} and updated and added information where necessary, 
or wrote new chapters for the features which were introduced in the last years. 
We feel that a single book is the better format for offline
reading. The PDF version of this guide has a clickable table of
contents and clickable hyperlinks.

These new editions of the Guide (since v0.15) do not contain notes about using
earlier versions than 0.13 or using very outdated hardware.
%% (No such margin notes in 1.0!)
%New features are marked with a version number like v0.15.2 in the margin. 
Some references to previous versions may still be made for completeness, 
but if you are using earlier versions of Stellarium 
for particular reasons, please use the older guides.


%%% Local Variables: 
%%% mode: latex
%%% TeX-PDF-mode: t
%%% TeX-master: "guide"
%%% End: 
