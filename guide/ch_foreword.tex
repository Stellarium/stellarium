% Status info:
% 2025-03 Foreword for V25.1

\chapter*{Foreword for Stellarium 25.1}
\label{ch:Foreword}


Almost  twenty-five years have elapsed since Fabien Ch\'ereau typed the
first lines of code of a desktop planetarium which should be based on
modern algorithms from computer graphics to provide a trustworthy
simulation of the night sky, utilizing OpenGL real-time graphics technology which best
runs on dedicated graphics hardware. 
%
%What evolved from these first
%attempts has turned into a beautiful program which has been
%
The beautiful program that evolved since then has been 
downloaded millions of times to run on desktop computers and notebooks
all around the planet, on various operating systems and in dozens of
languages. The first team of developers also developed ways to control Goto 
telescopes which became a popular feature. In the years to come this was occasionally 
extended by further external contributors to provide support for INDI and RTS2 on Linux and ASCOM on Windows, 
but keeping the systems working by occasionally upgrading the protocols was left to 
the maintainers who may be limited in their access to instruments or just have 
not even enough time to go observing themselves.  


Only in late 2022 we were confident enough in the accuracy of our implementations of planetary 
algorithms and various coordinate refinements that we have finally dropped the zero version number and 
released version 1.0, and started using the year number as basis for later versions. 
On this occasion we also moved forward to upgrading the foundational Qt GUI framework from Qt5 to Qt6 
to be prepared for upgrades in the next years. To keep compatibility with older PCs, 
we kept providing  builds based on the aging Qt5 framework. 
However it turned out that the new Qt6 framework and the widely used telescope 
control framework ASCOM~6 apparently don't play nicely together. We saw several reports about crashes coming in, 
but were not able to really solve the issue, requiring us to finally disable support on Qt6-based builds. 
The builds on the older Qt5 framework seemed to always work nicely. 

%What we don't have currently is expertise in Goto telescopes. 


With only three core developers working part-time at best and otherwise busy in other projects or daytime jobs, 
we cannot do giant leaps, but the program has 
evolved nicely in other areas like the AstroCalc dialog that becomes richer in every version. 



While many improvements had involved  refinements for planetary positions and modelling of observing conditions, 
other users were more interested in star positions over millennia. 
The star catalog we have used since 2006, which was based on the 
then best positional measurements made by ESA's Hipparcos satellite,  was discovered to have 
a systematic error that caused slightly wrong coordinates millennia from the present. 
Also, binary stars could be observed to fly apart, because their motions around each other were not modelled. 

Just a few weeks before release 24.4 a new developer, Henry Leung, showed up and offered a new star catalog for Stellarium. 
We needed a few weeks to test, check for inconsistencies and finding how to best link this to our existing infrastructure, 
but days after 24.4 was out, we fully embraced this new catalog, now a combination of ESA's latest 
and best astrometry satellite Gaia's DR3 catalog with -- still -- the Hipparcos data, 
required for the brightest stars which were too bright for Gaia to measure. 
Computers have become more powerful since 2006, and so this new catalog comes with new features: While previously 
we just showed numerical data for parallax, now you can observe the stars' actual parallaxes as you move around the Sun! 
If needed in demonstrations, you can exaggerate the effect.
Proper motions are now modelled accurately, and even stars' distances change, so you can observe the brightness change of 
close stars like Barnard's as they pass our Solar system. And, what some may like best, 
binary stars with known orbits move around each other while doing all this. 
So, fasten your seat belts and observe, for example, the fast spatial motion of 61 Cygni, affected by annual parallax and aberration, 
and observe the two components dancing around each other over centuries! Add a HiPS survey displayed behind the stars, 
and see when the actual plates were taken so that the fast running stars fall in place. 
Observe the combination of motions and you will get a higher appreciation for how difficult 
finding stellar distances from positional measurements actually must have been for the pioneers in the field in the 19th century.
%That each 
Each star is now identified by its catalog number in Gaia DR3 as an added benefit. 
%This new catalog also made the MissingStars plugin obsolete and solved many reports about ``missing stars''. 

If you need to run an older version of Stellarium and have downloaded the higher-level star catalogs, 
these will still be available for running these older versions, but you will want to download the new star catalogs as soon as possible. 


We had actually planned for a different large development to begin in early 2025. Fabien's newer projects, Stellarium-web and Stellarium-mobile, 
are utilizing a more modern data format for their skyculture descriptions: JSON. Also in our core team, 
we were convinced that the file formats used so far are not flexible enough to model the many aspects of the rich astronomical 
heritage of cultures found all over the world, and we had already collected ideas for years. 
In a scientific symposium in Jena in late 2023 that was also supported by Stellarium's donation 
fund we were able to work out a few definitions and requirements of how Stellarium can help scientists collecting, displaying and communicating 
all this. Starting with this version 25.1, we are therefore using a completely new file format for the sky 
cultures which should be compatible with all programs of the Stellarium family. 
Ruslan, who also optimizes the OpenGL rendering in countless little improvements, did a great job converting all existing data, 
especially the translation into dozens of existing languages, so that hopefully no vocabulary that was already translated was lost.  
See chapter \ref{ch:SkyCultures} for details about the new features as we use them now, 
and expect more features to come in this area as we continue to work during this year. 

Coming back to ASCOM, while testing with version 7 we could not trigger the bad crash that was reported before, 
therefore we hope that re-enabling ASCOM in this release will work for our users with actual instruments. 
Your feedback is important to find this out, and if you are familiar with programming, we can need support in this part of the program. 

We want to thank all sponsors and backers of the project as meanwhile we have collected substantial funds to support development 
also in work hours. We like working on this project, but also must pay the bills, so please keep up your kind support! 


\iffalse
%% Aargh, these only come in 25.2...
If that was not enough, yet another contributor, Wang Siliang, has fulfilled a long-standing 
desire for quite a few astrophotographers.  He created a plugin with which you can place your own 
photos inside Stellarium, with just a few mouseclicks and connection to astrometry.net.

Have you ever wondered what the actual size of the focal area was for the modern generation of automated survey telescopes?  
Thanks to yet another external contribution by Josh Meyers, a professional astronomer involved in commissioning 
the Vera C. Rubin telescope, his plugin that he developed and used during this commissioning can show the sensor grid of 
this and other telescopes, and you can even configure your own. 
\fi


Work on other things has started, but they are scheduled for completion only in upcoming versions. 
So, Stellarium is going strong as we approach the  25th anniversary of this project!


In the name of all prior and current developers we wish you much enjoyment with
this and future versions of the Stellarium desktop planetarium!

\begin{flushright}
Georg Zotti and Alexander Wolf, March 2025
\end{flushright}

%\vspace{2\baselineskip}




%%% Local Variables: 
%%% mode: latex
%%% TeX-PDF-mode: t
%%% TeX-master: "guide"
%%% End: 
