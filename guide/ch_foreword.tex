% Status info:
% 2025-06 Foreword for V25.2

\chapter*{Foreword for Stellarium 25.2}
\label{ch:Foreword}

Only a few months ago, in the previous foreword, we could announce big updates with the 
new Gaia-based star catalog and the new file format for Skycultures. But there are more surprises!


Just as we prepared version 24.4 last December and had first trials with the new star catalog, a new contributor 
presented his plugin. Josh Meyers has developed a \emph{Stellarium plugin for} his \emph{commissioning work at the new
Vera C. Rubin observatory's Simonyi Survey Telescope}\footnote{Wording clarified since the 25.2 User Guide.}. 

Some messages have to be read more than once to sink in. 
\emph{Stellarium is used during commissioning of the survey telescope at the Vera C. Rubin observatory.}
This is doubtlessly one of the most jaw-dropping astronomical instruments to become 
operational in this decade, capturing repeated surveys of the whole sky that take only 3 days with its 
gigantic 3.2 Gigapixel sensor array, and scheduled to producing 20 Terabytes of data. Per night! 
Josh's plugin just didn't make it into 25.1, but is now integrated in version 25.2 (see \ref{sec:plugins:MosaicCamera}), 
just in time as first-light images have been announced for the week after our June 2025 solstice release.
It is great to see that Stellarium is used in the control room of this magnificent telescope!  

You may not run your own survey telescope, but with this plugin you can display several preconfigured sensor areas, 
and even configure your own dream system. I can imagine it can help in outreach, 
when the fields of view for professional telescopes can be demonstrated in Stellarium's sky rendition.


Another new plugin that certainly more of you have been waiting for: Wang Siliang has made 
integrating your own astrophotos into Stellarium so much easier! In his NebulaTextures plugin (see \ref{sec:plugins:NebulaTextures}), 
just upload an image to Astrometry.net and receive its accurate location within seconds, then display and finetune, 
and store in your private collection for running in future Stellarium sessions. You can also manually adjust positions, 
which might be useful to integrate your sketches and paintings (which Astrometry.net may not like). 
We would like to see a collection of sketches that more represent what observers can see to use as alternative 
to the ever more stunning photos.



While these new plugins will certainly find friends soon, the core team and our tireless translators were busy 
in continuing what was first visible in version 25.1. 
Stellarium has become the de-facto most relevant simulation environment for cultural astronomy, 
be it archaeoastronomical simulation of old standing stone rows under simulated skies of yesteryear, 
or explaining the constellations and how they can be used in the lives of the many cultures living on our planet. 

The new skyculture file format has grown far, closing the deficiencies of the old system that we and some of you had identified in the last years. 
Stellarium can now display multiple linguistic variants of constellation and star labels, can show ``single star constellations'' and ``dark constellations'' 
formed by dust clouds in the Milky Way, can properly identify Morning and Evening star, and for those cultures where it is relevant, 
illustrate the zodiac and lunar stations or lunar mansions in various culture-dependent versions, and some more. 
All details of the new format can be found in chapter~\ref{ch:SkyCultures}.

If something about star names is not found in a skyculture description, chances are such data are 
or will soon be available in a new online project by Susanne M. Hoffmann, the All Skies Encyclopaedia (ASE). 
Stellarium's OnlineQueries plugin (\ref{sec:plugins:OnlineQueries}) can now retrieve data for stars and constellations from ASE into its integrated web browser. 


So, Stellarium is going strong as we approach the  25th anniversary of this project! 
Twenty-five years ago, in summer of 2000,  Fabien Ch\'ereau typed the
first lines of code of this desktop planetarium to try something with astronomy and computer graphics. 
What has grown from these first lines is just awesome.


However, developing and maintaining a project of this size needs dedication, especially when done as unpaid side occupation. 
Large progress needs funding which is sometimes hard to find. 
We want to thank all sponsors and backers of the project as meanwhile we had collected substantial funds to support this development 
also in work hours. We like working on this project, but also must pay our bills, so please keep up your kind support! 




In the name of all prior and current developers we wish you much enjoyment with
this and future versions of the Stellarium desktop planetarium!

\begin{flushright}
Georg Zotti and Alexander Wolf, June 2025
\end{flushright}

\vspace{2\baselineskip}

\section*{\ldots and on to V25.3\ldots}
Over the summer we have further refined many details around skycultures. 
Contributors have provided and updated many details especially in Arab, Chinese and even Mesopotamian skycultures. 
Don't imagine, just see Cuneiform labels in the sky!

We have also updated the DSO catalog and further enhanced the AstroCalc module. 
And we dare to increase the maximum range of dates to $\pm 200.000$ years to observe stellar proper motion. 
(Please note that planet positions by far cannot be modelled in that long time range!)

Clear skies!

\begin{flushright}
Georg Zotti and Alexander Wolf, September 2025
\end{flushright}

\vspace{2\baselineskip}

\section*{\ldots and V25.4}
Our refinements around skycultures continued. Our skyculture researcher, Susanne M. Hoffmann, and her international peers and students 
have created no less than 4 new skycultures and refined several more.
One is a remarkable Chinese adaptation of the standard IAU constellations which replaces the internationally known constellations by 
characters from the \emph{Journey to the West}, a well-known story from classical Chinese literature. While we cannot provide the whole story 
and surely all of us not familiar with the story will miss a lot, the cute artwork alone, together with the standard names available in the ``Byname'' labels, 
may bring a fresh view to the old figures. 

Some time ago, a user asked why our speech output was so bad. To be honest, I was not even aware of the current 
state of operating system support to output audibly whatever text is shown in the user interface. 
But indeed there are places where speech output may be helpful even for users without visibility problems. 
When you have selected a constellation, you may now hear whatever the skyculture author had to say about it. 
Spoken Information is also available for stars, planets and deep-sky objects.
Also the introductory sections of skycultures, and landscape descriptions can be read out. 
Of course this needs texts translated to your user language, so our thanks again go to our team of translators, 
we know the amount of text in the sky cultures is quite a challenge. It also needs support of your language by the 
speech engines, this is out of our hands.  
Some details may still sound a bit rough, we are also still learning here, so further improvements are likely.
And: If you are expert in the field of accessibility and want to redesign and extend this to become fully functional, we welcome your support!

You may also see this User Guide has fewer pages than the last. We have slightly redesigned the appearance and reduced font size, but the actual content has again grown. 

Clear skies!

\begin{flushright}
Georg Zotti and Alexander Wolf, December 2025
\end{flushright}


%%% Local Variables: 
%%% mode: latex
%%% TeX-PDF-mode: t
%%% TeX-master: "guide"
%%% End: 
