
\chapter{Surveys}
\label{ch:surveys}
\chapterauthor*{Guillaume Chéreau}

\section{Introduction}
\label{sec:surveys:introduction}


A \indexterm{sky survey} \newFeature{v0.18.0} is a map of the sky stored as a hierarchical set of a potentially
large number of smaller images (called tiles).  The advantage compared to a
regular texture is that we need to render only the visible tiles of a potentially gigantic image at the
lowest resolution needed.  This is particularly interesting for rendering
online images that can be stored on a server, while the client only has
to download the parts he currently uses.

Since version 0.18.0, Stellarium added some preliminary support for loading and
rendering online surveys in the \indexterm{Hierarchical Progressive Surveys} (HiPS) format,
developed by the \indexterm{International Virtual Observatory Alliance}.
A full description of the format can be found on the IVOA website\footnote{%
\url{http://www.ivoa.net/documents/HiPS/20170519/REC-HIPS-1.0-20170519.pdf}}.

\section{Hipslist file and default surveys}
\label{sec:surveys:hipslistFile}

Hipslist files are text files used to describe catalogs of HiPS surveys.  The
full specification is part of the HiPS format, and looks like that:

\begin{configfileScr}
# Example of a hipslist file.
# Date: 2018-03-19

hips_service_url  = https://data.stellarium.org/surveys/callisto
hips_release_date = 2018-03-18T14:01Z
hips_status       = public mirror clonableOnce
\end{configfileScr}

As of v0.18.0, by default Stellarium tries to load HiPS from two sources:
\url{http://alaskybis.unistra.fr/hipslist}\footnote{Since version 0.20.3 the first URL 
was changed to \url{http://alasky.u-strasbg.fr/MocServer/query?*/P/*&get=record} due to changes in HiPS services} and
\url{https://data.stellarium.org/surveys/hipslist}.
This can be changed with the hips/sources options in the configuration
file (see also section~\ref{sec:config.ini:hips}).  For example, you can add your
own private HiPS surveys by running a local webserver and adding:

\begin{configfile}
[hips]
sources/1/url = http://alasky.u-strasbg.fr/MocServer/query?*/P/*&get=record
sources/2/url = https://data.stellarium.org/surveys/hipslist
sources/3/url = http://localhost/Stellarium/hips/hipslist
sources/size  = 3
\end{configfile}

\section{Solar system HiPS survey}

Though not specified in the HiPS standard, Stellarium recognises HiPS surveys
representing planet textures, as opposed to sky surveys.  If the
\texttt{obs\_frame} property of a survey is set to the name of a planet or
other solar system body, Stellarium will render it in place of the default
texture used for the body.


\section{Digitized Sky Survey 2 (TOAST Survey)}
\sectionauthor*{Georg Zotti, Alexander Wolf}
\label{sec:TOAST}

The older way to provide a tessellated all-sky survey uses the TOAST
encoding\footnote{Please see
  \url{http://montage.ipac.caltech.edu/docs/WWT/} for details}.
Stellarium provides access to the Digitized Sky Survey 2, a
combination of high-resolution scans of red- and blue-sensitive
photographic plates taken in 1983--2006 at Palomar Observatory and the
Anglo-Australian Observatory.\footnote{The original data are available
  at \url{https://archive.stsci.edu/cgi-bin/dss_form}.}

To enable access to the DSS layer, see
section~\ref{sec:gui:configuration:extras} and enable the DSS
button. Then just press that DSS button in the lower button bar, wait
a moment, zoom in and enjoy!

\subsection{Local Installation}

This display normally requires access to the Internet. However, in
some situations like frequent and extensive use in fixed observatories,
or use in the field when Internet connection is not possible, you can
download all image tiles to your local harddisk for local use. Please be
considerate, don't waste bandwidth, and do this only if you really need it.

The images are stored in subdirectories that increase in size and
number of files (see Table~\ref{tab:TOAST:levels}). Zooming in loads
the next level if available.

For partial downloads, you can limit the maximally used level (e.g. 9
or 10). The difference from level 10 to 11 is really hardly
noticeable, yet level 11 contains almost 75\% of all data. On a small
system like Raspberry Pi 3, you may run into troubles with too little
texture memory when level is more than 7.

Another issue: the level 11 subdirectory holds over 4 million
files. Windows \program{Explorer} is not optimized to open and display
this number of files and will take a long time to open. Just unpack
this archive, but don't access the folder with \program{Explorer}.

If all that does not discourage you, you can download the archives at
\url{https://dss.stellarium.org/offline/}.

Then add a few entries to the \texttt{[astro]} section in
Stellarium's \file{config.ini}.  On Windows, if you have a harddisk
\file{T:} with path \file{T:\textbackslash{}StelDSS} that contains the
unpacked image subdirectories 0, 1, 2, \ldots, 10, the section may look like
\begin{configfile}
[astro]
toast_survey_directory =StelDSS
toast_survey_host=T:/
toast_survey_levels=10
\end{configfile}
On Linux, with the files stored in \file{/usr/local/share/Stellarium/StelDSS}, the same section could look like
\begin{configfile}
[astro]
toast_survey_directory =usr/local/share/Stellarium/StelDSS
toast_survey_host=/
toast_survey_levels=10
\end{configfile}

\begin{table}[htbp]
  \centering
\begin{tabular}{rrr}
Level & No of files & Filespace (kB)\\
0     &          1  &         32\\
1     &          4  &        124\\
2     &         16  &        472\\
3     &         64  &      1.820\\
4     &        256  &      7.172\\
5     &      1.024  &     27.920\\
6     &      4.096  &    109.628\\
7     &     16.384  &    429.800\\
8     &     65.536  &  1.661.808\\
9     &    262.144  &  6.119.092\\
10    &  1.048.576  & 20.250.216\\
11    &  4.194.305  & 83.583.724
\end{tabular}
\caption{Number of files and storage requirements for local DSS TOAST installation}
\label{tab:TOAST:levels}
\end{table}




%%% Local Variables: 
%%% mode: latex
%%% TeX-master: "guide"
%%% End: 
